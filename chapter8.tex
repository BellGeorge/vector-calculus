\chapter{Integration of Forms} \label{ch:integralofforms}

\section{Differential Forms}


\begin{df}
A \negrito{$k$-differential form field in $\reals^n$} is an expression
of the form
$$\omega =  \dsum _{ 1 \leq j_1 \leq j_2 \leq \cdots \leq j_k \leq n} a_{j_1j_2\ldots j_k}\d{x_{j_1}} \wedge \d{x_{j_2}} \wedge \cdots \d{x_{j_k}},
$$where the $a_{j_1j_2\ldots j_k}$ are differentiable
functions in $\reals^n$.
\end{df}

A $0$-differential form in $\reals^n$ is simply a differentiable
function in $\reals^n$.


\begin{exa}
$$g(x, y, z, w) = x + y^2 + z^3 + w^4$$is a 0-form in $\reals^4$.
\end{exa}
\begin{exa}
An example of a  1-form field in $\reals^3$ is
$$\omega = x\d{x}+ y^2\d{y} + xyz^3\d{z}.$$
\end{exa}
\begin{exa}
An example of a  2-form field in $\reals^3$ is
$$\omega = x^2\d{x}\wedge \d{y} + y^2\d{y} \wedge \d{z} + \d{z} \wedge \d{x}.$$
\end{exa}
\begin{exa}
An example of a  3-form  field in $\reals^3$ is
$$\omega = (x + y + z)\d{x}\wedge \d{y} \wedge \d{z}.$$
\end{exa}
We shew now how to multiply differential forms.
\begin{exa}
The product of the 1-form fields in $\reals^3$
$$\omega_1 = y\d{x}+ x\d{y},$$
$$\omega_2 = -2x\d{x}+ 2y\d{y},$$is
$$\omega_1 \wedge \omega_2 = (2x^2 + 2y^2)\d{x}\wedge \d{y}.$$

\end{exa}

\begin{df}Let $f(x_1, x_2, \ldots , x_n)$ be a $0$-form in
$\reals^n$. The \negrito{exterior derivative} $\d{f}$ of $f$ is

$$\d{f} = \dsum _{i = 1} ^n \dfrac{\partial f}{\partial x_i}\d{x_i}.     $$
Furthermore, if $$ \omega = f(x_1, x_2, \ldots , x_n) \d{x_{j_1}}
\wedge \d{x_{j_2}} \wedge \cdots \wedge \d{x_{j_k}}
$$is a $k$-form in $\reals^n$, the \negrito{exterior derivative} $\d{\omega}$ of $\omega$ is the $(k +
1 )$-form
$$ \d{\omega} = \d{f(x_1, x_2, \ldots , x_n)} \wedge \d{x_{j_1}} \wedge
\d{x_{j_2}} \wedge \cdots \wedge \d{x_{j_k}}.
                   $$
\end{df}
\begin{exa}
If in $\reals^2$, $\omega = x^3y^4$, then
$$\d (x^3y^4) = 3x^2y^4\d{x}+ 4x^3y^3\d{y}.$$
\end{exa}
\begin{exa}
If in $\reals^2$, $\omega = x^2y\d{x}+ x^3y^4\d{y}$ then
$$\begin{array}{lll}
\d{\omega} & = & \d (x^2y\d{x}+ x^3y^4\d{y}) \\
& = & (2xy\d{x}+ x^2\d{y})\wedge \d{x}+ (3x^2y^4\d{x}+ 4x^3y^3\d{y})\wedge \d{y} \\
& = & x^2 \d{y} \wedge \d{x}+ 3x^2y^4\d{x}\wedge \d{y} \\
& = & (3x^2y^4 - x^2)\d{x}\wedge \d{y} \\
\end{array}$$
\end{exa}

\begin{exa}
Consider the change of variables $x = u + v, y = uv$. Then
$$\d{x}= \d{u}+\d{v},$$
$$\d{y}  = v\d{u}+ u\d v,$$whence
$$\d{x}\wedge \d{y} = (u - v) \d{u}\wedge\d{v}.$$
\end{exa}
\begin{exa}
Consider the transformation of coordinates $xyz$ into $uvw$
coordinates given by
$$u = x + y + z,\  v = \dfrac{z}{y + z},\  w = \dfrac{y + z}{x + y + z}.$$
Then
$$\d{u}= \d{x}+ \d{y} + \d{z},$$
$$\d v = -\dfrac{z}{(y + z)^2}\d{y} + \dfrac{y}{(y + z)^2}\d{z},
$$
$$\d{w} =
-\dfrac{y + z}{(x + y + z)^2}\d{x}+ \dfrac{x}{(x + y + z)^2}\d{y} +
\dfrac{x}{(x + y + z)^2}\d{z}.$$ Multiplication gives
$$\everymath{\displaystyle}{\begin{array}{lll}\d{u}\wedge\d{v} \wedge \d{w} & = & \left(-\dfrac{zx}{(y + z)^2(x + y + z)^2} -
\dfrac{y(y + z)}{(y + z)^2(x + y + z)^2} \right. \\
& & \qquad \left. + \dfrac{z(y + z)}{(y + z)^2(x + y + z)^2} -
\dfrac{xy}{(y + z)^2(x + y + z)^2} \right)\d{x}\wedge \d{y} \wedge
\d{z}
\\ &
= &\dfrac{z^2 - y^2 - zx - xy}{(y + z)^2(x + y + z)^2}\d{x}\wedge
\d{y} \wedge \d{z}. \end{array}}$$
\end{exa}

\section{Integrating Differential Forms}
Let

\[\omega=\dsum_{i_1 < \cdots < i_k} a_{i_1,\dots,i_k}({\vector{x}})\,dx^{i_1} \wedge \ldots \wedge dx^{i_k}\]

be a differential form and $M$ a differentiable-manifold over 
which we
wish to integrate, where $M$ has the parameterization

\[M({\vector{u}})=(x^1({\vector{u}}),\dots,x^k({\vector{u}}))\]

for in the parameter $\vector{u}$ domain $D$ . Then defines the integral of the
differential form over as

\[\int_S \omega =\int_D \dsum_{i_1 < \cdots < i_k} a_{i_1,\dots,i_k}(M({\vector{u}})) \frac{\partial(x^{i_1},\dots,x^{i_k})}{\partial(u^{1},\dots,u^{k})}\,du^1 \cdots du^k,\]

where the integral on the right-hand side is the standard Riemann  integral over $D$, and

\[\frac{\partial(x^{i_1},\dots,x^{i_k})}{\partial(u^{1},\dots,u^{k})}\]

is the determinant of the Jacobian.



\section{Zero-Manifolds }
\begin{df}
A  {$0$-dimensional  oriented manifold of $\reals^n$} is simply
a point $\point{x}\in \reals^n$, with a choice of the  $+$ or $-$
sign. A general oriented $0$-manifold is a union of oriented points.
\end{df}
\begin{df}
Let $M = +\{\point{b}\} \cup -\{\point{a}\}$ be an oriented
$0$-manifold, and let $\omega$ be a $0$-form. Then
$$ \dint _M \omega = \omega (\point{b}) - \omega (\point{a}).   $$
\end{df}

\begin{rem}$-\point{x}$ has opposite orientation to $+\point{x}$ and
$$ \dint _{-\point{x}} \omega = -\dint _{+\point{x}} \omega .   $$
\end{rem}
\begin{exa}
Let $M = -\{(1,0,0)\} \cup +\{(1,2,3)\} \cup
-\{(0,-2,0)\}$\footnote{Do not confuse, say, $-\{(1,0,0)\}$ with
$-(1,0,0) = (-1,0,0)$. The first one means that the point
$(1,0,0)$ is given negative orientation, the second means that
$(-1,0,0)$ is the additive inverse of $(1,0,0)$. } be an oriented
$0$-manifold, and let $\omega = x + 2y + z^2$. Then
$$\dint _M\omega = -\omega ((1,0,0)) + \omega (1,2,3) - \omega (0,0,3) = -(1) + (14) - (-4) = 17.         $$
\end{exa}

\section{One-Manifolds}
\begin{df}
A  \textbf{$1$-dimensional  oriented manifold of $\reals^n$} is simply
an oriented smooth curve  $\Gamma\in \reals^n$, with a choice of a
$+$ orientation if the curve traverses in the direction of
increasing $t$,  or with a choice of a $-$ sign if the curve
traverses in the direction of decreasing $t$. A general oriented
$1$-manifold is a union of oriented curves.
\end{df}
\begin{rem}The curve $-\Gamma$ has opposite orientation to $\Gamma$ and
$$ \dint _{-\Gamma} \omega = -\dint _\Gamma \omega .   $$If $\vector{f}:\reals ^2 \to \reals
^2$ and if $\d{\vector{r}} = \colvec{\d{x}\\ \d{y}}$, the classical
way of writing this is $$  \dint _{\Gamma} \vector{f}\cdot
\d{\vector{r}}.
$$
\end{rem}
We now turn to the problem of integrating $1$-forms.
\begin{exa}
Calculate
$$\dint _\Gamma xy\d{x} + (x + y)\d{y}$$ where $\Gamma$ is the parabola
$y = x^2, \ x \in [-1; 2]$ oriented in the positive direction.
\end{exa}
\begin{solu} We parametrise the curve as $x = t, y = t^2$. Then
$$xy\d{x} + (x + y)\d{y} = t^3\d{t} + (t + t^2)\d{t^2} = (3t^3 + 2t^2)\d{t},    $$
whence
$$\begin{array}{lll}
\dint _{\Gamma} \omega & = & \dint _{-1} ^2 (3t^3 + 2t^2)\d{t} \\
& = & \left[\dfrac{2}{3}t^3  + \dfrac{3}{4}t^4\right] _{-1} ^2\\ &
= & \dfrac{69}{4}.
\end{array}$$
What would happen if we had given the curve above a different
parametrisation? First observe that the curve travels from
$(-1,1)$ to $(2,4)$ on the parabola $y = x^2$. These conditions
are met with the parametrisation $x = \sqrt{t} - 1, y =
(\sqrt{t}-1)^2$, $t\in [0; 9]$. Then
$$\begin{array}{lll}xy\d{x} + (x + y)\d{y} & = & (\sqrt{t}-1)^3\d{(\sqrt{t}-1)} + ((\sqrt{t}-1) + (\sqrt{t}-1)^2)\d{(\sqrt{t}-1)^2}\\
& = & (3(\sqrt{t}-1)^3 + 2(\sqrt{t}-1)^2)\d{(\sqrt{t} - 1)} \\
& = & \dfrac{1}{2\sqrt{t}}(3(\sqrt{t}-1)^3 +
2(\sqrt{t}-1)^2)\d{t},
\end{array}$$ whence
$$\begin{array}{lll}
\dint _{\Gamma} \omega & = & \dint _{0} ^9 \dfrac{1}{2\sqrt{t}}(3(\sqrt{t}-1)^3 + 2(\sqrt{t}-1)^2)\d{t} \\
& = &
\left[\frac{3t^2}{4}-\frac{7t^{3/2}}{3}+\frac{5t}{2}-\sqrt{t}\right]
_{0} ^9\\[1ex] & = & \dfrac{69}{4},\\
\end{array}$$as before.







\end{solu}

\begin{rem}
It turns out that if two different parametrisations of the same
curve have the same orientation, then their integrals are equal.
Hence, we only need to worry about finding a suitable
parametrisation.
\end{rem}







\begin{exa}
 Calculate the line integral $$\dint _{\Gamma} y\sin x\d{x}+ x\cos y\d{y}  ,$$ where $\Gamma$ is the line
segment from $(0, 0)$ to $(1, 1)$ in the positive direction.
 \end{exa}
 \begin{solu} This line has equation $y = x$, so we choose the parametrisation $x = y = t$. The integral is thus
 $$\begin{array}{lll}
\dint _{\Gamma} y\sin x\d{x}+ x\cos y\d{y}  & = & \dint _0 ^1
(t\sin t  + t\cos t)\d t \\
& = &  [t(\sin x - \cos t)]_0 ^1 - \dint _0 ^1
(\sin t  - \cos t)\d t\\
 & = & 2\sin 1 - 1,
 \end{array}$$upon integrating by parts.






\end{solu}
\begin{exa}
 Calculate the path integral $$\dint _{\Gamma} \dfrac{x + y}{x^2 + y^2} \ \ \d{y} + \dfrac{x - y}{x^2 +
y^2}\ \ \d{x}$$around the closed square   $\Gamma = ABCD$ with $A
= (1, 1)$, $B = (-1, 1)$, $C = (-1, -1)$, and $D = (1, -1)$ in the
direction $ABCDA$.
 \end{exa}
 \begin{solu}  On $AB$, $y=1, \d{y} = 0$, on $BC$, $x = -1, \d{x} =
 0$, on $CD$, $y = -1, \d{y} = 0$, and on $DA$, $x = 1, \d{x} =
 0$. The integral is thus
 \renewcommand{\arraystretch}{2}
 $$\begin{array}{lll}
\dint _{\Gamma} \omega & = & \dint _{AB} \omega + \dint _{BC} \omega
+
\dint _{CD} \omega +  \dint _{DA} \omega \\
& = & \dint ^{-1} _1 \dfrac{x - 1}{x^2 + 1}\ \d{x}+  \dint ^{-1} _1
\dfrac{y - 1}{y^2 + 1}\ \d{y} +  \dint _{-1} ^1 \dfrac{x + 1}{x^2 +
1}\ \d{x}+  \dint _{-1} ^1 \dfrac{y + 1}{y^2 + 1}\
\d{y} \\
& = & 4 \dint _{-1} ^1 \dfrac{1}{x^2 + 1}\ \d{x}\\
& = & 4\arctan x|_{-1} ^1 \\ & = & 2\pi .
 \end{array}$$





 \end{solu}
 \begin{rem}
When the integral is along a closed path, like in the preceding
example, it is customary to use the symbol $\ \doint _\Gamma$
rather than $\dint_\Gamma$. The positive direction of integration
is that sense that when traversing the path, the area enclosed by
the curve is to the left of the curve.
\end{rem}

\begin{exa}
 Calculate the path integral $$\doint _{\Gamma}  x^2\d{y} + y^2\d{x},$$where $\Gamma$ is the ellipse
 $\dis{9x^2 + 4y^2 = 36}$
traversed once in the positive sense. \end{exa} \begin{solu}
Parametrise the ellipse as $x = 2\cos t, y = 3\sin t, t\in [0;
2\pi]$. Observe that when traversing this closed curve, the area of
the ellipse is on the left hand side of the path, so this
parametrisation traverses the curve in the positive sense. We have
$$\begin{array}{lll}
\doint _{\Gamma} \omega & = & \dint _0 ^{2\pi} ((4\cos^2t)(3\cos t)
+
(9\sin t)(-2\sin t) )\d t \\
& = & \dint _0 ^{2\pi} (12\cos^3t - 18\sin^3t) \d t \\
& = & 0.
\end{array}$$





\end{solu}



\begin{df}
 Let $\Gamma$ be a smooth curve. The integral $$\dint \limits_\Gamma f(\point{x}) \norm{\d{\point{x}}}
 $$is called the \negrito{path integral of $f$ along $\Gamma$.}
\end{df}
\begin{exa}\label{exa:int_along_a_length}
Find $\dint \limits_\Gamma x \norm{\d{\point{x}}} $ where $\Gamma$
is the triangle starting at  $A:(-1,-1)$ to $B:(2,-2)$, and ending
in $C:(1,2)$.
\end{exa}\begin{solu} The lines passing through the given  points have
equations $L_{AB}: y = \dfrac{-x - 4}{3}$,  and $L_{BC}: y = -4x +
6$. On $L_{AB}$ $$ x\norm{\d{\point{x}}} = x\sqrt{(\d{x})^2 +
(\d{y})^2} = x\sqrt{1 + \left(-\dfrac{1}{3}\right)^2}\d{x} =
\dfrac{x\sqrt{10}\d{x}}{3},
$$and on $L_{BC}$
$$ x\norm{\d{\point{x}}} = x\sqrt{(\d{x})^2 +
(\d{y})^2} = x(\sqrt{1 + \left(-4\right)^2})\d{x} =
x\sqrt{17}\d{x}.
$$Hence
$$\begin{array}{lll}\dint \limits_\Gamma x \norm{\d{\point{x}}} & = & \dint \limits_{L_{AB}} x \norm{\d{\point{x}}}  + \dint \limits_{L_{BC}} x \norm{\d{\point{x}}}  \\
& = & \dint _{-1} ^2  \dfrac{x\sqrt{10}\d{x}}{3} + \dint _2 ^1
x\sqrt{17}\d{x} \\
& = &  \dfrac{\sqrt{10}}{2}  - \dfrac{3\sqrt{17}}{2}.
\end{array}
$$






\end{solu}

\vspace*{2cm}
\begin{figure}[htpb]

$$ \psset{unit=2pc}\renewcommand{\pshlabel}[1]{{\tiny
#1}}
\renewcommand{\psvlabel}[1]{{\tiny #1}} \psaxes(0,0)(-3,-3)(3,3)
\psdots[dotscale=1, dotstyle=*](-1,-1)(2,-2)(1,2)
\psline(-1,-1)(2,-2)(1,2)
$$ \vspace*{2cm}\hangcaption{Example \ref{exa:int_along_a_length}. } \label{fig:int_along_a_length}
\end{figure}

\section*{\psframebox{Homework}}
\begin{multicols}{2}\columnseprule 1pt \columnsep 25pt\multicoltolerance=900
\begin{problem}
Consider $\dint _C x\d{x}+y\d{y}$ and $\dint _C xy
\norm{\d{\point{x}}} $.
\begin{enumerate}
\item  Evaluate $\dint _C x\d{x}+y\d{y}$ where $C$ is the straight line path that starts at $(-1,0)$ goes to $(0,1)$ and ends at $(1,0)$, by parametrising this
path. Calculate also $\dint _C xy \norm{\d{\point{x}}} $ using this
parametrisation.
\item  Evaluate $\dint _C x\d{x}+yd{y}$ where $C$ is the semicircle that starts at $(-1,0)$ goes to $(0,1)$ and ends at $(1,0)$, by parametrising this
path. Calculate also $\dint _C xy \norm{\d{\point{x}}} $ using this
parametrisation.
\end{enumerate}
\begin{answer}
\noindent
\begin{enumerate}
\item  Let $L_1: y =x+1$, $L_2: -x+1$. Then
$$
\begin{array}{lll}
\dint _C x\d{x}+y\d{y} & = & \dint _{L_1} x\d{x}+y\d{y}+\dint _{L_2}
x\d{x}+y\d{y}\\
& = & \dint _{-1} ^1 x\d{x} (x+1)\d{x} + \dint _0 ^1 x\d{x}
-(-x+1)\d{x}\\
& = & 0.
\end{array}
$$
Also, both on $L_1$ and on $L_2$ we have
$\norm{\d{\point{x}}}=\sqrt{2}\d{x}$, thus
$$
\begin{array}{lll}
\dint _C xy \norm{\d{\point{x}}}& = & \dint _{L_1}xy
\norm{\d{\point{x}}}+\dint _{L_2}
xy \norm{\d{\point{x}}}\\
& = & \sqrt{2}\dint _{-1} ^1 x(x+1)\d{x} - \sqrt{2}\dint _0 ^1 x(-x+1)\d{x}\\
& = & 0.
\end{array}
$$

\item We put $x=\sin t$, $y = \cos t$, $t\in\lcrc{-\frac{\pi}{2}}{\frac{\pi}{2}}$. Then

$$
\begin{array}{lll}
\dint _C x\d{x}+y\d{y} & = & \dint _{-\pi/2} ^{\pi/2}
(\sin t)(\cos t)\d{t}-(\cos t)(\sin t)\d{t}\\
& = & 0.
\end{array}
$$
Also, $\norm{\d{\point{x}}}=\sqrt{(\cos t)^2+(-\sin
t)^2}\d{t}=\d{t}$, and thus
$$
\begin{array}{lll}
\dint _C xy \norm{\d{\point{x}}}& = & \dint _{-\pi/2} ^{\pi/2} (\sin
t)(\cos t) \d{t}\\
& = & \dfrac{(\sin t)^2}{2}\Big| _{-\pi/2} ^{\pi/2} \\
& = & 0.
\end{array}
$$
\end{enumerate}
\end{answer}
\end{problem}


\begin{problem}
\label{pro:path-integral}  Find $\dint _{\Gamma} x\d{x} + y\d{y} $
where $\Gamma$ is the path shewn in figure \ref{fig:path-integral},
starting at  $O(0,0)$ going on a straight line to $A\left(4\cos
\tfrac{\pi}{6}, 4\sin \tfrac{\pi}{6}\right)$ and continuing on an
arc of a circle to $B\left(4\cos \tfrac{\pi}{5}, 4\sin
\tfrac{\pi}{5}\right)$.
\begin{answer}
Let $\Gamma _1$ denote the straight line segment path from $O$ to
$A=(2\sqrt{3},2)$ and $\Gamma _2$ denote the arc of the circle
centred at $(0,0)$ and radius $4$ going counterclockwise from
$\theta=\dfrac{\pi}{6}$ to  $\theta=\dfrac{\pi}{5}$.

\bigskip

Observe that the Cartesian equation of the line  $\line{OA}$ is $y
=\dfrac{x}{\sqrt{3}}$. Then on $\Gamma _1$
$$x\d{x} + y\d{y} = x\d{x}+ \dfrac{x}{\sqrt{3}}\d{\dfrac{x}{\sqrt{3}}} = \dfrac{4}{3}x\d{x}.$$
Hence $$\dint _{\Gamma _1} x\d{x} + y\d{y} = \dint _0 ^{2\sqrt{3}}
\dfrac{4}{3}x\d{x} = 8.
$$

On the arc of the circle we may put $x=4\cos \theta$, $y = 4\sin
\theta$ and integrate from $\theta = \dfrac{\pi}{6}$ to $\theta =
\dfrac{\pi}{5}$. Observe that there
$$ x\d{x} + y\d{y} = (\cos \theta )\d{\cos \theta}  +(\sin\theta)\d{\sin \theta} = -\sin\theta\cos\theta\d{\theta}+\sin\theta\cos\theta\d{\theta}=0,$$
and since the integrand is $0$, the integral will be zero.

\bigskip

Assembling these two pieces,
$$\dint _{\Gamma} x\d{x} + y\d{y} = \dint _{\Gamma _1} x\d{x} + y\d{y} +\dint _{\Gamma _2} x\d{x} + y\d{y}=8+0=8.    $$


\bigskip

Using the parametrisations from the solution of  problem
\ref{pro:path-integral2}, we find on $\Gamma _1$ that
$$ x\norm{\d{\point{x}}} = x\sqrt{(\d{x})^2+ (\d{y})^2} =x\sqrt{1+\dfrac{1}{3}}\d{x}=\dfrac{2}{\sqrt{3}}x\d{x},  $$
whence
$$ \dint _{\Gamma _1} x\norm{\d{\point{x}}} = \dint _0 ^{2\sqrt{3}}
\dfrac{2}{\sqrt{3}}x\d{x} = 4\sqrt{3}.  $$ On $\Gamma _2$ that
$$ x\norm{\d{\point{x}}} = x\sqrt{(\d{x})^2+ (\d{y})^2} =16\cos \theta\sqrt{\sin^2\theta + \cos^2\theta}\d{\theta}=16\cos\theta\d{\theta},  $$
whence
$$ \dint _{\Gamma _2} x\norm{\d{\point{x}}} = \dint _{\pi /6} ^{\pi /5}
16\cos\theta\d{\theta} = 16\sin \dfrac{\pi}{5} -16\sin
\dfrac{\pi}{6}= 4\sin \dfrac{\pi}{5}-8.
$$Assembling these we gather that
$$ \dint _{\Gamma } x\norm{\d{\point{x}}} =\dint _{\Gamma _1} x\norm{\d{\point{x}}}+\dint _{\Gamma _2}
x\norm{\d{\point{x}}}= 4\sqrt{3}-8+16\sin \dfrac{\pi}{5}.$$



\end{answer}
\end{problem}
\begin{problem}
Find $\doint _\Gamma  z\d{x} + x\d{y} + y\d{z}$ where $\Gamma$ is the
intersection of the sphere $x^2 + y^2 + z^2 = 1$ and the plane $x +
y = 1$, traversed in the positive direction. \begin{answer} The
curve lies on the sphere, and to parametrise this curve, we dispose
of one of the variables, $y$ say, from where $y = 1-x$ and $x^2 +
y^2 + z^2 = 1$ give
$$\begin{array}{lll}x^2 + (1 - x)^2 + z^2 = 1  & \implies & 2x^2 -2x + z^2 = 0\\ & \implies & 2\left(x - \frac{1}{2}\right)^2   + z^2 = \frac{1}{2}\\
& \implies & 4\left(x - \frac{1}{2}\right)^2 + 2z^2 =
1.\end{array}$$ So we now put $$x = \frac{1}{2} +  \frac{\cos t}{2},
\ \ \ z = \dfrac{\sin t}{\sqrt{2}},\ \ \  y = 1 - x =
\frac{1}{2}-\frac{\cos t}{2}.
$$We must integrate on the side of the plane that can be viewed
from   the point $(1,1,0)$ (observe that the vector  $\colvec{1\\ 1 \\
0}$ is normal to the plane). On the $zx$-plane, $4\left(x -
\frac{1}{2}\right)^2 + 2z^2 = 1$ is an ellipse. To obtain a positive
parametrisation we must integrate from $t = 2\pi$ to $t = 0$ (this
is because when you look at the ellipse from the point $(1,1,0)$ the
positive $x$-axis is to your left, and not your right).   Thus
$$\begin{array}{lll}\doint _\Gamma  z\d{x} + x\d{y} + y\d{z} & = &
\dint ^0 _{2\pi} \dfrac{\sin
t}{\sqrt{2}} \d{\left( \frac{1}{2} + \frac{\cos t}{2}\right)}  \\
& & + \dint ^0 _{2\pi} \left( \frac{1}{2} + \frac{\cos
t}{2}\right)\d{\left(\frac{1}{2} -\frac{\cos t}{2}\right)} \\ & & +
\dint ^0 _{2\pi} \left(\frac{1}{2} -\frac{\cos
t}{2}\right)\d{\left(\dfrac{\sin t}{\sqrt{2}}\right)} \\
& = & \dint ^0 _{2\pi} \left(\frac{\sin t}{4}+ \frac{\cos
t}{2\sqrt{2}}+ \frac{\cos t\sin t}{4}  -
\frac{1}{2\sqrt{2}}\right)\ \d{t}  \\
& = & \dfrac{\pi}{\sqrt{2}}.
\end{array}$$

\end{answer}
\end{problem}

\end{multicols}



\section{Closed and Exact Forms}
\begin{lemma}[Poincar\'{e} Lemma] \label{thm:poincare_lemma}
If $\omega$ is a $p$-differential form of continuously
differentiable functions  in $\reals^n$ then $$ \d{(\d{\omega})} =
0.
$$
\end{lemma}
\begin{proof}
We will prove this  by induction on $p$. For $p= 0$ if
$$\omega = f(x_1, x_2, \ldots , x_n)  $$
then
$$\d{\omega} = \dsum _{k = 1} ^n \frac{\partial f}{\partial x_k}\d{x_k}  $$
and
$$\begin{array} {lll}\d{(\d{\omega})} & =  & \dsum _{k = 1} ^n \d{\left(\frac{\partial f}{\partial x_k}\right)}\wedge\d{x_k} \\
& = &  \dsum _{k = 1} ^n \left(\dsum _{j = 1} ^n \frac{\partial ^2
f}{\partial x_j\partial x_k} \wedge \d{x_j}\right)\wedge\d{x_k}\\
& = &  \dsum _{1 \leq j \leq k \leq n} ^n  \left(\frac{\partial ^2
f}{\partial x_j\partial x_k} - \frac{\partial ^2
f}{\partial x_k\partial x_j} \right)\d{x_j}\wedge\d{x_k}\\
& = & 0,
\end{array}$$
since $\omega$ is continuously differentiable and so the mixed
partial derivatives are equal. Consider now an arbitrary $p$-form,
$p>0$. Since such a form can be written as
$$\omega =  \dsum _{ 1 \leq j_1 \leq j_2 \leq \cdots \leq j_p \leq n} a_{j_1j_2\ldots j_p}\d{x_{j_1}} \wedge \d{x_{j_2}} \wedge \cdots \d{x_{j_p}},
$$where the $a_{j_1j_2\ldots j_p}$ are continuous differentiable
functions in $\reals^n$, we have
$$\begin{array}{lll}\d{\omega}  & = &   \dsum _{ 1 \leq j_1 \leq j_2 \leq \cdots \leq j_p \leq n} \d{a_{j_1j_2\ldots j_p}} \wedge \d{x_{j_1}} \wedge \d{x_{j_2}} \wedge \cdots
\d{x_{j_p}}\\
& = &  \dsum _{ 1 \leq j_1 \leq j_2 \leq \cdots \leq j_p \leq n}
\left(\dsum _{i=1} ^n \frac{\partial a_{j_1j_2\ldots j_p}}{\partial
x_i}\d{x_i}\right) \wedge \d{x_{j_1}} \wedge \d{x_{j_2}} \wedge
\cdots \d{x_{j_p}}, \end{array}
$$
it is enough to prove that for each summand
$$\d{\left(\d{a} \wedge\d{x_{j_1}} \wedge \d{x_{j_2}} \wedge \cdots \d{x_{j_p}}\right)} = 0.$$
But $$\begin{array}{lll}\d{\left(\d{a}\wedge\d{x_{j_1}} \wedge
\d{x_{j_2}} \wedge \cdots \d{x_{j_p}}\right)}  & = &
\d{\d{a}}\wedge \left(\d{x_{j_1}} \wedge \d{x_{j_2}} \wedge \cdots
\d{x_{j_p}}\right) \\ & & \qquad  + \d{a}\wedge
\d{\left(\d{x_{j_1}} \wedge \d{x_{j_2}} \wedge \cdots
\d{x_{j_p}}\right)}\\ & = & \d{a}\wedge \d{\left(\d{x_{j_1}}
\wedge \d{x_{j_2}} \wedge \cdots \d{x_{j_p}}\right)},
\end{array} $$ since $\d{\d{a}} = 0$ from the case $p=0$. But an
independent induction argument proves that
$$  \d{\left(\d{x_{j_1}}
\wedge \d{x_{j_2}} \wedge \cdots \d{x_{j_p}}\right)}=0,$$
completing the proof.
\end{proof}


\begin{df}
A differential form $\omega$ is said to be \negrito{exact} if there is
a continuously differentiable function $F$ such that
$$\d F = \omega .$$
\end{df}

\begin{exa}
The differential form $$x\d{x}+ y\d{y} $$is exact, since
$$x\d{x}+ y\d{y} = \d \left(\dfrac{1}{2}(x^2 + y^2)\right) . $$
\end{exa}
\begin{exa}
The differential form $$y\d{x}+ x\d{y} $$is exact, since
$$y\d{x}+ x\d{y} = \d \left(xy\right) . $$
\end{exa}

\begin{exa}
The differential form $$\dfrac{x}{x^2 + y^2}\d{x}+ \dfrac{y}{x^2 +
y^2}\d{y}
$$is exact, since
$$\dfrac{x}{x^2 + y^2}\d{x}+ \dfrac{y}{x^2 + y^2}\d{y} = \d \left(\dfrac{1}{2}\log_e (x^2 + y^2)\right) . $$
\end{exa}
\begin{rem}
Let $\omega = \d{F}$ be an exact form. By the Poincar\'{e} Lemma
Theorem \ref{thm:poincare_lemma}, $\d{\omega} = \d{\d{F}} = 0$.  A
result of Poincar\'{e} says that for certain domains (called \negrito{
star-shaped domains}) the converse is also true, that is, if
$\d{\omega} = 0$ on a star-shaped domain then $\omega$ is exact.
\end{rem}\begin{exa} Determine whether the differential form $$\omega =
\dfrac{2x(1 - e^y)}{(1 + x^2)^2}\d{x}+ \dfrac{e^y}{1 + x^2}\d{y}$$
is exact.
\end{exa}
\begin{solu} Assume there is a function $F$ such that
$$\d F = \omega .$$By the Chain Rule
$$\d F = \dfrac{\partial F}{\partial x}\d{x}+ \dfrac{\partial F}{\partial y}\d{y} .$$
This demands that
$$\dfrac{\partial F}{\partial x} = \dfrac{2x(1 -
e^y)}{(1 + x^2)^2},$$
$$\dfrac{\partial F}{\partial y} = \dfrac{e^y}{1 + x^2}.$$
We have a choice here of integrating either the first, or the
second expression. Since integrating the second expression (with
respect to $y$) is easier, we find
$$F(x, y) = \dfrac{e^y}{1 + x^2} + \phi (x),$$where $\phi (x)$ is a
function depending only on $x$. To find it, we differentiate the
obtained expression for $F$ with respect to $x$ and find
$$\dfrac{\partial F}{\partial x} = -\dfrac{2xe^y}{(1 + x^2)^2} + \phi '(x).$$Comparing this with our first
expression for $\dfrac{\partial F}{\partial x}$, we find
$$\phi '(x) = \dfrac{2x}{(1 + x^2)^2}, $$that is
$$\phi (x) = -\dfrac{1}{1 + x^2} + c,$$where $c$ is a constant.
We then take
$$F(x, y) =  \dfrac{e^y - 1}{1 + x^2} + c.$$
\end{solu}
\begin{exa}
Is there a continuously differentiable function such that $$ \d{F}
= \omega = y^2z^3 \d{x} + 2xyz^3\d{y} + 3xy^2z^2\d{z}\ \ ?
$$
\end{exa}
\begin{solu} We have $$\begin{array}{lll}\d{\omega} &  = & (2yz^3\d{y} +
3y^2z^2\d{z})\wedge \d{x} \\ & &  + (2yz^3\d{x} + 2xz^3\d{y} +
6xyz^2\d{z})\wedge \d{y} \\ & & +
(3y^2z^2\d{x} + 6xyz^2\d{y} + 6xy^2z\d{z})\wedge\d{z} \\
&  = &  0, \end{array}$$ so this form is exact in a star-shaped
domain. So put $$ \d{F} =  \frac{\partial F}{\partial x}\d{x} +
\frac{\partial F}{\partial y}\d{y} + \frac{\partial F}{\partial
z}\d{z} = y^2z^3 \d{x} + 2xyz^3\d{y} + 3xy^2z^2\d{z}.
$$Then $$\frac{\partial F}{\partial x} = y^2z^3 \implies F = xy^2z^3 + a(y,z), $$
$$\frac{\partial F}{\partial y} =  2xyz^3\implies F = xy^2z^3 + b(x,z), $$
$$\frac{\partial F}{\partial z} = 3xy^2z^2 \implies F = xy^2z^3 + c(x,y), $$
Comparing these three expressions for $F$, we obtain $F(x,y,z) =
xy^2z^3$.
\end{solu}













 We have the following equivalent of the Fundamental
Theorem of Calculus.
\begin{thm}\label{thm:exact_forms}
Let $U \subseteq \reals^n$ be an open set. Assume $\omega = \d F$ is an exact form, and $\Gamma$ a path in $U$ with starting
point $A$ and endpoint $B$. Then
$$\dint _{\Gamma} \omega = \dint _{A} ^B \d F = F(B) - F(A).
$$In particular, if $\Gamma $ is a simple closed path, then
$$\doint _{\Gamma} \omega = 0.$$
\end{thm}
\begin{exa}
Evaluate the integral $$\doint\limits_\Gamma \dfrac{2x}{x^2 +
y^2}\d{x} + \dfrac{2y}{x^2 + y^2}\ \d{y}
$$ where $\Gamma$ is the closed polygon with vertices at $A=(0,0)$, $B=(5,0)$, $C=(7,2)$, $D= (3,2)$, $E=(1,1)$, traversed in the order
$ABCDEA$.
\end{exa}
\begin{solu}Observe that $$ \d{\left(\dfrac{2x}{x^2 + y^2}\d{x} +
\dfrac{2y}{x^2 + y^2}\ \d{y}\right)}   = -\dfrac{4xy}{(x^2 +
y^2)^2}\d{y} \wedge \d{x}  -\dfrac{4xy}{(x^2 + y^2)^2}\d{x} \wedge
\d{y} =0,
$$and so the form is exact in a start-shaped domain. By virtue of Theorem
\ref{thm:exact_forms}, the integral is $0$.
\end{solu}
\begin{exa}
Calculate the path integral $$\doint _{\Gamma} (x^2 - y)\d{x}+ (y^2 -
x)\d{y},$$ where $\Gamma$ is a loop of  $\dis{ x^3 + y^3 - 2xy = 0}$
traversed once in the positive sense.
\end{exa}\begin{solu} Since
$$\dfrac{\partial}{\partial y} (x^2 - y) = -1 =
\dfrac{\partial}{\partial x} (y^2 - x), $$the form is exact, and
since this is a closed simple path, the integral is $0$.
\end{solu}







\section{Two-Manifolds}
\begin{df}
A  \textbf{$2$-dimensional  oriented manifold of $\reals^2$} is simply
an open set  (region) $D\in \reals^2$, where the $+$ orientation is
counter-clockwise and the $-$  orientation is clockwise. A general
oriented $2$-manifold is a union of open sets.
\end{df}
\begin{rem}The region $-D$ has opposite orientation to $D$ and
$$ \dint _{-D} \omega = -\dint _D \omega .   $$ We will often write
$$\dint _Df(x,y)\d{A} $$ where $\d{A}$ denotes the \negrito{area
element}.
\end{rem}

\begin{rem}
In this section, unless otherwise noticed, we will choose the
positive orientation for the regions considered. This corresponds to
using the area form $\d{x} \d{y}$.
\end{rem}
Let $D \subseteq \reals^2$. Given a function $f:D \rightarrow
\reals$, the integral
$$\dint \limits_D f\d{A} $$is the sum of all the values of $f$ restricted to $D$. In particular,
$$\dint \limits_D \d{A}$$is the area of $D$.


\section{Three-Manifolds}
\begin{df}
A  \textbf{$3$-dimensional  oriented manifold of $\reals^3$} is simply
an open set  (body) $V\in \reals^3$, where the $+$ orientation is in
the direction of the outward pointing normal to the body, and
 the $-$ orientation is in the direction of the inward pointing  normal to the body. A general
oriented $3$-manifold is a union of open sets.
\end{df}
\begin{rem}The region $-M$ has opposite orientation to $M$ and
$$ \dint _{-M} \omega = -\dint _M \omega .   $$ We will often write
$$\dint _Mf\d{V} $$ where $\d{V}$ denotes the \negrito{volume
element}.
\end{rem}

\begin{rem}
In this section, unless otherwise noticed, we will choose the
positive orientation for the regions considered. This corresponds
to using the volume form $\d{x}\wedge \d{y}\wedge \d{z}$.
\end{rem}
Let $V \subseteq \reals^3$. Given a function $f:V \rightarrow
\reals$, the integral
$$\dint \limits_V f\d{V} $$is the sum of all the values of $f$ restricted to $V$. In particular,
$$\dint \limits_V \d{V} $$is the oriented volume of $V$.

\begin{exa}
Find $$\dint \limits_{[0;1]^3} \ x^2ye^{xyz} \ \d{V}.$$
\end{exa}
\begin{solu}The integral is
$$\begin{array}{lll}
\dint_0 ^1 \left(\dint_0 ^1 \left(\dint_0 ^1 x^2ye^{xyz}\
\d{z}\right)\d{y}\right) \d{x}& = & \dint_0 ^1 \left(\dint_0 ^1
x(e^{xy} - 1)\ \d{y}\right) \d{x}\\
& = & \dint_0 ^1 (e^x - x - 1) \d{x}\\
& = & e  - \dfrac{5}{2}.
\end{array}$$
\end{solu}
s

\section{Surface Integrals}
\begin{df}
A  \textbf{$2$-dimensional  oriented manifold of $\reals^3$} is simply
a smooth surface $D\in \reals^3$, where the $+$ orientation is in
the direction of the outward normal pointing away from the origin
and the $-$ orientation is in the direction of the inward normal
pointing towards the origin. A general oriented $2$-manifold in
$\reals^3$ is a union of surfaces.
\end{df}
\begin{rem}The surface $-\Sigma$ has opposite orientation to $\Sigma$ and
$$ \dint _{-\Sigma} \omega = -\dint _\Sigma \omega .   $$
\end{rem}

\begin{rem}
In this section, unless otherwise noticed, we will choose the
positive orientation for the regions considered. This corresponds
to using the ordered basis  $$\{\d{y}\wedge \d{z}, \
\d{z}\wedge\d{x}, \ \d{x}\wedge\d{y}\}.$$
\end{rem}

\begin{df}
Let $f:\reals^3\rightarrow \reals$. The integral of $f$ over the
smooth surface $\Sigma$ (oriented in the positive sense) is given by
the expression  $$\dint \limits_\Sigma f \norm{\d{ ^2\vector{x}}}.$$Here
$$\norm{\d{ ^2\vector{x}}} = \sqrt{(\d{x} \wedge \d{y})^2 + (\d{z} \wedge \d{x})^2 + (\d{y} \wedge \d{z})^2}$$
is the \negrito{surface area element}.
\end{df}
\begin{exa}
Evaluate $\dint \limits_\Sigma z\norm{\d{ ^2\vector{x}}}$ where
$\Sigma$ is the outer surface of the section of the paraboloid $z =
x^2 + y^2, 0 \leq z \leq 1.$
\end{exa}
\begin{solu} We parametrise the paraboloid as follows. Let $x = u, y = v, z
= u^2 + v^2.$ Observe that the domain $D$ of $\Sigma$ is the unit
disk $u^2 + v^2 \leq 1$. We see that
$$\d{x} \wedge \d{y} =   \d{u}\wedge \d{v},$$
$$\d{y} \wedge \d{z}  = -2u \d{u}\wedge \d{v}, $$
$$\d{z} \wedge \d{x}  = -2v \d{u}\wedge \d{v},
$$and so
$$\norm{\d{ ^2\vector{x}}} = \sqrt{1 + 4u^2 + 4v^2}\d{u} \wedge \d{v} .$$
Now,
$$\dint \limits_\Sigma z \norm{\d{ ^2\vector{x}}}= \dint \limits_D (u^2 + v^2)\sqrt{1 + 4u^2 + 4v^2}\d{u}\d{v}.
$$To evaluate this last integral we use polar coordinates, and
so
$$\begin{array}{lll}\dint \limits_D (u^2 + v^2)\sqrt{1 + 4u^2 + 4v^2}\d{u}
\d{v} & = & \dint _0 ^{2\pi} \dint _0 ^1 \rho^3 \sqrt{1 +
4\rho^2} \d{\rho} \d{\theta} \\
& = & \dfrac{\pi}{12}(5\sqrt{5} + \dfrac{1}{5}).
\end{array}$$
\end{solu}
\begin{exa} Find the area of that part of the cylinder $x^2 +
y^2 = 2y$ lying inside the sphere $x^2+y^2+z^2=4$. \end{exa}
\begin{solu} We have $$x^2 + y^2 = 2y \iff x^2 + (y - 1)^2 = 1. $$
We parametrise the cylinder by putting $x = \cos u, y - 1 = \sin u$,
and $z = v$. Hence
$$\d{x} = -\sin u\d{u}, \ \d{y} = \cos u\d{u}, \ \d{z} = d{v},    $$
whence
$$\d{x} \wedge \d{y} =  0, \d{y} \wedge \d{z} = \cos u \d{u} \wedge \d{v},
\d{z} \wedge \d{x} = \sin u\d{u} \wedge \d{v},  $$ and so
$$\begin{array}{lll}\norm{\d{ ^2\vector{x}}} & = & \sqrt{(\d{x} \wedge \d{y})^2 + (\d{z} \wedge \d{x})^2 + (\d{y} \wedge \d{z})^2}\\ &  = & \sqrt{\cos^2u + \sin^2u}\ \d{u}\wedge\d{v} \\
& = & \d{u}\wedge\d{v}.
\end{array}$$The cylinder and the sphere intersect when $x^2 + y^2 = 2y$
and $x^2 + y^2 + z^2 = 4$, that is, when $z^2 = 4 - 2y$, i.e. $v^2 =
4 - 2(1 + \sin u) = 2 - 2\sin u$. Also $ 0 \leq u \leq \pi.$ The
integral is thus
$$\begin{array}{lll} \dint \limits_\Sigma \norm{\d{ ^2\vector{x}}} & = & \dint _0 ^{\pi}\dint _{-\sqrt{2 - 2\sin u}} ^{\sqrt{2 - 2\sin u}} \d{v}\d{u} = \dint _0 ^{\pi} 2\sqrt{2 - 2\sin u } \d{u}\\
& = & 2\sqrt{2} \dint _0 ^{\pi} \sqrt{1 - \sin u} \ \d{u} \\
& = & 2\sqrt {2} \left( 4\sqrt {2}-4 \right) . \\
\end{array}
$$
\end{solu}
\begin{exa}
Evaluate  $$\dint \limits_\Sigma x\d{y}\d{z} + (z^2 - zx) \d{z}
\d{x} - xy\d{x} \d{y},
$$where $\Sigma$ is the top side of the triangle with vertices at
$(2,0,0)$, $(0,2,0)$, $(0,0,4)$.
\end{exa}
\begin{solu} Observe that the plane passing through the three given points
has equation $2x + 2y + z = 4$. We project this plane onto the
coordinate axes obtaining
$$ \dint \limits_\Sigma x\d{y}\d{z} = \dint _0 ^4 \dint _0 ^{2 - z/2} (2-y-z/2) \d{y} \d{z} =  \frac{8}{3},  $$
$$ \dint \limits_\Sigma (z^2 - zx)\d{z}\d{x} = \dint _0 ^2 \dint _0 ^{4-2x} (z^2-zx) \d{z}  \d{x} =  8,  $$
$$ -\dint \limits_\Sigma xy\d{x}\d{y} = -\dint _0 ^2 \dint _0 ^{2 - y} xy \d{x}  \d{y} =  -\frac{2}{3},  $$
and hence
$$\dint \limits_\Sigma x\d{y}\d{z} + (z^2 - zx)
\d{z} \d{x} - xy\d{x} \d{y} = 10.
$$
\end{solu}
\section*{\psframebox{Homework}}
\begin{problem}
Evaluate $\dint \limits_\Sigma y \norm{\d{ ^2\vector{x}}}$ where
$\Sigma$ is the surface $z = x + y^2, 0 \leq x \leq 1, 0 \leq y \leq
2.$ \begin{answer} We parametrise the surface by letting $x = u, y =
v, z = u + v^2.$ Observe that the domain $D$ of $\Sigma$ is the
square $[0; 1]\times [0; 2]$. Observe that
$$\d{x} \wedge \d{y}
=   \d{u} \wedge \d{v} ,$$
$$\d{y} \wedge \d{z} = -\d{u} \wedge \d{v}, $$
$$\d{z} \wedge \d{x} = -2v\d{u} \wedge \d{v},
$$and so
$$\norm{\d{ ^2\vector{x}}} = \sqrt{2 + 4v^2}\d{u} \wedge \d{v}.$$
The integral becomes
$$\begin{array}{lll}
\dint \limits_\Sigma y \norm{\d{ ^2\vector{x}}} & = & \dint_0 ^2\dint _0
^1
v\sqrt{2 + 4v^2}\d{u}\d{v} \\
& = & \left(\dint_0 ^1 \d{u}\right) \left(\dint_0 ^2 y\sqrt{2 +
4v^2}\d{v}\right) \\
& = & \dfrac{13\sqrt{2}}{3}.
\end{array}$$
\end{answer}
\end{problem}
\begin{problem}
Consider the cone $z = \sqrt{x^2+y^2}$. Find the surface area of the
part of the cone which lies between the planes $z = 1$ and $z = 2$.
\begin{answer}
Using $x=r\cos \theta$, $y=r\sin \theta$, $1\leq r \leq 2$, $0 \leq
\theta \leq 2\pi$, the surface area is
$$ \sqrt{2}\dint _0  ^{2\pi}\dint _1 ^2 r\d{r}\d{\theta}=3\pi\sqrt{2}.  $$
\end{answer}
\end{problem}

\begin{problem}
Evaluate $\dint \limits_\Sigma x^2 \norm{\d{ ^2\vector{x}}}$ where
$\Sigma$ is the surface of the unit sphere $x^2 + y^2 + z^2 = 1.$
\begin{answer} We use spherical coordinates, $(x, y, z) =
(\cos\theta\sin\phi, \sin\theta\sin\phi, \cos\phi)$. Here $\theta
\in [0; 2\pi]$ is the latitude and $\phi \in [0; \pi]$ is the
longitude. Observe that
$$\d{x}\wedge\d{y}=
\sin\phi\cos\phi \d{\phi}\wedge\d{\theta} ,$$
$$\d{y}\wedge\d{z} = \cos\theta\sin^2\phi \d{\phi}\wedge\d{\theta},
$$
$$\d{z}\wedge\d{x}= -\sin\theta\sin^2\phi \d{\phi}\wedge\d{\theta},
$$and so
$$\norm{\d{ ^2 \vector{x}}} = \sin\phi \d{\phi}\wedge\d{\theta}.$$
The integral becomes
$$\begin{array}{lll}
\dint \limits_\Sigma x^2 \norm{\d{ ^2\vector{x}}} & = & \dint _0 ^{2\pi}
\dint _{0} ^{\pi} \cos^2\theta\sin^3\phi \d{\phi}  \d{\theta} \\
& = & \dfrac{4\pi}{3}.
\end{array}$$
\end{answer}
\end{problem}
\begin{problem}
Evaluate $\dint _S z\norm{\d{ ^2\vector{x}}}$ over the conical surface
$z=\sqrt{x^2 + y^2}$ between $z=0$ and $z=1$. \begin{answer} Put $x
= u, y = v, z^2 = u^2 + v^2$. Then
$$\d{x} = \d{u}, \ \d{y} = \d{v}, \ z\d{z} = u\d{u} + v\d{v},    $$
whence
$$\d{x} \wedge \d{y} =  \d{u} \wedge \d{v}, \d{y} \wedge \d{z} = -\dfrac{u}{z}  \d{u} \wedge \d{v},
\d{z} \wedge \d{x} = -\dfrac{v}{z}  \d{u} \wedge \d{v},  $$ and so
$$\begin{array}{lll}\norm{\d{ ^2\vector{x}}} & = &  \sqrt{(\d{x} \wedge \d{y})^2 + (\d{z} \wedge \d{x})^2 + (\d{y} \wedge \d{z})^2} \\
& = & \sqrt{1 + \dfrac{u^2 + v^2}{z^2}}\ \d{u}\wedge\d{v} \\ &  = &
\sqrt{2}\ \d{u}\wedge\d{v}.
\end{array}$$ Hence $$\dint \limits_{\Sigma} \ z\norm{\d{ ^2\vector{x}}} =
\dint\limits_{u^2 + v^2 \leq 1} \sqrt{u^2 + v^2}\ \sqrt{2}\
\d{u}\d{v} = \sqrt{2}\dint _0 ^{2\pi}\dint _0 ^1  \rho^2\
\d{\rho}\d{\theta} = \dfrac{2\pi\sqrt{2}}{3}.
$$
\end{answer}
\end{problem}

\begin{problem}
You put a perfectly spherical egg through an egg slicer, resulting
in $n$ slices of identical height, but you forgot to peel it first!
Shew that the amount of egg shell in any of the slices is the same.
Your argument must use surface integrals.\begin{answer} If the egg
has radius $R$,  each slice will have height $2R/n$. A slice can be
parametrised by $0 \leq \theta \leq 2\pi$, $\phi_1 \leq \phi \leq
\phi_2$, with $$R\cos\phi_1 - R\cos\phi_2 = 2R/n.$$ The area of the
part of the surface of the sphere in slice is $$\dint _0
^{2\pi}\dint_{\phi_1} ^{\phi_2} R^2\sin\phi \d{\phi}\d{\theta} = 2\pi
R^2(\cos\phi_1 - \cos\phi_2 ) = 4\pi R^2/n.$$ This means that each
of the $n$ slices has identical area $4\pi R^2 /n$.
\end{answer}
\end{problem}

\begin{problem}
Evaluate  $$\dint \limits_\Sigma xy\d{y}\d{z} - x^2\d{z} \d{x} + (x
+ z)\d{x} \d{y},
$$where $\Sigma$ is the top of the triangular region of the plane $2x + 2y + z =
6$ bounded by the first octant. \begin{answer} We project this plane
onto the coordinate axes obtaining
$$ \dint \limits_\Sigma xy\d{y}\d{z} = \dint _0 ^6 \dint _0 ^{3-z/2} (3-y-z/2)y \d{y}  \d{z} =  \frac{27}{4},  $$
$$ -\dint \limits_\Sigma x^2\d{z}\d{x} = -\dint _0 ^3 \dint _0 ^{6-2x} x^2 \d{z}  \d{x} =  -\frac{27}{2},  $$
$$ \dint \limits_\Sigma (x + z)\d{x}\d{y} = \dint _0 ^3 \dint _0 ^{3 - y} (6 - x - 2y) \d{x}  \d{y} =  \frac{27}{2},  $$
and hence
$$\dint \limits_\Sigma xy\d{y}\d{z} -
x^2\d{z} \d{x} + (x + z)\d{x} \d{y} = \frac{27}{4}.
$$
\end{answer}
\end{problem}
\section{Green's, Stokes', and Gauss' Theorems}
We are now in position to state the general Stoke's Theorem.

\begin{thm}[General Stoke's Theorem]
Let $M$ be a smooth oriented manifold, having boundary $\partial
M$. If $\omega$ is a differential form, then
$$\dint _{\partial M} \omega = \dint _{M} \d{\omega} .$$
\end{thm}
In $\reals^2$, if $\omega$ is a $1$-form, this takes the name of
\negrito{Green's Theorem.}
\begin{exa}
Evaluate $\doint _C (x - y^3)\d{x}+ x^3\d{y}$ where  $C$ is the
circle $x^2 + y^2 = 1$.
\end{exa}
\begin{solu} We will first use Green's Theorem and then evaluate the
integral directly. We have
$$\begin{array}{lll}\d{\omega} &  = &  \d (x - y^3) \wedge \d{x}+ \d (x^3) \wedge \d{y}\\
& = & (\d{x}- 3y^2\d{y})\wedge \d{x}+ (3x^2\d{x})\wedge \d{y} \\
& = & (3y^2 + 3x^2) \d{x}\wedge \d{y}.
\end{array}$$
The region $M$ is the area enclosed by the circle $x^2 + y^2 = 1. $
Thus by Green's Theorem, and using polar coordinates,
$$\begin{array}{lll}\doint _C (x - y^3)\d{x}+ x^3\d{y} & = &
\dint _M (3y^2 + 3x^2) \d{x}\d{y} \\
& = & \dint _0 ^{2\pi} \dint _0 ^1 3\rho^2 \rho \d{\rho} \d{\theta}
\\
& = & \dfrac{3\pi}{2}. \end{array}$$ {\em Aliter:} We can evaluate
this integral directly, again resorting to polar coordinates.
$$\begin{array}{lll}\doint _C (x - y^3)\d{x}+ x^3\d{y} & = &
\dint _0 ^{2\pi} (\cos\theta -\sin^3\theta)(-\sin\theta)\d{\theta} + (\cos^3\theta)(\cos\theta)\d{\theta} \\
& = & \dint _0 ^{2\pi} (\sin^4\theta + \cos^4\theta -
\sin\theta\cos\theta)\d{\theta}.
\end{array}$$
To evaluate the last integral, observe that $1 = (\sin^2\theta +
\cos^2\theta )^2 = \sin^4\theta + 2\sin^2\theta\cos^2\theta +
\cos^4\theta$, whence the integral equals
$$\begin{array}{lll}
\dint _0 ^{2\pi} (\sin^4\theta + \cos^4\theta -
\sin\theta\cos\theta)\d{\theta} & = & \dint _0 ^{2\pi} (1 -
2\sin^2\theta\cos^2\theta  - \sin\theta\cos\theta)\d{\theta} \\
& = & \dfrac{3\pi}{2}.
\end{array}$$
\end{solu}


In general, let $$ \omega = f(x,y)\d{x} + g(x,y)\d{y}  $$ be a
$1$-form in $\reals ^2$. Then
 $$\begin{array}{lll} \d{\omega}  & =  & \d{f(x,y)}\wedge\d{x} + \d{g(x,y)}\wedge\d{y} \\
 &=& \left(\dfrac{\partial }{\partial x}f(x,y)\d{x}+ \dfrac{\partial }{\partial y}f(x,y)\d{y}\right)\wedge\d{x}
 +  \left(\dfrac{\partial }{\partial x}g(x,y)\d{x}+ \dfrac{\partial }{\partial
 y}g(x,y)\d{y}\right)\wedge\d{y}\\
 & = &  \left(\dfrac{\partial }{\partial x}g(x,y)- \dfrac{\partial }{\partial y}f(x,y)\right)\d{x}\wedge\d{y}
 \end{array}$$
which gives the classical Green's Theorem
$$  \dint \limits _{\partial M}  f(x,y)\d{x} + g(x,y)\d{y} = \dint \limits _M   \left(\dfrac{\partial }{\partial x}g(x,y)- \dfrac{\partial }{\partial y}f(x,y)\right)\d{x}\d{y}.$$

In $\reals^3$, if $\omega$ is a 2-form, the above theorem takes the
name of \negrito{Gauss} or the \negrito{Divergence Theorem}.

\bigskip

\begin{exa}
 Evaluate $\dint _S (x - y)\d{y}  \d{z} + z\d{z}  \d{x}- y\d{x} \d{y}$ where $S$ is the surface of the sphere $$x^2 + y^2 + z^2 = 9$$
and the positive direction is the outward normal.
\end{exa}
\begin{solu} The region $M$ is the interior of the sphere $x^2 + y^2 + z^2
= 9$. Now,
$$\begin{array}{lll}\d{\omega} &  = & (\d{x}- \d{y}) \wedge\d{y} \wedge \d{z}
+ \d{z}\wedge\d{z} \wedge \d{x}-  \d{y}\wedge\d{x}\wedge \d{y} \\
& = &  \d{x}\wedge \d{y} \wedge \d{z}. \\
\end{array}$$
The integral becomes
$$\begin{array}{lll}\dint \limits_M \d{x} \d{y}  \d{z} & = &  \dfrac{4\pi}{3}(27)\\ &  = & 36\pi. \end{array}$$
{\em Aliter:} We could evaluate this integral directly. We have
$$\dint _\Sigma (x - y)\d{y}  \d{z}  = \dint _\Sigma x\d{y}  \d{z},
$$since $(x, y , z) \mapsto -y$ is an odd function of $y$ and the
domain of integration is symmetric with respect to $y$. Now,
$$\begin{array}{lll}\dint _\Sigma x\d{y}  \d{z} & = &
\dint_{-3} ^3 \dint _0 ^{2\pi} |\rho|\sqrt{9 - \rho^2}\d{\rho}
\d{\theta}
\\
& = &  36\pi.
\end{array}
$$
Also
$$\dint _\Sigma z\d{z} \d{x}= 0,
$$since $(x, y , z) \mapsto z$ is an odd function of $z$ and the
domain of integration is symmetric with respect to $z$. Similarly
$$\dint _\Sigma -y\d{x} \d{y}   = 0,
$$since $(x, y , z) \mapsto -y$ is an odd function of $y$ and the
domain of integration is symmetric with respect to $y$.
\end{solu}

In general, let $$ \omega = f(x,y,z)\d{y} \wedge \d{z} +
g(x,y,z)\d{z} \wedge \d{x} + h(x,y,z) \d{x}  \wedge \d{y}
$$ be a $2$-form in $\reals ^3$.
Then
 $$\begin{array}{lll} \d{\omega}  & =  & \d{f(x,y,z)}\d{y} \wedge \d{z} + \d{g(x,y,z)}\d{z}
\wedge \d{x} + \d{h(x,y,z)} \d{x}  \wedge \d{y} \\
 &=& \left(\dfrac{\partial }{\partial x}f(x,y,z)\d{x}+ \dfrac{\partial }{\partial y}f(x,y,z)\d{y}+ \dfrac{\partial }{\partial z}f(x,y,z)\d{z}\right)\wedge \d{y}\wedge\d{z}
\\ & & \quad  +  \left(\dfrac{\partial }{\partial x}g(x,y,z)\d{x}+ \dfrac{\partial
}{\partial
 y}g(x,y,z)\d{y}+ \dfrac{\partial
}{\partial
 z}g(x,y,z)\d{z}\right)\wedge \d{z}\wedge\d{x}\\
 & & \quad  +  \left(\dfrac{\partial }{\partial x}h(x,y,z)\d{x}+ \dfrac{\partial
}{\partial
 y}h(x,y,z)\d{y}+ \dfrac{\partial
}{\partial
 z}h(x,y,z)\d{z}\right)\wedge \d{x}\wedge\d{y}\\
 & = &  \left(\dfrac{\partial }{\partial x}f(x,y,z)+ \dfrac{\partial }{\partial
 y}g(x,y,z) +\dfrac{\partial }{\partial
 z}h(x,y,z)  \right)\d{x}\wedge\d{y} \wedge \d{z},
 \end{array}$$
which gives the classical Gauss's Theorem
$$  \dint \limits _{\partial M}  f(x,y,z)\d{y}  \d{z} + g(x,y,z)\d{z}
\d{x} + h(x,y,z) \d{x}  \d{y} = \dint \limits _M
\left(\dfrac{\partial }{\partial x}f(x,y,z)+\dfrac{\partial
}{\partial y}g(x,y,z)+\dfrac{\partial }{\partial
z}h(x,y,z)\right)\d{x}\d{y}\d{z}.$$ Using classical notation, if
$$\vector{a}=\colvec{f(x,y,z)\\ g(x,y,z)\\ h(x,y,z)}, \d{\vector{S}} = \colvec{\d{y}\d{z}\\ \d{z}\d{x}\\ \d{x}\d{y}}, $$
then
$$\dint  \limits _{ M}  (\nabla \cdot \vector{a} )\d{V} = \dint \limits _{\partial M} \vector{a} \cdot \d{\vector{S}}.  $$




\bigskip


The classical Stokes' Theorem occurs when $\omega$ is a $1$-form in
$\reals^3$.

\begin{exa}
Evaluate $\doint _C y\d{x}+ (2x - z)\d{y} + (z - x)\d{z} $ where $C$
is the intersection of the sphere $x^2 + y^2 + z^2 = 4$ and the
plane $z = 1$.
\end{exa}
\begin{solu} We have
$$\begin{array}{lll}
\d{\omega}  & = & (\d{y})\wedge \d{x}+ (2\d{x}- \d{z} )\wedge \d{y}
+ (\d{z} - \d{x})\wedge \d{z}
\\
& = & -\d{x}\wedge \d{y} + 2\d{x}\wedge \d{y} + \d{y} \wedge \d{z} + \d{z} \wedge \d{x}\\
& = & \d{x}\wedge \d{y} + \d{y} \wedge \d{z} + \d{z} \wedge \d{x}.\\
\end{array}$$
Since on $C$, $z = 1$, the surface $\Sigma$ on which we are
integrating is the inside of the circle $x^2 + y^2 + 1 = 4,$ i.e.,
$x^2 + y^2 = 3.$ Also, $z = 1 $ implies $\d{z} = 0$ and so
$$\dint \limits_\Sigma \d{\omega} = \dint \limits_\Sigma \d{x} \d{y}.$$Since this is just the area of the circular region $x^2 +
y^2 \leq 3$, the integral evaluates to
$$ \dint \limits_\Sigma \d{x} \d{y} = 3\pi.$$
\end{solu}


In general, let $$ \omega = f(x,y,z)\d{x} + g(x,y,z)\d{y} ++
h(x,y,z)\d{z}  $$ be a $1$-form in $\reals ^3$. Then
 $$\begin{array}{lll} \d{\omega}  & =  & \d{f(x,y,z)}\wedge\d{x} + \d{g(x,y,z)}\wedge\d{y}+ \d{h(x,y,z)}\wedge\d{z} \\
 &=& \left(\dfrac{\partial }{\partial x}f(x,y,z)\d{x}+ \dfrac{\partial }{\partial y}f(x,y,z)\d{y}+ \dfrac{\partial }{\partial
 z}f(x,y,z)\d{z}\right)\wedge\d{x} \\ & &
 +  \left(\dfrac{\partial }{\partial x}g(x,y,z)\d{x}+ \dfrac{\partial }{\partial
 y}g(x,y,z)\d{y} +\dfrac{\partial }{\partial z}g(x,y,z)\d{z}\right)\wedge\d{y}\\
 & & \quad +  \left(\dfrac{\partial }{\partial x}h(x,y,z)\d{x}+ \dfrac{\partial }{\partial
 y}h(x,y,z)\d{y} +\dfrac{\partial }{\partial z}h(x,y,z)\d{z}\right)\wedge\d{z}\\
 & = &
 \left(\dfrac{\partial }{\partial y}h(x,y,z)- \dfrac{\partial }{\partial
 z}g(x,y,z)\right)\d{y}\wedge\d{z} \\
 & & \quad + \left(\dfrac{\partial }{\partial z}f(x,y,z)- \dfrac{\partial }{\partial
 x}h(x,y,z)\right)\d{z}\wedge\d{x}\\
 & & \quad
  \left(\dfrac{\partial }{\partial x}g(x,y,z)- \dfrac{\partial }{\partial y}f(x,y,z)\right)\d{x}\wedge\d{y}
 \end{array}$$
which gives the classical Stokes' Theorem
$$ \begin{array}{lll} \dint \limits _{\partial M}  f(x,y,z)\d{x} + g(x,y,z)\d{y}+h(x,y,z)\d{z} \\
 & = & \dint \limits _M   \left(\dfrac{\partial }{\partial y}h(x,y,z)- \dfrac{\partial }{\partial
z}g(x,y,z)\right)\d{y}\d{z}\\
& & \quad + \left(\dfrac{\partial }{\partial z}g(x,y,z)-
\dfrac{\partial
}{\partial x}f(x,y,z)\right)\d{x}\d{y} \\
& & \quad + \left(\dfrac{\partial }{\partial x}h(x,y,z)-
\dfrac{\partial }{\partial y}f(x,y,z)\right)\d{x}\d{y}
.\end{array}$$

Using classical notation, if
$$\vector{a}=\colvec{f(x,y,z)\\ g(x,y,z)\\ h(x,y,z)}, \quad \d{\vector{r}}=\colvec{\d{x}\\ \d{y}\\ \d{z}},\quad  \d{\vector{S}} = \colvec{\d{y}\d{z}\\ \d{z}\d{x}\\ \d{x}\d{y}}, $$
then
$$\dint  \limits _{ M}  (\nabla \cross \vector{a} )\cdot \d{\vector{S}} = \dint \limits _{\partial M} \vector{a} \cdot \d{\vector{r}}.  $$






\section*{\psframebox{Homework}}
\begin{problem}
Evaluate $\doint _C x^3y\d{x}+ xy\d{y}$ where  $C$ is the square with
vertices at $(0, 0)$, $(2, 0)$, $(2, 2)$ and $(0, 2)$.
\begin{answer}Evaluating this directly would result in evaluating four path
integrals, one for each side of the square. We will use Green's
Theorem. We have
$$\begin{array}{lll}\d{\omega} &  = &  \d (x^3y) \wedge \d{x}+ \d (xy) \wedge \d{y}\\
& = & (3x^2y\d{x}+ x^3\d{y}) \wedge \d{x}+ (y\d{x}+
x\d{y} ) \wedge \d{y} \\
& = & (y - x^3)\d{x}\wedge \d{y}.
\end{array}$$
The region $M$ is the area enclosed by the square. The integral
equals
$$\begin{array}{lll}
\doint _C x^3y\d{x}+ xy\d{y} & = & \dint _0 ^2 \dint _0 ^2 (y
- x^3)\d{x} \d{y} \\
& = & -4.
\end{array}$$
\end{answer}
\end{problem}
\begin{problem}
Consider the triangle $\triangle$ with vertices $A:(0,0)$, $B:(1,
1),$ $C:(-2, 2)$.
\begin{dingautolist}{202} \item If $L_{PQ}$ denotes the equation of the
line joining $P$ and $Q$ find $L_{AB}$,  $L_{AC}$, and $L_{BC}$.
 \item Evaluate
$$\doint _\triangle y^2\d{x}+ x\d{y}.$$

 \item  Find $$\dint\limits_{D} (1 - 2y) \d{x}\wedge \d{y}$$where ${D}$ is the interior of $\triangle$.
\end{dingautolist} \begin{answer}We have
\begin{dingautolist}{202} \item $L_{AB}$ is $y = x$;  $L_{AC}$  is
$y = -x$, and  $L_{BC}$ is clearly $y = -\dfrac{1}{3}x +
\dfrac{4}{3}$.
 \item  We have \renewcommand{\arraystretch}{1.5}
$${\everymath{\dis}\begin{array}{lllll}  \dint _{AB} y^2\d{x}+ x\d{y}  & = &  \dint _0 ^1 (x^2 + x)\d{x}            & = &  \dfrac{5}{6}  \\
\dint _{BC} y^2\d{x}+ x\d{y}  & = &  \dint _1 ^{-2}
\left(\left(-\dfrac{1}{3}x + \dfrac{4}{3}\right)^2 -\dfrac{1}{3}x \right)\d{x}            & = &  -\dfrac{15}{2}  \\
\dint _{CA} y^2\d{x}+ x\d{y}  & = &  \dint _{-2} ^0 (x^2 - x)\d{x}            & = &  \dfrac{14}{3}  \\
\end{array} }$$Adding these integrals we find $$ \doint _\triangle y^2\d{x}+ x\d{y} = -2.$$

 \item   We have
$${\everymath{\dis}\begin{array}{lll}\dint\limits_{D} (1 - 2y) \d{x}\wedge\d{y} & = & \dint _{-2} ^0 \left(\dint _{-x} ^{-x/3 + 4/3} (1 - 2y)
\d{y}\right) \d{x}\\ & &\quad + \dint _{0} ^1 \left(\dint _{x} ^{-x/3
+ 4/3}
(1 - 2y)\d{y}\right) \d{x} \\
& = & -\dfrac{44}{27} - \dfrac{10}{27} \\
& = & -2.
\end{array}}$$
\end{dingautolist}
\end{answer}
\end{problem}

\begin{problem}
Problems \ref{pro:form-a} through \ref{pro:form-b} refer to the
differential form $$ \omega = x\d{y}\wedge
\d{z}+y\d{z}\wedge\d{x}+2z\d{x}\wedge \d{y}, $$ and the solid $M$
whose boundaries are the paraboloid $z=1-x^2-y^2$, $0 \leq z \leq 1$
and the disc $x^2+y^2\leq 1$, $z=0$. The surface $\partial M$ of the
solid is positively oriented upon considering  outward normals.
\begin{enumerate}
\item \label{pro:form-a}  Prove that $\d{\omega}=4 \d{x}\wedge\d{y}\wedge\d{z}$.
\item Prove that in Cartesian coordinates, $\dint _{\partial M}\omega =\dint _{-1} ^1  \dint _{-\sqrt{1-x^2}}
^{\sqrt{1-x^2}} \dint _{0} ^{1-x^2-y^2} 4\d{z}\d{y}\d{x}$.
\item Prove that in cylindrical coordinates, $\dint _{M}\d{\omega}=\dint _0 ^{2\pi}\dint
_0 ^1 \dint _0 ^{1-r^2} 4r\d{z}\d{r}\d{\theta} $.
\item  \label{pro:form-b} Prove that $ \dint _{\partial M} x\d{y}\d{z}+y\d{z}\d{x}+2z\d{x}\d{y} =2\pi. $
\end{enumerate}
\end{problem}

\begin{problem}
Problems \ref{pro:gauss-a} through \ref{pro:gauss-b} refer to the
box
$$M=\{(x,y,z)\in\reals^3: 0 \leq x \leq 1, \ 0 \leq y \leq 1, \ 0 \leq z \leq 2 \},  $$
the upper face of the box
$$U=\{(x,y,z)\in\reals^3: 0 \leq x \leq 1, \ 0 \leq y \leq 1, \ z=2 \},  $$
the boundary of the box without the upper top $S=\partial M\setminus
U$,
 and the differential form
$$ \omega = (\arctan y-x^2)\d{y}\wedge\d{z} + (\cos x\sin z-y^3)\d{z}\wedge\d{x} + (2zx+6zy^2)\d{x}\wedge\d{y}. $$


\begin{enumerate}
\item  \label{pro:gauss-a} Prove that
$\d{\omega}=3y^2\d{x}\wedge\d{y}\wedge\d{z}$.
\item Prove that $\dint _{\partial M} (\arctan y-x^2)\d{y}\d{z} + (\cos x\sin z-y^3)\d{z}\d{x} + (2zx+6zy^2)\d{x}\d{y}
=\dint _0 ^2 \dint _0 ^1 \dint _0 ^1 3y^2\d{x}\d{y}\d{z}=2$. Here
the boundary of the box is positively oriented considering outward
normals.
\item Prove that the integral on the
upper face of the box is $\dint _{U} (\arctan y-x^2)\d{y}\d{z} +
(\cos x\sin z-y^3)\d{z}\d{x} + (2zx+6zy^2)\d{x}\d{y}=\dint _0 ^1
\dint _0 ^1 4x+12y^2\d{x}\d{y}=6$.
\item  \label{pro:gauss-b} Prove that the integral on the open box is $\dint _{\partial M\setminus U} (\arctan y-x^2)\d{y}\d{z} + (\cos
x\sin z-y^3)\d{z}\d{x} + (2zx+6zy^2)\d{x}\d{y}=-4$.
\end{enumerate}
\end{problem}

\begin{problem}  Problems \ref{pro:triangle-a} through \ref{pro:triangle-b}
refer to a triangular surface  $T$ in $\reals^3$ and a differential
form $\omega$. The vertices of $T$ are at $A(6,0,0)$, $B(0,12,0)$,
and $C(0,0,3)$. The boundary of of the triangle $\partial T$ is
oriented positively by starting at $A$, continuing to $B$, following
to $C$, and ending again at  $A$. The surface $T$ is oriented
positively by considering the top of the triangle, as viewed from a
point far above the triangle. The differential form is
$$\omega = \left(2xz+\arctan e^x\right)\d{x} + \left(xz+(y+1)^y\right)\d{y} + \left(xy+\dfrac{y^2}{2}+\log (1+z^2)\right)\d{z}.   $$

\begin{enumerate}
\item  \label{pro:triangle-a} Prove that the equation of the plane that
 contains the triangle $T$ is $ 2x+y+4z=12 $.
 \item Prove that $\d{\omega}=y\d{y}\wedge\d{z}
+\left(2x-y\right)\d{z}\wedge\d{x}+z\d{x}\wedge\d{y}$.
\item  \label{pro:triangle-b} Prove that $\dint _{\partial T}  \left(2xz+\arctan e^x\right)\d{x} + \left(xz+(y+1)^y\right)\d{y} + \left(xy+\dfrac{y^2}{2}+\log
(1+z^2)\right)\d{z}=\dint _0 ^3 \dint _0 ^{12-4z} y\d{y}\d{z} +\dint
_0 ^6 \dint _0 ^{3-x/2} 2x\d{z}\d{x}$=108.
\end{enumerate}
\end{problem}


\begin{problem}
Use Green's Theorem to prove that
$$\dint _\Gamma (x^2 + 2y^3)\d{y} = 16\pi,
$$where $\Gamma$ is the circle $(x - 2)^2 + y^2 = 4$. Also, prove this
directly by using  a path integral. \begin{answer} Observe that
$$ \d{(x^2 + 2y^3)\wedge\d{y}} = 2x\d{x}\wedge\d{y}.$$Hence by the
generalised Stokes' Theorem the integral equals $$\dint
\limits_{\{(x - 2)^2 + y^2 \leq 4\}} 2x\d{x}\wedge\d{y} = \dint
_{-\pi/2} ^{\pi/2} \dint _0 ^{4\cos\theta} 2\rho^2 \cos\theta\d{\rho}
\wedge\d\theta = 16\pi .
$$
To do it directly, put $x - 2 = 2\cos t, y = 2\sin t, 0 \leq t \leq
2\pi$. Then the integral becomes
$$\begin{array}{lll}\dint _0 ^{2\pi} ((2 + 2\cos t)^2 + 16\sin^3t)\d{2\sin t} & = & \dint _0 ^{2\pi} (8\cos t + 16\cos^2 t\\ & & \qquad  + 8\cos^3t + 32\cos t\sin^3t) \d{t}\\ & =  & 16\pi.  \end{array}  $$

\end{answer}
\end{problem}

\begin{problem}
Let $\Gamma$ denote the curve of intersection of the plane $x + y =
2$ and the sphere $x^2 -2x + y^2 - 2y + z^2 = 0$, oriented clockwise
when viewed from the origin. Use Stoke's Theorem to prove that
$$\dint \limits_\Gamma y \d{x} + z\d{y} + x\d{z} = -2\pi\sqrt{2}.
$$ Prove this directly by parametrising the boundary of the
surface and evaluating the path integral. \begin{answer} At the
intersection path
$$ 0 = x^2 + y^2 + z^2 - 2(x + y) = (2 - y)^2 + y^2 + z^2 - 4 = 2y^2 - 4y + z^2 = 2(y - 1)^2 + z^2 - 2,  $$
which describes an ellipse on the $yz$-plane. Similarly we get $2(x
- 1)^2 + z^2 = 2$ on the $xz$-plane.  We have
$$ \d{\left(y\d{x} + z\d{y} + x\d{z}\right)}  = \d{y}\wedge\d{x} +
\d{z}\wedge\d{y} + \d{x}\wedge\d{z} = -\d{x}\wedge\d{y}
-\d{y}\wedge\d{z} - \d{z}\wedge\d{x}.$$ Since $\d{x}\wedge \d{y} =
0$, by Stokes' Theorem the integral sought is
$$ -\dint \limits_{2(y - 1)^2 + z^2 \leq 2} \d{y}\d{z}  -\dint \limits_{2(x- 1)^2 + z^2 \leq 2} \d{z}\d{x}
= -2\pi (\sqrt{2}). $$ (To evaluate the integrals you may resort to
the fact that  the area of the elliptical region
$\dfrac{(x-x_0)^2}{a^2} + \dfrac{(y-y_0)^2}{b^2} \leq 1$ is $\pi
ab)$.

\bigskip

If we were to evaluate this integral directly, we would set $$  y =
1 + \cos \theta, \ z =  \sqrt{2}\sin \theta , x = 2 - y = 1 -
\cos\theta .
$$The integral becomes
$$ \dint _0 ^{2\pi} (1 + \cos \theta)\d{(1 - \cos \theta)} + \sqrt{2}\sin\theta\d{(1 + \cos\theta)} + (1 - \cos\theta)\d{(\sqrt{2}\sin\theta)} $$
which in turn $$ = \dint_0 ^{2\pi} \sin\theta + \sin\theta\cos\theta
- \sqrt{2} + \sqrt{2}\cos\theta \d{\theta} = -2\pi\sqrt{2}.$$
\end{answer}
\end{problem}
\begin{problem}
Use Green's Theorem to evaluate
$$\doint _C (x^3-y^3)\d{x} +(x^3+y^3)\d{y},  $$
where $C$ is the  positively oriented boundary of the region between
the circles $x^2 +y^2 = 2$ and $x^2 +y^2 = 4$.
\end{problem}





%\usepackage[fleqn]{amsmath}
% \setlength{\mathindent}{3.5em}
\usepackage{marginnote}
\reversemarginpar
\usepackage{url}
\usepackage{xcolor}
\usepackage{colortbl}
 \usepackage[disable]{todonotes}
\usepackage{overpic}
\usepackage{rotating}
\usepackage[explicit]{titlesec}
\usepackage{type1cm}
\usepackage{eso-pic}
\usepackage{chngpage,calc}
\usepackage{asymptote}
\usepackage{pstricks}
\usepackage{tabularx}
\usepackage{pstricks,pstricks-add,pst-math,pst-xkey,pst-func}
\usepackage{caption}
\usepackage{pgf,tikz}
\usepackage{pgfplots}
\usetikzlibrary[patterns]
\pgfplotsset{compat=newest}
\usepackage{mathrsfs}
\usepackage{framed}
\usepackage{xspace}
% \usepackage{enumerate}
\usepackage{verbatim} %ambiente comment dentre outros
\usepackage{scalefnt}
\usepackage{commath}
\usepackage{vwcol}   %%%%Colunas de larguras diferentes
\usetikzlibrary{arrows}
% \usepackage[includeheadfoot,includemp,marginpar=5cm,marginparsep=0.8cm,top=3cm,bottom=3cm,left=0.5cm,right=1.3cm,headsep=10pt,letterpaper]{geometry} % Page margins
%%%%%%%%%%%%O padrão é3cm e a4paper
 % Specifies the directory where pictures are stored

\usepackage{tikz,pgfplots} % Required for drawing custom shapes
\usepackage{fix-cm}
% \usepackage{enumitem} % Customize lists
\usepackage{polynom}
% \setlist{nolistsep} % Reduce spacing between bullet points and numbered lists
\usepackage{booktabs} % Required for nicer horizontal rules in tables
\usepackage{xcolor,nicefrac}

%%%%%%%%%%%%%%%%%%%%%%%%%%%%%%%%%%
%%%%%%%%%%%%%%%%%%%%%%%%%%%%%%%%%% Cores
\definecolor{ocre}{cmyk}{0, 0.87, 0.68, 0.32} %Maroon
\definecolor{Maroon}{RGB}{30,144,255}
\definecolor{azul}{RGB}{30,144,255}
\definecolor{azulescuro}{RGB}{0.252, 0.419,0.584}
\definecolor{verde}{RGB}{127, 137, 97}
\definecolor{vermelho}{cmyk}{0, 0.87, 0.68, 0.32}
\definecolor{laranja}{RGB}{243,102,25}
\definecolor{blue}{rgb}{0.252, 0.419,0.584}
\definecolor{red}{cmyk}{0, 0.87, 0.68, 0.32}
\definecolor{amarelinho}{RGB}{255,249,215}
\definecolor{azulzinho}{RGB}{238,247,250}


%----------------------------------------------------------------------------------------
%	FONTS
%----------------------------------------------------------------------------------------

\usepackage{avant} % Use the Avantgarde font for headings
\usepackage{helvet}
\usepackage{mathptmx} % Use the Adobe Times Roman as the default text font together with math symbols from the Sym­bol, Chancery and Com­puter Modern fonts
\usepackage{eulervm}
%\usepackage[bitstream-charter]{mathdesign}
\usepackage{microtype} % Slightly tweak font spacing for aesthetics
\usepackage[T1]{fontenc} %----------------------------------------------------------------------------------------
%	BIBLIOGRAPHY AND INDEX
%----------------------------------------------------------------------------------------

%\addbibresource{bibliography.bib} % BibTeX bibliography file
\usepackage{calc} % For simpler calculation - used for spacing the index letter headings correctly
\usepackage{makeidx} % Required to make an index
\makeindex % Tells LaTeX to create the files required for indexing

%----------------------------------------------------------------------------------------
%	MAIN TABLE OF CONTENTS
%----------------------------------------------------------------------------------------

\usepackage{titletoc} % Required for manipulating the table of contents
\contentsmargin{0cm} % Removes the default margin
% Part text styling
\titlecontents{part}[0cm]
{\addvspace{20pt}\centering\large\bfseries}
{}
{}
{}

%Chapter text styling
\titlecontents{chapter}[1.25cm] % Indentation
{\addvspace{12pt}\Large\sffamily\bfseries} % Spacing and font options for chapters
{\color{Maroon!60}\contentslabel[\Large\thecontentslabel]{1.25cm}\color{Maroon}} % Chapter number
{\color{Maroon}}
{\color{Maroon!60}\normalsize\;\titlerule*[.5pc]{.}\;\thecontentspage} % Page number


\titlecontents{chapter}[1.25cm] % Indentation
{\addvspace{12pt}\Large\sffamily\bfseries} % Spacing and font options for chapters
{\color{Maroon!60}\contentslabel[\Large\thecontentslabel]{1.25cm}\color{Maroon}} % Chapter number
{\color{Maroon}}
{\color{Maroon!60}\normalsize\;\titlerule*[.5pc]{.}\;\thecontentspage} % Page number





% Section text styling
\titlecontents{section}[1.25cm] % Indentation
{\addvspace{3pt}\large\sffamily\bfseries} % Spacing and font options for sections
{\contentslabel[\thecontentslabel]{1.25cm}} % Section number
{}
{\hfill\color{black}\thecontentspage} % Page number
[]

% Subsection text styling
\titlecontents{subsection}[1.25cm] % Indentation
{\addvspace{1pt}\sffamily\small} % Spacing and font options for subsections
{\contentslabel[\thecontentslabel]{1.25cm}} % Subsection number
{}
{\ \titlerule*[.5pc]{.}\;\thecontentspage} % Page number
[]

% List of figures
\titlecontents{figure}[0em]
{\addvspace{-5pt}\sffamily}
{\thecontentslabel\hspace*{1em}}
{}
{\ \titlerule*[.5pc]{.}\;\thecontentspage}
[]

% List of tables
\titlecontents{table}[0em]
{\addvspace{-5pt}\sffamily}
{\thecontentslabel\hspace*{1em}}
{}
{\ \titlerule*[.5pc]{.}\;\thecontentspage}
[]

%----------------------------------------------------------------------------------------
%	MINI TABLE OF CONTENTS IN PART HEADS
%----------------------------------------------------------------------------------------

% Chapter text styling
\titlecontents{lchapter}[0em] % Indenting
{\addvspace{15pt}\large\sffamily\bfseries} % Spacing and font options for chapters 15
{\color{Maroon}\contentslabel[\Large\thecontentslabel]{1.25cm}\color{Maroon}} % Chapter number
{}
{\color{Maroon}\normalsize\sffamily\bfseries\;\titlerule*[.5pc]{.}\;\thecontentspage} % Page number

% Section text styling
\titlecontents{lsection}[0em] % Indenting
{\sffamily\small} % Spacing and font options for sections
{\contentslabel[\thecontentslabel]{1.25cm}} % Section number
{}
{}

% Subsection text styling
\titlecontents{lsubsection}[.5em] % Indentation
{\normalfont\footnotesize\sffamily} % Font settings
{}
{}
{}

%----------------------------------------------------------------------------------------
%	PAGE HEADERS
%----------------------------------------------------------------------------------------

\usepackage{fancyhdr,calc} % Required for header and footer configuration

\newlength{\myoddoffset}
\setlength{\myoddoffset}{\marginparwidth + \marginparsep}
\pagestyle{fancy}
\fancyheadoffset[leh,roh]{\marginparsep}
\fancyheadoffset[loh,reh]{\myoddoffset}
\renewcommand{\chaptermark}[1]{\markboth{\sffamily\normalsize\bfseries\chaptername\ \thechapter.\ #1}{}} % Chapter text font settings
\renewcommand{\sectionmark}[1]{\markright{\sffamily\normalsize\thesection\hspace{5pt}#1}{}} % Section text font settings
\fancyhf{} \fancyhead[LE,RO]{\sffamily\normalsize\thepage} % Font setting for the page number in the header
\fancyhead[LO]{\rightmark} % Print the nearest section name on the left side of odd pages
\fancyhead[RE]{\leftmark} % Print the current chapter name on the right side of even pages
\renewcommand{\headrulewidth}{0.5pt} % Width of the rule under the header
\addtolength{\headheight}{2.5pt} % Increase the spacing around the header slightly
\renewcommand{\footrulewidth}{0pt} % Removes the rule in the footer
\fancypagestyle{plain}{\fancyhead{}\renewcommand{\headrulewidth}{0pt}} % Style for when a plain pagestyle is specified

% Removes the header from odd empty pages at the end of chapters
\makeatletter
\renewcommand{\cleardoublepage}{
\clearpage\ifodd\c@page\else
\hbox{}
\vspace*{\fill}
\thispagestyle{empty}
\newpage
\fi}

%----------------------------------------------------------------------------------------
%	THEOREM STYLES
%----------------------------------------------------------------------------------------
%\newcommand{\intoo}[2]{\mathopen{]}#1\,;#2\mathclose{[}}
\newcommand{\ud}{\mathop{\mathrm{{}d}}\mathopen{}}
\newcommand{\intff}[2]{\mathopen{[}#1\,;#2\mathclose{]}}
\newtheorem{notation}{Notation}[chapter]

% Boxed/framed environments
\newtheoremstyle{ocrenumbox}% % Theorem style name
{0pt}% Space above
{0pt}% Space below
{\normalfont}% % Body font
{}% Indent amount
{\small\bf\sffamily\color{ocre}}% % Theorem head font
{\;}% Punctuation after theorem head
{0.25em}% Space after theorem head
{\small\sffamily\color{ocre}\thmname{#1}\nobreakspace\thmnumber{\@ifnotempty{#1}{}\@upn{#2}}% Theorem text (e.g. Theorem 2.1)
\thmnote{\nobreakspace\the\thm@notefont\sffamily\bfseries\color{black}---\nobreakspace#3.}} % Optional theorem note
\renewcommand{\qedsymbol}{\tiny$\blacksquare$}% Optional qed square

\newtheoremstyle{blacknumex}% Theorem style name
{5pt}% Space above
{5pt}% Space below
{\normalfont}% Body font
{} % Indent amount
{\small\bfseries\sffamily\scshape \color{Maroon}}% Theorem head font
{\;}% Punctuation after theorem head
{0.25em}% Space after theorem head
{\small\sffamily{\tiny\ensuremath{\blacksquare}}\nobreakspace\thmname{#1}\nobreakspace\thmnumber{\@ifnotempty{#1}{}\@upn{#2}}% Theorem text (e.g. Theorem 2.1)
\thmnote{\nobreakspace\the\thm@notefont\sffamily\bfseries---\nobreakspace#3.}}% Optional theorem note


\newtheoremstyle{azulnumbox} % Theorem style name
{0pt}% Space above
{0pt}% Space below
{\normalfont}% Body font
{}% Indent amount
{\small\bfseries\sffamily \color{azul} }% Theorem head font
{\;}% Punctuation after theorem head
{0.25em}% Space after theorem head
{\small\sffamily\thmname{#1}\nobreakspace\thmnumber{\@ifnotempty{#1}{}\@upn{#2}}% Theorem text (e.g. Theorem 2.1)
\thmnote{\nobreakspace\the\thm@notefont\sffamily\bfseries---\nobreakspace#3.}}% Optional theorem note



\newtheoremstyle{blacknumbox} % Theorem style name
{0pt}% Space above
{0pt}% Space below
{\normalfont}% Body font
{}% Indent amount
{\small\bfseries\sffamily }% Theorem head font
{\;}% Punctuation after theorem head
{0.25em}% Space after theorem head
{\small\sffamily\thmname{#1}\nobreakspace\thmnumber{\@ifnotempty{#1}{}\@upn{#2}}% Theorem text (e.g. Theorem 2.1)
\thmnote{\nobreakspace\the\thm@notefont\sffamily\bfseries---\nobreakspace#3.}}% Optional theorem note

% Non-boxed/non-framed environments
\newtheoremstyle{ocrenum}% % Theorem style name
{5pt}% Space above
{5pt}% Space below
{\normalfont}% % Body font
{}% Indent amount
{\small\bfseries\sffamily\color{ocre}}% % Theorem head font
{\;}% Punctuation after theorem head
{0.25em}% Space after theorem head
{\small\sffamily\color{ocre}\thmname{#1}\nobreakspace\thmnumber{\@ifnotempty{#1}{}\@upn{#2}}% Theorem text (e.g. Theorem 2.1)
\thmnote{\nobreakspace\the\thm@notefont\sffamily\bfseries\color{black}---\nobreakspace#3.}} % Optional theorem note
\renewcommand{\qedsymbol}{$\blacksquare$}% Optional qed square
\makeatother

% Defines the theorem text style for each type of theorem to one of the three styles above
\newcounter{dummy}
\numberwithin{dummy}{section}
\theoremstyle{ocrenumbox}
\newtheorem{theoremeT}[dummy]{Theorem}
\newtheorem{problem}{Problem}[chapter]
\newtheorem{exerciseT}{Exercise}[chapter]
\theoremstyle{blacknumex}
\newtheorem{exampleT}{Exemplo}[chapter]
\theoremstyle{blacknumbox}
\newtheorem{vocabulary}{Vocabulary}[chapter]
\newtheorem{corollaryT}[dummy]{Corollary}
\theoremstyle{ocrenum}
\newtheorem{lemma}[dummy]{Lemma}


\theoremstyle{azulnumbox}
\newtheorem{definitionT}{Definition}[section]

%----------------------------------------------------------------------------------------
%	DEFINITION OF COLORED BOXES
%----------------------------------------------------------------------------------------
%\usepackage[framemethod=tikz]{mdframed}
\RequirePackage[framemethod=default]{mdframed} % Required for creating the theorem, definition, exercise and corollary boxes

% Theorem box
\newmdenv[skipabove=7pt,
skipbelow=7pt,
%backgroundcolor=ocre!1.5,
linecolor=ocre,
innerleftmargin=5pt,
innerrightmargin=5pt,
innertopmargin=10pt,
leftmargin=0.1cm,
rightmargin=0.1cm,
innerbottommargin=10pt,nobreak=true]{tBox}

% Exercise box
\newmdenv[skipabove=7pt,
skipbelow=7pt,
rightline=false,
leftline=true,
topline=false,
bottomline=false,
backgroundcolor=ocre!10,
linecolor=ocre,
innerleftmargin=5pt,
innerrightmargin=5pt,
innertopmargin=5pt,
innerbottommargin=5pt,
leftmargin=0cm,
rightmargin=0cm,
linewidth=4pt]{eBox}

% Definition box
\newmdenv[skipabove=7pt,
backgroundcolor=mybluei!1,
skipbelow=7pt,
rightline=true,
leftline=true,
topline=true,
bottomline=true,
linecolor=mybluei,
innerleftmargin=5pt,
innerrightmargin=5pt,
innertopmargin=10pt,
leftmargin=0.1cm,
rightmargin=0.1cm,
linewidth=0.5pt,
innerbottommargin=10pt,nobreak=true]{dBox}

% Corollary box
\newmdenv[skipabove=7pt,
skipbelow=7pt,
rightline=false,
leftline=true,
topline=false,
bottomline=false,
linecolor=gray,
backgroundcolor=black!5,
innerleftmargin=5pt,
innerrightmargin=5pt,
innertopmargin=5pt,
leftmargin=0cm,
rightmargin=0cm,
linewidth=4pt,
innerbottommargin=5pt]{cBox}

% Creates an environment for each type of theorem and assigns it a theorem text style from the "Theorem Styles" section above and a colored box from above
\newenvironment{theorem}{\begin{tBox}\begin{theoremeT}}{\end{theoremeT}\end{tBox}}
\newenvironment{exercise}{\begin{eBox}\begin{exerciseT}}{\hfill{\color{ocre}\tiny\ensuremath{\blacksquare}}\end{exerciseT}\end{eBox}}
\newenvironment{definition}{\begin{dBox}\begin{definitionT}}{\end{definitionT}\end{dBox}}
\newenvironment{example}{\begin{exampleT}}{\hfill{\tiny\ensuremath{\blacksquare}}\end{exampleT}}
\newenvironment{corollary}{\begin{cBox}\begin{corollaryT}}{\end{corollaryT}\end{cBox}}
\newenvironment{cor}{\begin{cBox}\begin{corollaryT}}{\end{corollaryT}\end{cBox}}


%----------------------------------------------------------------------------------------
%	REMARK ENVIRONMENT
%----------------------------------------------------------------------------------------

\newenvironment{remark}{\par\vspace{10pt}\small % Vertical white space above the remark and smaller font size
\begin{list}{}{
\leftmargin=35pt % Indentation on the left
\rightmargin=25pt}\item\ignorespaces % Indentation on the right
\makebox[-2.5pt]{\begin{tikzpicture}[overlay]
\node[draw=Maroon!60,line width=1pt,circle,fill=Maroon!25,font=\sffamily\bfseries,inner sep=2pt,outer sep=0pt] at (-15pt,0pt){\textcolor{Maroon}{R}};\end{tikzpicture}} % Orange R in a circle
\advance\baselineskip -1pt}{\end{list}\vskip5pt} % Tighter line spacing and white space after remark

%----------------------------------------------------------------------------------------
%	SECTION NUMBERING IN THE MARGIN
%----------------------------------------------------------------------------------------

\makeatletter
\renewcommand{\@seccntformat}[1]{\llap{\textcolor{ocre}{\csname the#1\endcsname}\hspace{1em}}}
\renewcommand{\section}{\@startsection{section}{1}{\z@}
{-4ex \@plus -1ex \@minus -.4ex}
{1ex \@plus.2ex }
{\normalfont\large\sffamily\bfseries}}
\renewcommand{\subsection}{\@startsection {subsection}{2}{\z@}
{-3ex \@plus -0.1ex \@minus -.4ex}
{0.5ex \@plus.2ex }
{\normalfont\sffamily\bfseries}}
\renewcommand{\subsubsection}{\@startsection {subsubsection}{3}{\z@}
{-2ex \@plus -0.1ex \@minus -.2ex}
{.2ex \@plus.2ex }
{\normalfont\small\sffamily\bfseries}}
\renewcommand\paragraph{\@startsection{paragraph}{4}{\z@}
{-2ex \@plus-.2ex \@minus .2ex}
{.1ex}
{\normalfont\small\sffamily\bfseries}}

%----------------------------------------------------------------------------------------
%	PART HEADINGS
%----------------------------------------------------------------------------------------

% numbered part in the table of contents
\newcommand{\@mypartnumtocformat}[2]{%
\setlength\fboxsep{0pt}%
\noindent\colorbox{Maroon!20}{\strut\parbox[c][.7cm]{\ecart}{\color{Maroon!70}\Large\sffamily\bfseries\centering#1}}\hskip\esp\colorbox{Maroon!40}{\strut\parbox[c][.7cm]{\linewidth-\ecart-\esp}{\Large\sffamily\centering#2}}}%
%%%%%%%%%%%%%%%%%%%%%%%%%%%%%%%%%%
% unnumbered part in the table of contents
\newcommand{\@myparttocformat}[1]{%
\setlength\fboxsep{0pt}%
\noindent\colorbox{Maroon!40}{\strut\parbox[c][.7cm]{\linewidth}{\Large\sffamily\centering#1}}}%
%%%%%%%%%%%%%%%%%%%%%%%%%%%%%%%%%%
\newlength\esp
\setlength\esp{4pt}
\newlength\ecart
\setlength\ecart{1.2cm-\esp}
\newcommand{\thepartimage}{}%
\newcommand{\partimage}[1]{\renewcommand{\thepartimage}{#1}}%
\def\@part[#1]#2{%
\ifnum \c@secnumdepth >-2\relax%
\refstepcounter{part}%
\addcontentsline{toc}{part}{\texorpdfstring{\protect\@mypartnumtocformat{\thepart}{#1}}{\partname~\thepart\ ---\ #1}}
\else%
\addcontentsline{toc}{part}{\texorpdfstring{\protect\@myparttocformat{#1}}{#1}}%
\fi%
\startcontents%
\markboth{}{}%
{\thispagestyle{empty}%
\begin{tikzpicture}[remember picture,overlay]%
\node at (current page.north west){\begin{tikzpicture}[remember picture,overlay]%
\fill[Maroon!20](0cm,1.8cm) rectangle (\paperwidth,-\paperheight+1.6cm);
\node[anchor=north] at (2cm,-2.25cm){\color{Maroon!40}\fontsize{220}{100}\sffamily\bfseries\@Roman\c@part};
\node[anchor=south east] at (\paperwidth-1cm,-\paperheight+2cm){\parbox[t][][t]{8.5cm}{
\printcontents{l}{0}{\setcounter{tocdepth}{1}}%
}};
\node[anchor=north east] at (\paperwidth-1.5cm,-3.25cm){\parbox[t][][t]{15cm}{\strut\raggedleft\color{white}\fontsize{30}{30}\sffamily\bfseries#2}};
\end{tikzpicture}};
\end{tikzpicture}}%
\@endpart}
\def\@spart#1{%
\startcontents%
\phantomsection
{\thispagestyle{empty}%
\begin{tikzpicture}[remember picture,overlay]%
\node at (current page.north west){\begin{tikzpicture}[remember picture,overlay]%
\fill[Maroon!20](0cm,0cm) rectangle (\paperwidth,-\paperheight);
\node[anchor=north east] at (\paperwidth-1.5cm,-3.25cm){\parbox[t][][t]{15cm}{\strut\raggedleft\color{white}\fontsize{30}{30}\sffamily\bfseries#1}};
\end{tikzpicture}};
\end{tikzpicture}}
\addcontentsline{toc}{part}{\texorpdfstring{%
\setlength\fboxsep{0pt}%
\noindent\protect\colorbox{Maroon!40}{\strut\protect\parbox[c][.7cm]{\linewidth}{\Large\sffamily\protect\centering #1\quad\mbox{}}}}{#1}}%
\@endpart}
\def\@endpart{\vfil\newpage
\if@twoside
\if@openright
\null
\thispagestyle{empty}%
\newpage
\fi
\fi
\if@tempswa
\twocolumn
\fi}

% ----------------------------------------------------------------------------------------
% 	CHAPTER HEADINGS
% ----------------------------------------------------------------------------------------

% \newcommand{\thechapterimage}{}%
% \newcommand{\chapterimage}[1]{\renewcommand{\thechapterimage}{#1}}%
% \def\@makechapterhead#1{%
% {\parindent \z@ \raggedright \normalfont
% \ifnum \c@secnumdepth >\m@ne
% \if@mainmatter
% \begin{tikzpicture}[remember picture,overlay]
% \node at (current page.north west)
% {\begin{tikzpicture}[remember picture,overlay]
% \node[anchor=north west,inner sep=0pt] at (0,0) {\includegraphics[width=\paperwidth]{\thechapterimage}};
% \draw[anchor=west] (\Gm@lmargin,-7.5cm) node [line width=2pt,rounded corners=15pt,draw=ocre,fill=white,fill opacity=0.65,inner sep=15pt]{\strut\makebox[22cm]{}};
% \draw[anchor=west] (\Gm@lmargin+.3cm,-7.5cm) node {\huge\sffamily\bfseries\color{black}\thechapter. #1\strut};
% \end{tikzpicture}};
% \end{tikzpicture}
% \else
% \begin{tikzpicture}[remember picture,overlay]
% \node at (current page.north west)
% {\begin{tikzpicture}[remember picture,overlay]
% \node[anchor=north west,inner sep=0pt] at (0,0) {\includegraphics[width=\paperwidth]{\thechapterimage}};
% \draw[anchor=west] (\Gm@lmargin,-7.5cm) node [line width=2pt,rounded corners=15pt,draw=ocre,fill=white,fill opacity=0.5,inner sep=15pt]{\strut\makebox[25cm]{}};
% \draw[anchor=west] (\Gm@lmargin+.3cm,-7.5cm) node {\huge\sffamily\bfseries\color{black}#1\strut};
% \end{tikzpicture}};
% \end{tikzpicture}
% \fi\fi\par\vspace*{200\p@}}}
%
% %-------------------------------------------
%
% \def\@makeschapterhead#1{%
% \begin{tikzpicture}[remember picture,overlay]
% \node at (current page.north west)
% {\begin{tikzpicture}[remember picture,overlay]
% \node[anchor=north west,inner sep=0pt] at (0,0) {\includegraphics[width=\paperwidth]{\thechapterimage}};
% \draw[anchor=west] (\Gm@lmargin,-7.5cm) node [line width=1.5pt,rounded corners=7pt,draw=ocre,fill=white,fill opacity=0.6,inner sep=15pt]{\strut\makebox[29cm]{}};
% \draw[anchor=west] (\Gm@lmargin+.3cm,-7.5cm) node {\huge\sffamily\bfseries\color{black}#1\strut};
% \end{tikzpicture}};
% \end{tikzpicture}
% \par\vspace*{200\p@}}
% \makeatother







%----------------------------------------------------------------------------------------
%	HYPERLINKS IN THE DOCUMENTS
%----------------------------------------------------------------------------------------

\usepackage{hyperref}
\hypersetup{hidelinks,backref=true,pagebackref=true,hyperindex=true,colorlinks=false,breaklinks=true,urlcolor= Maroon,bookmarks=true,bookmarksopen=false,pdftitle={Title},pdfauthor={Author}}
\usepackage{bookmark}
\bookmarksetup{
open,
numbered,
addtohook={%
\ifnum\bookmarkget{level}=0 % chapter
\bookmarksetup{bold}%
\fi
\ifnum\bookmarkget{level}=-1 % part
\bookmarksetup{color=Maroon,bold}%
\fi
}
}


\renewcommand\labelitemi{\scalebox{0.7}{$\blacksquare$}}


%%%%%%%%%%%%%%%%%%%%%%CAPA
\newcommand\BackgroundPic{
\put(-2,0){
\parbox[b][\paperheight]{\paperwidth}{%
\vfill
\centering
\includegraphics[width=\paperwidth,height=\paperheight+3.9cm,
keepaspectratio]{./graficos/capa.eps}%
\vfill
}}}

\newcommand{\titulo}{Vector Calculus}
\newcommand{\subtitulo}{Frankenstein's Note}
\newcommand{\autores}{Daniel Miranda}
\newcommand{\universidade}{Universidade Federal do ABC}
\newcommand{\lugar}{Santo André}
\newcommand{\data}{05/2017}
\newcommand{\versao}{Version  0.9}
\newcommand{\pagweb}{ \textcolor{ocre}{\url{http://}}}


\setlength{\marginparpush}{2ex}


% A new command "\mfigure" to put images to the margin:
% Usage: \mfigure{image_file}{Image caption text \label{fg:img1}}
%
% The width of the image can be justified with an additional parameter
% between (0,1] which is relational to the \marginparwidth
% Example:\mfigure[0.5]{image_file}{Image caption...}
%
\newcounter{marg_fig_counter}[section]

\newcommand{\mfigure}[4]{
\begin{figure}
\includegraphics[width=4cm]{#4}
\captionof{figure}{#2}
\label{#3}
\end{figure}
}

\newcommand{\mfiguret}[4]{
\begin{figure}
\input{#4}
\captionof{figure}{#2}
\label{#3}
\end{figure}}

\makeatletter
\newcommand\HUGE{\@setfontsize\Huge{40}{40}}
\makeatother

% \newlength\BoxWd
% \setlength\BoxWd{1cm}
% \newlength\Aux
%
% \titleformat{\chapter}[display]
%   {\normalfont\sffamily\bfseries\LARGE}
%   {\renewcommand{\thechapter}{\arabic{chapter}}\hspace*{0.5em}\colorbox{ocre}{%
%     \parbox[c][1.5cm][c]{1.5cm}{%
%       \centering\textcolor{white}  {\HUGE\thechapter}}}}
%   {-1ex}
%   {\color{ocre}\Huge \titlerule[4pt]\vspace{.7ex}\filleft\MakeUppercase{#1}}
%   [\vspace{.2ex}]
% % chapter tiltes spacing
% \titlespacing*{\chapter}{0pt}{50pt}{80pt}


\definecolor{mybluei}{RGB}{0,173,239}
\definecolor{myblueii}{RGB}{63,200,244}
\definecolor{myblueiii}{RGB}{199,234,253}

 \usepackage[includeheadfoot,top=2cm,bottom=2.7cm,left=3.2cm,right=3.2cm,headsep=10pt,letterpaper]{geometry}

\renewcommand\thepart{\arabic{chapter}}

\newcommand\partnumfont{% font specification for the number
  \fontsize{320}{130}\color{myblueii}\selectfont \bfseries
}

\newcommand\partnamefont{% font specification for the name "PART"
  \normalfont\color{white}\scshape\small\bfseries
}

\titleformat{\chapter}[display]
  {\normalfont\huge\filleft}
  {}
  {20pt}
  {\begin{tikzpicture}[remember picture,overlay]
  \fill[myblueiii]
    (current page.north west) rectangle ([yshift=-7.8cm]current page.north east);
  \node[
      fill=mybluei,
      text width=2\paperwidth,
      rounded corners=3cm,
      text depth=11.9cm,
      anchor=center,
      inner sep=0pt] at (current page.north east) (parttop)
    {\thepart};%
  \node[
      anchor=south east,
      inner sep=0pt,
      outer sep=0pt] (partnum) at ([xshift=-40pt]parttop.south)
    {\partnumfont\thepart};
  \node[
      anchor=south,
      inner sep=0pt] (partname) at ([yshift=2pt]partnum.south)
  {\partnamefont Chapter};
  \node[
      anchor=north east,
      align=right,
      inner xsep=0pt] at ([yshift=-0.0cm]partname.east|-partnum.south)
  {\parbox{1.4\textwidth}{\vspace{0.2cm}\color{ocre}\sffamily\bfseries\Huge\raggedleft#1}};
  \end{tikzpicture}%
  \vspace{1.7cm}
  }

 \titleformat{name=\chapter,numberless}[block]
 {\normalfont\huge\filleft}
  {}
  {20pt}
  {\begin{tikzpicture}[remember picture,overlay]
  \fill[myblueiii]
    (current page.north west) rectangle ([yshift=-7.6cm]current page.north east);
  \node[
      fill=mybluei,
      text width=2\paperwidth,
      rounded corners=3cm,
      text depth=11.4cm,
      anchor=center,
      inner sep=0pt] at (current page.north east) (parttop)
    {\thepart};%
  \node[
      anchor=south east,
      inner sep=0pt,
      outer sep=0pt] (partnum) at ([xshift=-20pt]parttop.south)
    {};
  \node[
      anchor=south,
      inner sep=0pt] (partname) at ([yshift=2pt]partnum.south)
  {\partnamefont };
  \node[
      anchor=north east,
      align=right,
      inner xsep=0pt] at ([yshift=-0.0cm]partname.east|-partnum.south)
  {\parbox{1.4\textwidth}{\vspace{0.2cm}\color{ocre}\sffamily\bfseries\Huge\raggedleft#1}};
  \end{tikzpicture}%
  \vspace{1.6cm}
  }

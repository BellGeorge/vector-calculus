
  \chapter{Multiple Integrals}%endchapter

  \section{Double Integrals}

  Let $R = [a, b] \times [c, d] \subseteq \bbR^2$ be a rectangle, and $f:R \to \bbR$ be continuous.
  Let $P = \set{ x_0, \dots, x_M, y_0, \dots, y_M}$ where $a = x_0 < x_1 < \cdots < x_M = b$ and $c = y_0 < y_1 < \cdots < y_M = d$.
  The set $P$ determines a partition of $R$ into a grid of (non-overlapping) rectangles $R_{i,j} = [x_i, x_{i+1}] \times [y_j, y_{j+1}]$ for $0 \leq i < M$ and $0 \leq j < N$.
  Given $P$, choose a collection of points $\Xi = \set{\xi_{i,j}}$ so that $\xi_{i,j} \in R_{i,j}$ for all $i, j$.
  \begin{definition}
    The \emph{Riemann sum} of $f$ with respect to the partition $P$ and points $\Xi$ is defined by
    \begin{equation*}
      \mathcal R(f, P, \Xi)
	\defeq \dsum_{i=0}^{M-1} \dsum_{j = 0}^{N-1}
	  f(\xi_{i, j}) \area(R_{i,j})
	= \dsum_{i=0}^{M-1} \dsum_{j = 0}^{N-1}
	  f(\xi_{i, j}) (x_{i+1} - x_i) (y_{j+1} - y_j)
    \end{equation*}
  \end{definition}
  

  \newcommand\DrawBlock[3]{
\ifx#1b\relax
  \path[draw]
    (lm\the\numexpr#2-1\relax) -- ++(0,0,#3) coordinate (blocklf)
    (bm\the\numexpr#2-1\relax) -- ++(0,0,#3) coordinate (blocklb)
    (lm#2) -- ++(0,0,#3) coordinate (blockrf)
    (bm#2) -- ++(0,0,#3) coordinate (blockrb);
  \filldraw[fill=white,draw=black]
    (lm\the\numexpr#2-1\relax) -- (blocklf) -- (blocklb) -- (blockrb) -- 
(blockrf) -- (lm#2);
\else  
  \ifx#1f\relax
    \path[draw]
      (fm\the\numexpr#2-1\relax) -- ++(0,0,#3) coordinate (blocklf)
      (lm\the\numexpr#2-1\relax) -- ++(0,0,#3) coordinate (blocklb)
      (fm#2) -- ++(0,0,#3) coordinate (blockrf)
      (lm#2) -- ++(0,0,#3) coordinate (blockrb);
    \filldraw[fill=white,draw=black]
      (fm\the\numexpr#2-1\relax) -- (blocklf) -- (blocklb) -- (blockrb) -- 
(blockrf) -- (fm#2);
  \fi
\fi
\draw (blocklf) -- (blockrf);
}

\begin{tikzpicture}[y={(0:1cm)},x={(225:0.86cm)}, z={(90:1cm)}]

% coordinates for the lower grid
\path
  (1,3,0) coordinate (bm0) -- 
  (4,3,0) coordinate (fm0) coordinate[midway] (lm0) --
  (4,8,0) coordinate[pos=0.25] (fm1) coordinate[midway] (fm2) 
coordinate[pos=0.75] (fm3) coordinate (fm4) --
  (1,8,0) coordinate (bm4) coordinate[midway] (lm4)--
  (bm0) coordinate[pos=0.25] (bm3) coordinate[midway] (bm2) 
coordinate[pos=0.75] 
(bm1);
\draw[dashed]
  (lm0) -- 
  (lm4) coordinate[pos=0.25] (lm1) coordinate[midway] (lm2) 
coordinate[pos=0.75] 
(lm3);

% the blocks
\DrawBlock{b}{1}{4}
\DrawBlock{b}{2}{3.7}
\DrawBlock{b}{3}{4.3}
\DrawBlock{b}{4}{5}
\DrawBlock{f}{1}{3.3}
\DrawBlock{f}{2}{3.5}
\DrawBlock{f}{3}{4}
\DrawBlock{f}{4}{4.7}

\foreach \Point/\Height in {lm1/3.7,lm2/4.3,lm3/5}
  \draw[ultra thin,dashed,opacity=0.2] (\Point) -- ++(0,0,\Height);

% the lower grid
\foreach \x in {1,2,3}
  \draw[dashed] (fm\x) -- (bm\x);
\draw[dashed] (fm0) -- (bm0) -- (bm4);
\draw (fm0) -- (fm4) -- (bm4);
\draw[dashed] (lm0) -- (lm4);

% coordinates for the surface
\coordinate (curvefm0) at ( $ (fm0) + (0,0,4) $ );
\coordinate (curvebm0) at ( $ (bm0) + (0,0,4) $ );
\coordinate (curvebm4) at ( $ (bm4) + (0,0,6) $ );
\coordinate (curvefm4) at ( $ (fm4) + (0,0,5.7) $ );

% the surface
\filldraw[ultra thick,fill=gray!25,fill opacity=0.2]
  (curvefm0) to[out=-30,in=210] 
  (curvefm4) to[out=-4,in=260]
  (curvebm4) to[out=215,in=330]
  (curvebm0) to[out=240,in=-20]
  (curvefm0);

% lines from grid to surface
\draw[very thick,name path=leftline] (curvefm0) -- (fm0);
\draw[very thick] (curvefm4) -- (fm4);
\draw[very thick,name path=rightline] (curvebm4) -- (bm4);
\draw[very thick,dashed] (curvebm0) -- (bm0);

% coordinate system
\coordinate (O) at (0,0,0);
\draw[-latex] (O) -- +(5,0,0) node[above left] {$x$};
\path[name path=yaxis] (O) -- +(0,10,0) coordinate (yaxisfinal) node[above] 
{$y$};
\draw[-latex] (O) -- +(0,0,5) node[left] {$z$};
\path[name intersections={of=yaxis and leftline,by={yaxis1}}];
\path[name intersections={of=yaxis and rightline,by={yaxis2}}];
\draw (O) -- (yaxis1);
\draw[densely dashed,opacity=0.1] (yaxis1) -- (yaxis2);
\draw[-latex] (yaxis2) -- (yaxisfinal);

% for debugging
%\foreach \Name in 
%curvefm0,curvebm0,curvebm4,curvefm4}
%  \node at (\Name) {\Name};  
\end{tikzpicture}
  
  \begin{definition}
    The \emph{mesh size} of a partition $P$ is defined by
    \begin{equation*}
      \norm{P} = \max \set{ x_{i+1} - x_i \st 0 \leq i < M} \cup \set{y_{j+1} - y_j \st 0 \leq j \leq N }.
    \end{equation*}
  \end{definition}

  \begin{definition}
    The \emph{Riemann integral} of $f$ over the rectangle $R$ is defined by
    \begin{equation*}
      \dint_R f(x, y) \, dx \, dy
      \defeq
      \lim_{\norm{P} \to 0}
	\mathcal R( f, P, \Xi ),
    \end{equation*}
    provided the limit exists and is independent of the choice of the points 
$\Xi$.
    A function is said to be \emph{Riemann integrable} over $R$ if the Riemann 
integral exists and is finite.
  \end{definition}

  \begin{remark}
    A few other popular notation conventions used to denote the integral are
    \begin{equation*}
      \iint_R f \, dA,
      \quad
      \iint_R f \, dx \, dy,
      \quad
      \iint_R f \, dx_1 \, dx_2,
      \quad\text{and}\quad
      \iint_R f.
    \end{equation*}
  \end{remark}

  \begin{remark}
    The double integral represents the volume of the region under the graph of 
$f$.
    Alternately, if $f(x, y)$ is the density of a planar body at point $(x, y)$, 
the double integral is the total mass.
  \end{remark}

  \begin{theorem}
    Any bounded continuous function is Riemann integrable on a bounded 
rectangle.
  \end{theorem}

  \begin{remark}
    Most bounded functions we will encounter will be Riemann integrable.
    Bounded functions with reasonable discontinuities (e.g. finitely many jumps) 
are usually Riemann integrable on bounded rectangle.
    An example of a ``badly discontinuous'' function that is not Riemann 
integrable is the function $f(x, y) = 1$ if $x, y \in \mathbb{Q}$ and $0$ 
otherwise.
    %s \emph{provided} they are bounded or don't grow ``too fast'' near their 
singularities.
  \end{remark}



  Now suppose $U \subseteq \bbR^2$ is an nice bounded%
  \footnote{%
    We will subsequently always assume $U$ is ``nice''. Namely, $U$ is open, 
connected and the boundary of $U$ is a piecewise differentiable curve.
    More precisely, we need to assume that the ``area'' occupied by the boundary 
of $U$ is $0$.
    While you might suspect this should be true for all open sets, it isn't!
    There exist open sets of \emph{finite area} whose boundary occupies an 
infinite area!}
  domain, and $f:U \to \bbR$ is a function.
  Find a bounded rectangle $R \supseteq U$, and as before let $P$ be a partition 
of $R$ into a grid of rectangles.
  Now we define the Riemann sum by only summing over all rectangles $R_{i,j}$ 
that are completely contained inside $U$.
  Explicitly, let
  \begin{equation*}
    \chi*{i,j}
      = \begin{cases}
	  1 & R_{i,j} \subseteq U\\
	  0 & \text{otherwise}.
	\end{cases}
  \end{equation*}
  and define
  \begin{equation*}
    \mathcal R(f, P, \Xi, U)
      %\defeq \dsum_{i=0}^{M-1} \dsum_{j = 0}^{N-1}
      %  \chi*{i,j} f(\xi_{i, j}) \area(R_{i,j})
      \defeq \dsum_{i=0}^{M-1} \dsum_{j = 0}^{N-1} \chi*{i,j}
	f(\xi_{i, j}) (x_{i+1} - x_i) (y_{j+1} - y_j).
  \end{equation*}
  \begin{definition}
    The \emph{Riemann integral} of $f$ over the \emph{domain $U$} is defined by
    \begin{equation*}
      \dint_U f(x, y) \, dx \, dy
      \defeq
      \lim_{\norm{P} \to 0}
	\mathcal R( f, P, \Xi, U ),
    \end{equation*}
    provided the limit exists and is independent of the choice of the points 
$\Xi$.
    A function is said to be \emph{Riemann integrable} over $R$ if the Riemann 
integral exists and is finite.
  \end{definition}

  \begin{theorem}
    Any bounded continuous function is Riemann integrable on a bounded region.
  \end{theorem}
  \begin{remark}
    As before, most reasonable \emph{bounded} functions we will encounter will 
be Riemann integrable.
  \end{remark}

  To deal with unbounded functions over unbounded domains, we use a limiting 
process.
  \begin{definition}
    Let $U \subseteq \bbR^2$ be a domain (which is not necessarily bounded) and 
$f:U \to \bbR$ be a (not necessarily bounded) function.
    We say $f$ is integrable if
    \begin{equation*}
      \lim_{R \to \infty} \dint_{U \cap B(0, R)} \chi*R \abs{f} \, dA
    \end{equation*}
    exists and is finite.
    Here $\chi*R(x) = 1$ if $\abs{f(x)} < R$ and $0$ otherwise.
  \end{definition}
  \begin{proposition}
    If $f$ is integrable on the domain $U$, then
    \begin{equation*}
      \lim_{R \to \infty} \dint_{U \cap B(0, R)} \chi*R f \, dA
    \end{equation*}
    exists and is finite.
  \end{proposition}
  \begin{remark}
    If $f$ is integrable, then the above limit is independent of how you expand 
your domain.
    Namely, you can take the limit of the integral over $U \cap [-R, R]^2$ 
instead, and you will still get the same answer.
  \end{remark}
  \begin{definition}
    If $f$ is integrable we define
    \begin{equation*}
      \dint_U f \, dx \, dy = \lim_{R \to \infty} \dint_{U \cap B(0, R)} \chi*R 
f \, dA
    \end{equation*}
  \end{definition}
  \section{Iterated integrals and Fubini's theorem}

  Let $U \subseteq \bbR^2$ be a domain.
  \begin{definition}
    For $x \in \bbR$, define
    \begin{equation*}
      S_x U = \set{y \st (x, y) \in U}
      \quad\text{and}\quad
      T_y U = \set{x \st (x, y) \in U}
    \end{equation*}
  \end{definition}

  \begin{example}
    If $U = [a, b] \times [c, d]$ then
    \begin{equation*}
      S_x U = 
      \begin{cases}
	[c,d] & x \in [a, b]\\
	\emptyset & x \not\in [a, b]
      \end{cases}
      \quad\text{and}\quad
      T_y U = \begin{cases}
	[a, b] & y \in [c, d]\\
	\emptyset & y \not\in [c,d].
      \end{cases}
    \end{equation*}
  \end{example}

  For domains we will consider, $S_x U$ and $T_y U$ will typically be an 
interval (or a finite union of intervals).
  \begin{definition}
    Given a function $f:U \to \bbR$, we define the two \emph{iterated} integrals 
by
    \begin{equation*}
      \dint_{x \in \bbR} \paren[\Big]{ \dint_{y \in S_xU} f(x, y) \, dy } \, dx
      \quad\text{and}\quad
      \dint_{y \in \bbR} \paren[\Big]{ \dint_{x \in T_y U} f(x, y) \, dx } \, 
dy,
    \end{equation*}
    with the convention that an integral over the empty set is $0$.
    (We included the parenthesis above for clarity; and will drop them as we 
become more familiar with iterated integrals.)
  \end{definition}

  Suppose $f(x, y)$ represents the density of a planar body at point $(x, y)$.
  For any $x \in \bbR$,
  \begin{equation*}
    \dint_{y \in S_xU} f(x, y) \, dy
  \end{equation*}
  represents the mass of the body contained in the vertical line through the 
point $(x, 0)$.
  It's only natural to expect that if we integrate this with respect to $y$, we 
will get the total mass, which is the double integral.
  By the same argument, we should get the same answer if we had sliced it 
horizontally first and then vertically.
  Consequently, we expect both iterated integrals to be equal to the double 
integral.
  This is true, under a finiteness assumption.

  \begin{theorem}[Fubini's theorem]
    Suppose $f:U \to \bbR$ is a function such that either
    \begin{equation}\label{eqnFubiniFiniteness}
      \dint_{x \in \bbR} \paren[\Big]{ \dint_{y \in S_xU} \abs{f(x, y)} \, dy } 
\, dx < \infty
      \quad\text{or}\quad
      \dint_{y \in \bbR} \paren[\Big]{ \dint_{x \in T_y U} \abs{f(x, y)} \, dx } 
\, dy < \infty,
    \end{equation}
    then $f$ is integrable over $U$ and
    \begin{equation*}
      \dint_U f \, dA
	= \dint_{x \in \bbR} \paren[\Big]{ \dint_{y \in S_xU} f(x, y) \, dy } \, 
dx
	= \dint_{y \in \bbR} \paren[\Big]{ \dint_{x \in T_y U} f(x, y) \, dx } 
\, dy.
    \end{equation*}
  \end{theorem}

  Without the assumption~\eqref{eqnFubiniFiniteness} the iterated integrals need 
not be equal, even though both may exist and be finite.
  \begin{example}
    Define
    \begin{equation*}
      f(x, y)
	= -\partial_x \partial_y \tan\inv\paren[\big]{\frac{y}{x}}
	= \frac{x^2 - y^2}{(x^2 + y^2)^2}.
    \end{equation*}
    Then
    \begin{equation*}
      \dint_{x = 0}^1 \dint_{y = 0}^1 f(x, y) \, dy \, dx = \frac{\pi}{4}
      \quad\text{and}\quad
      \dint_{y = 0}^1 \dint_{x = 0}^1 f(x, y) \, dx \, dy = -\frac{\pi}{4}
    \end{equation*}
  \end{example}
  \begin{example}
    Let $f(x, y) = (x - y) / (x + y)^3$ if $x, y > 0$ and $0$ otherwise, and $U 
= (0, 1)^2$.
    The iterated integrals of $f$ over $U$ both exist, but are not equal.
  \end{example}

  \begin{example}
    Define
    \begin{equation*}
      f(x, y) = \begin{cases}
	1 & y \in (x, x+1) \text{ and } x \geq 0\\
	-1 & y \in (x-1, x) \text{ and } x \geq 0\\
	0 & \text{otherwise}.
      \end{cases}
    \end{equation*}
    Then the iterated integrals of $f$ both exist and are not equal.
  \end{example}
  Fubini's Theorem allows us to convert the double integral into
iterated (single) integrals.
\begin{exa}
$$\begin{array}{lll}
\dint _{[0; 1]\times [2; 3]} xy \d{A} & = &
\dint _0 ^1 \left(\dint _2 ^3 xy \d{y}\right)\d{x}\\
& = & \dint _0 ^1 \left(\left[\dfrac{xy^2}{2}\right]_2 ^3\right)
\d{x}\\
& = & \dint _0 ^1 \left(\dfrac{9x}{2} - 2x \right) \d{x}\\
& = & \left[ \dfrac{5x^2}{4} \right] _0 ^1 \\
& = & \dfrac{5}{4}.
\end{array}$$
Notice that if we had integrated first with respect to $x$ we
would have obtained the same result:
$$\begin{array}{lll}
\dint _2 ^3 \left(\dint _0 ^1 xy \d{x}\right)\d{y}& = & \dint _2 ^3
\left(\left[\dfrac{x^2y}{2}\right]_0 ^1\right)
\d{y}\\
& = & \dint _2 ^3 \left(\dfrac{y}{2} \right) \d{x}\\
& = & \left[ \dfrac{y^2}{4} \right] _2 ^3 \\
& = & \dfrac{5}{4}.
\end{array}$$
Also, this integral is ``factorable into $x$ and $y$ pieces''
meaning that
$$\begin{array}{lll}
\dint _{[0; 1]\times [2; 3]} xy \d{A} & = & \left(\dint _0 ^1
x\d{x}\right)\left(\dint _2 ^3 y\d{y}\right)
\\
& = & \left(\dfrac{1}{2}\right)\left(\dfrac{5}{2}\right) \\
& = & \dfrac{5}{4}
\end{array}$$



\end{exa}
\begin{exa}\label{exa:double_int_for_change1}
We have $$\begin{array}{lll}\dint _3 ^4  \dint_0 ^1 (x + 2y)(2x + y) \
\d{x} \d{y} & = & \dint _3 ^4 \dint _0 ^1 (2x^2+5xy+2y^2) \
\d{x}\d{y} \\
& = &  \dint_3 ^4 \left(\frac{2}{3}+\frac{5}{2}y+2y^2\right)  \ \d{y} \\
& = & \frac{409}{12}.
\end{array}$$
\end{exa}



In the cases when the domain of integration is not a rectangle, we
decompose so that, one variable is kept constant.
\begin{exa}\label{exa:int_triag_region_1}
Find $\dint \limits_D xy \ \d{x}\d{y} $ in the triangle with
vertices $A:(-1,-1)$, $B:(2,-2)$, $C:(1,2)$. \end{exa}\begin{solu}
The lines passing through the given  points have equations $L_{AB}:
y = \dfrac{-x - 4}{3}$, $L_{BC}: y = -4x + 6$, $L_{CA}: y =
\dfrac{3x + 1}{2}$. Now, we draw the region {\em carefully}. If we
integrate first with respect to $y$, we must divide the region as in
figure \ref{fig:int_triag_region_1}, because there are two upper
lines which the upper value of $y$ might be. The lower point of the
dashed line is $(1, -5/3)$.  The integral is thus $$ \dint _{-1} ^1 x
\left(\dint _{(-x - 4)/3} ^{(3x + 1)/2} y \ \d{y}\right)\d{x}+ \dint
_{1} ^2 x \left(\dint _{(-x - 4)/3} ^{-4x + 6} y \ \d{y}\right)\d{x}=
-\dfrac{11}{8}.
$$
If we integrate first with respect to $x$, we must divide the region
as in figure \ref{fig:int_triag_region_1.1}, because there are two
left-most lines which the left value of $x$ might be. The right
point of the dashed line is $(7/4, -1)$. The integral is thus
$$ \dint _{-2} ^{-1} y \left(\dint _{-4 - 3y} ^{(6 - y)/4} x \ \d{x}\right)\d{y} 
+
\dint _{-1} ^2 y \left(\dint _{(2y - 1)/3} ^{(6 - y)/4} x \
\d{x}\right)\d{y} = -\dfrac{11}{8}.
$$


\end{solu}


\vspace*{2cm}
\begin{figure}[htpb]
\begin{minipage}{4.5cm}
$$ \psset{unit=1.5pc}\renewcommand{\pshlabel}[1]{{\tiny
#1}}
\renewcommand{\psvlabel}[1]{{\tiny #1}} \psaxes(0,0)(-3,-3)(3,3)
\psdots[dotscale=1, dotstyle=*](-1,-1)(2,-2)(1,2)(1,-1.66)
\psline(-1,-1)(2,-2)(1,2)(-1,-1)
\psline[linestyle=dashed](1,2)(1,-1.66)
$$ \vspace*{2cm}\hangcaption{Example \ref{exa:int_triag_region_1}. Integration 
order $\d{y}\d{x}$.} \label{fig:int_triag_region_1}
\end{minipage}\hfill\begin{minipage}{4.5cm}     $$ 
\psset{unit=1.5pc}\renewcommand{\pshlabel}[1]{{\tiny #1}}
\renewcommand{\psvlabel}[1]{{\tiny #1}} \psaxes(0,0)(-3,-3)(3,3)
\psdots[dotscale=1, dotstyle=*](-1,-1)(2,-2)(1,2)(1.75,-1)
\psline(-1,-1)(2,-2)(1,2)(-1,-1)
\psline[linestyle=dashed](-1,-1)(1.75,-1)
$$\vspace*{2cm}\hangcaption{Example \ref{exa:int_triag_region_1}. Integration 
order $\d{x}\d{y}$.} \label{fig:int_triag_region_1.1}
     \end{minipage}
     \hfill
\begin{minipage}{4.5cm}
$$\psset{unit=1pc}
\renewcommand{\pshlabel}[1]{{\tiny
#1}}
\renewcommand{\psvlabel}[1]{{\tiny
#1}}\rput(-4,0){\psaxes(0,0)(-.5,-.5)(10,7)\psline(6,3)(8,4)(9,6)(7,5)(6,3)
\psline[linestyle=dashed](6.5,4)(8,
4)\psline[linestyle=dashed](7,5)(8.5,5)\psdots[dotstyle=*,dotscale=.8](6,3)(8,
4)(7,5)(9,6)(6.5,4)(8.5,5)}
$$\vspace*{1cm} \footnotesize \hangcaption{Example \ref{exa:int_parall1}.} 
\label{fig:int_parall1}
\end{minipage}\end{figure}


\begin{exa}\label{exa:int_parall1}
Consider the region inside the parallelogram $P$ with vertices at
$A:(6, 3)$, $B:(8,4) $, $C:(9,6)$, $D:(7,5) $, as in figure
\ref{fig:int_parall1}. Find
$$\dint \limits_P  xy \ \ \d{x} \d{y}.    $$
\end{exa} \begin{solu} The lines joining the points have
equations
$$L_{AB}: \ y = \frac{x}{2},
$$ $$L_{BC}: \
y = 2x - 12,
$$ $$L_{CD}: \
y = \frac{x}{2} + \frac{3}{2},
$$ $$L_{DA}: \
y = 2x - 9.
$$
The integral is thus
$$\dint _3 ^4 \dint _{(y + 9)/2} ^{2y} xy\ \d{x}
\d{y}  + \dint _4 ^{5}  \dint _{(y + 9)/2} ^{(y + 12)/2} xy\  \d{x}
\d{y}    +
 \dint _5 ^{6}  \dint _{2y-3} ^{(y+12)/2} xy\  \d{x}  \d{y}   =   \frac{409}{4}. 
$$



\end{solu}

\begin{exa}
 Find $$\dint \limits_D \dfrac{y}{x^2 + 1} \d{x}\d{y}$$where $$D =  \{(x, 
y)\in\reals^2|x \geq 0, x^2 + y^2 \leq
 1\}.$$\end{exa}
\begin{solu} The integral is $0$. Observe that if $(x, y)\in D$ then $(x,
-y)\in D$. Also, $f(x, -y) = -f(x, y)$.
\end{solu}


  
\section*{\psframebox{Homework}}
\begin{multicols}{2}\columnseprule 1pt \columnsep 25pt\multicoltolerance=900
\begin{pro}
Evaluate the iterated integral $\dint _1 ^3 \dint _0 ^x
\dfrac{1}{x}\ \d{y}\d{x}$.
\begin{answer}
$2$
\end{answer}
\end{pro}
\begin{pro}
Let $S$ be the interior and boundary of the  triangle with vertices
$(0,0)$, $(2,1)$,  and $(2,0)$. Find $\dint \limits_{S} y\d{A}$.
\begin{answer}
$\dfrac{1}{3} $
\end{answer}
\end{pro}
\begin{pro}
Let $$S=\{(x,y)\in \reals^2: x\geq 0,\  y\geq 0,\  1 \leq
x^2+y^2\leq 4\}.$$ Find $\dint \limits_{S} x^2\d{A}$.
\begin{answer}
$\dfrac{15\pi}{16}$
\end{answer}
\end{pro}
\begin{pro}
 Find $$\dint \limits_D xy \d{x}\d{y}$$ where $$D =  \{(x, y)\in\reals^2|y \geq 
x^2, x \geq
 y^2\}.$$
\begin{answer} The integral equals
$$\begin{array}{lll}
\dint \limits_D xy \d{x}\d{y} & = & \dint _0 ^1 x\left(\dint _{x^2}
^{\sqrt{x}}  y \ \d{y} \right) \ \d{x}\\
& = & \dint _0 ^1 \dfrac{1}{2}x(x - x^4) \ \d{x}\\
& = & \dfrac{1}{12}.
\end{array}$$
\end{answer}
\end{pro}

\begin{pro}
Find $$\dint \limits_D (x + y)(\sin x)(\sin y) \d{A}$$ where $D =
[0; \pi]^2.$ \begin{answer} The integral equals
$$
\begin{array}{lll}\dint \limits_D x\sin x\sin y \d{x}\d{y}  +
\dint \limits_D y\sin x\sin y \d{x}\d{y} & = & 2\left(\dint_0 ^\pi
y\sin
y \ \d{y}\right)\left(\dint_0 ^\pi \sin x \ \d{x}\right) \\
& = & 4\pi.
\end{array}$$
\end{answer}
\end{pro}
\begin{pro}
Find $\dint _0 ^1 \dint _0 ^1 \min (x^2,y^2)\d{x}\d{y}$.
\begin{answer}
The integral is
$$\begin{array}{lll}\dint\limits _{x\leq y} x^2 \d{x}\d{y} + \dint\limits_{y\leq 
x} y^2 \d{x}\d{y}
& = & \dint _0 ^1 \dint _0 ^y x^2 \d{x}\d{y} + \dint _0 ^1 \dint _y ^1
y^2 \d{x}\d{y}\\
& = & \dint _0 ^1 \dfrac{y^3}{3}\d{y} + \dint _0 ^1
\left(y^2-y^3\right)\d{y}\\
& = & \dfrac{y^4}{12}\Big| _0 ^1 +
\left(\dfrac{y^3}{3}-\dfrac{y^4}{4}\right) \Big| _0 ^1 \\
& = & \dfrac{1}{12} + \dfrac{1}{3} -\dfrac{1}{4} \\
& = & \dfrac{1}{6}.
\end{array}$$
\end{answer}
\end{pro}
\begin{pro}
Find $\dint _D xy\d{x}\d{y}$ where
$$ D=\{(x,y)\in \reals^2: x>0,y>0, 9<x^2+y^2<16, 1<x^2-y^2<16\}. $$
\begin{answer}
$\dfrac{21}{8}$
\end{answer}
\end{pro}
\begin{pro}
\label{pro:polar-integral2}  Evaluate $\dint_{{\cal R}}x\d{A}$ where
${\cal R}$ is the (unoriented) circular segment in figure
\ref{fig:polar-integral2}, which is created by the intersection of
regions
$$\{(x,y)\in\reals^2: x^2+y^2 \leq 16 \}$$ and  $$\left\{(x,y)\in\reals^2:  y 
\geq -\dfrac{\sqrt{3}}{3}x+4\right\}. $$
\begin{answer}
Observe that $$ x^2+y^2=16, y = -\dfrac{\sqrt{3}}{3}x+4\implies
16-x^2=\left(-\dfrac{\sqrt{3}}{3}x+4\right)^2 \implies x = 0,
2\sqrt{3}.
$$The integral is
$$\begin{array}{lll} \dint _0 ^{2\sqrt{3}} \dint _{-\frac{\sqrt{3}}{3}x+4} 
^{\sqrt{16-x^2}}x\d{y}\d{x} & = &
\dint _0 ^{2\sqrt{3}}
x\left(\sqrt{16-x^2}+\dfrac{\sqrt{3}}{3}x-4\right) \d{x}\\
& = & -\dfrac{1}{3}(16-x^2)^{3/2}+\dfrac{\sqrt{3}}{9}x^3-2x^2\Big|
_0
^{2\sqrt{3}}\\
& = & \dfrac{8}{3}.
\end{array}$$

\end{answer}
\end{pro}

\end{multicols}
\section{Change of Variables}
We now perform a multidimensional analogue of  the change of
variables theorem in one variable.
\begin{thm}
Let $(D, \Delta) \in (\reals^n)^2$ be open, bounded sets in
$\reals^n$ with volume and let $g:\Delta \rightarrow D$ be a
continuously differentiable bijective mapping such that $\det g'(u)
\neq 0$, and both $|\det g'(u)|, \dfrac{1}{|\det g'(u)|}$ are
bounded on $\Delta$. For $f:D \rightarrow \reals$ bounded and
integrable, $f\circ g|\det g'(u)|$ is integrable on $\Delta$ and
$$\dint \cdots \dint _D f = \dint \cdots \dint _\Delta (f\circ g)|\det
 g'(u)|,$$ that is
$$\dint \cdots \dint _D f(x_1, x_2, \ldots, x_n) \d{x}_1 \wedge
\d{x}_2 \wedge \ldots \wedge \d{x}_n  $$
$$\qquad \qquad \qquad = \dint \cdots \dint _\Delta f(g(u_1, u_2, \ldots, u_n))
|\det g'(u)|\d{u}_1 \wedge \d{u}_2 \wedge \ldots \wedge \d{u}_n .$$
\end{thm}
One normally chooses changes of variables that map into
rectangular regions, or that simplify the integrand. Let us start
with a rather trivial example.


\vspace*{4cm}
\begin{figure}[htpb]
\begin{minipage}{7cm}
$$\psset{unit=1pc}
\renewcommand{\pshlabel}[1]{{\tiny
#1}}
\renewcommand{\psvlabel}[1]{{\tiny #1}}
\rput(-4,0){\psaxes(0,0)(-.5,-.5)(10,7)\pscustom[fillcolor=yellow,
fillstyle=solid]{\psline(0,3)(0,4)(1,4)(1,3)(0,3)}\psdots[dotstyle=*,dotscale=.8
](0,3)(0,4)(1,4)(1,3)}$$
\vspace*{1cm} \footnotesize \hangcaption{Example
\ref{exa:change_of_var_1}. $xy$-plane.} \label{fig:change_of_var_1}
\end{minipage}
\hfill
\begin{minipage}{7cm}
$$\psset{unit=1pc}
\renewcommand{\pshlabel}[1]{{\tiny
#1}}
\renewcommand{\psvlabel}[1]{{\tiny
#1}}\rput(-4,0){\psaxes(0,0)(-.5,-.5)(10,7)\pscustom[fillcolor=yellow,
fillstyle=solid]{\psline(6,3)(8,4)(9,6)(7,5)(6,3)}\psdots[dotstyle=*,dotscale=.8
](6,3)(8,4)(7,5)(9,6)}
$$\vspace*{1cm} \footnotesize \hangcaption{Example \ref{exa:change_of_var_1}. 
$uv$-plane.} \label{fig:change_of_var_1}\end{minipage}
\end{figure}


\begin{exa}
Evaluate the integral $$\dint _3 ^4 \dint _0 ^1  (x + 2y)(2x + y)\\
\d{x} \d{y}.
$$
\label{exa:change_of_var_1}\end{exa} \begin{solu} Observe that we
have already computed this integral in example
\ref{exa:double_int_for_change1}. Put
$$u = x + 2y \implies \d{u} = \d{x} + 2\d{y},     $$
$$v = 2x + y \implies \d{v} = 2\d{x} + \d{y},     $$

giving $$\d{u}\wedge\d{v} = -3\d{x}\wedge\d{y}.
$$
Now, $$(u,v) = \begin{bmatrix} 1 & 2 \cr 2 & 1 \cr
\end{bmatrix}\colvec{x
\\ y}
$$is a linear transformation, and hence it maps quadrilaterals into
quadrilaterals. The corners of the rectangle in the area of
integration in the $xy$-plane are $(0,3)$,  $(1,3)$, $(1,4)$, and
$(0, 4)$, (traversed counter-clockwise) and they map into $(6, 3)$,
$(7,5) $, $(9,6)$, and $(8,4) $, respectively, in the $uv$-plane
(see figure \ref{fig:change_of_var_1}). The form $\d{x}\wedge\d{y}$
has opposite orientation to $\d{u}\wedge\d{v}$ so we use $$
\d{v}\wedge\d{u} = 3\d{x}\wedge\d{y}    $$instead. The integral
sought is
$$  \frac{1}{3}\dint\limits_P uv \ \d{v}\d{u}     =
\frac{409}{12},
$$from example \ref{exa:int_parall1}.
\end{solu}


\begin{exa}
The integral
$$\dint _{[0;1]^2} (x^4 - y^4) \d{A}
= \dint _0 ^1  \left( \dfrac{1}{5} - y^4\right)\d{y}
 = 0.$$ Evaluate it
using the change of variables $u = x^2 - y^2, v = 2xy.$
\end{exa}
\begin{solu} First we find
$$\d{u}= 2x\d{x}- 2y\d{y},$$
$${\rm d}v = 2y\d{x}+ 2x\d{y},$$and so
$$\d{u}\wedge\d{v} = (4x^2 + 4y^2)\d{x}\wedge \d{y}.$$


We now determine the region $\Delta$ into which the square $D =
[0; 1 ]^2$ is mapped. We use the fact that boundaries will be
mapped into boundaries. Put $$AB = \{(x, 0): 0 \leq x \leq 1\},$$
$$BC = \{(1, y): 0 \leq y \leq 1\},$$ $$CD = \{(1 - x, 1): 0 \leq x
\leq 1\},$$ $$DA = \{(0, 1 - y): 0 \leq y \leq 1\}.$$




On $AB$ we have $u = x, v = 0.$ Since $0 \leq x \leq 1$, $AB$ is
thus mapped into the line segment $0 \leq u \leq 1, v = 0$.




On $BC$ we have $u = 1 - y^2, v = 2y$. Thus $u = 1 -
\dfrac{v^2}{4}$. Hence $BC$ is mapped to the portion of the
parabola $u = 1 - \dfrac{v^2}{4}, 0 \leq v \leq 2.$




On $CD$ we have $u = (1 - x)^2 - 1, v = 2(1 - x).$ This means that
$u = \dfrac{v^2}{4} - 1, 0 \leq v \leq 2.$




Finally, on $DA,$ we have $u = -(1 - y)^2, v = 0.$ Since $0 \leq y
\leq 1$, $DA$ is mapped into the line segment $-1 \leq u \leq 0, v
= 0.$ The region $\Delta$ is thus the area in the $uv$ plane
enclosed by the parabolas $u \leq \dfrac{v^2}{4} - 1, u \leq 1 -
\dfrac{v^2}{4}$ with $ -1 \leq u \leq 1, 0 \leq v \leq 2.$



We deduce that
$$\begin{array}{lll}\dint _{[0;1]^2} (x^4 - y^4) \d{A}& = &  \dint _{\Delta} 
(x^4 - y^4) \dfrac{1}{4(x^2 + y^2)}
\d{u}\d{v} \\
& = &  \dfrac{1}{4}\dint _{\Delta} (x^2 - y^2)
\d{u}\d{v} \\
& = &  \dfrac{1}{4}\dint _{\Delta} u
\d{u}\d{v} \\
& = & \dfrac{1}{4}\dint _0 ^2 \left( \dint _{v^2/4 - 1} ^{1 - v^2/4}
u
\d{u}\right)\d{v} \\
& = & 0,
\end{array}$$as before.
\end{solu}
\begin{exa}
Find $$\dint \limits_D  e^{(x^3 + y^3)/xy} \ \ \ \d{A}$$ where
$$D =  \{(x, y)\in\reals^2|y^2 - 2px \leq 0, x^2 - 2py \leq 0, p\in
]0;+\infty[ \ {\rm fixed}\},$$  using the change of variables $x =
u^2v, \ y = uv^2$.
\end{exa}
\begin{solu} We have $$\d{x} = 2uv\d{u} + u^2\d{v},    $$
$$\d{y} = v^2\d{u} + 2uv\d{v},    $$
$$\d{x} \wedge \d{y} = 3u^2v^2\d{u}\wedge\d{v}.    $$
The region transforms into
$$\Delta = \{(u, v)\in\reals^2| 0 \leq u \leq (2p)^{1/3}, \ 0 \leq v \leq
(2p)^{1/3}\}.$$The integral becomes
$$\begin{array}{lll}
\dint \limits_D  f(x, y) \d{x}\d{y} & = & \dint \limits_\Delta \exp
\left(\dfrac{u^6v^3 +
u^3v^6}{u^3v^3}\right) (3u^2v^2) \ \d{u}{\rm d}v \\
& = & 3\dint \limits_\Delta e^{u^3}e^{v^3}u^2v^2 \ \d{u}{\rm d}v \\
& = & \dfrac{1}{3}\left(\dint _0 ^{(2p)^{1/3}} 3u^2e^{u^3}\ \d{u}
\right)^2 \\
& = & \dfrac{1}{3}(e^{2p} - 1)^2.
\end{array}$$
As an exercise, you may try the (more natural) substitution $x^3 =
u^2v, y^3 = v^2u$ and verify that the same result is obtained.
\end{solu}


\vspace*{2cm}
\begin{figure}[htpb]
\begin{minipage}{7cm}
$$
\psaxes[linewidth=2pt,arrows={->}](0,0)(0,0)(1.5,1.5)
\pspolygon*[showpoints=true,linecolor=green](0,0)(1,0)(1,1)(0,1)
$$
\vspace*{1cm} \footnotesize \hangcaption{Example
\ref{exa:change_of_var_4}. $xy$-plane.} \label{fig:change_of_var_4}
\end{minipage}
\hfill
\begin{minipage}{7cm}
$$\psaxes[linewidth=2pt,arrows={->},labels=none](0,0)(0,0)(1.9,1.9)
\pspolygon*[showpoints=true,linecolor=green](0,0)(1.5708,0)(0,1.5708)
\uput[d](1.5708,0){\frac{\pi}{2}} \uput[l](0,1.5708){\frac{\pi}{2}}
$$\vspace*{1cm} \footnotesize \hangcaption{Example \ref{exa:change_of_var_4}. 
$uv$-plane.} \label{fig:change_of_var_5}\end{minipage}
\end{figure}


\begin{exa}In this problem we will follow an argument of Calabi, Beukers, and 
Kock to prove that $\dsum _{n=1}
^{+\infty}\dfrac{1}{n^2}=\dfrac{\pi ^2}{6}$.
\begin{enumerate}
\item
Prove that if $S = \dsum _{n=1} ^{+\infty}\dfrac{1}{n^2}$, then
$\dfrac{3}{4}S = \dsum _{n=1} ^{+\infty}\dfrac{1}{(2n-1)^2}$.
\item Prove that $ \dsum _{n=1} ^{+\infty}\dfrac{1}{(2n-1)^2} = \dint _0 ^1 
\dint _0 ^1
\dfrac{\d{x}\d{y}}{1-x^2y^2}$.
\item Use the change of variables $x=\dfrac{\sin u}{\cos v}$, $y=\dfrac{\sin
v}{\cos u}$ in order to evaluate $\dint _0 ^1 \dint _0 ^1
\dfrac{\d{x}\d{y}}{1-x^2y^2}$.
\end{enumerate}
\label{exa:change_of_var_4}
\end{exa}
\begin{solu}
\noindent
\begin{enumerate}
\item Observe that the sum of the even terms is
$$ \dsum _{n=1} ^{+\infty}\dfrac{1}{(2n)^2} = \dfrac{1}{4}\dsum _{n=1} 
^{+\infty}\dfrac{1}{n^2} = \dfrac{1}{4}S,
$$a quarter of the sum, hence the sum of the odd terms must be three quarters of 
the sum,
$\dfrac{3}{4}S$.
\item Observe that
$$ \dfrac{1}{2n-1} = \dint _0 ^1 x^{2n-2}\d{x} \implies 
\left(\dfrac{1}{2n-1}\right)^2 = \left(\dint _0 ^1 
x^{2n-2}\d{x}\right)\left(\dint _0 ^1 y^{2n-2}\d{y}\right)= \dint _0 ^1 \dint _0 
^1 (xy)^{2n-2}\d{x}\d{y}. $$
Thus
$$\dsum _{n=1} ^{+\infty}\dfrac{1}{(2n-1)^2} =  \dsum _{n=1} ^{+\infty} \dint _0 
^1 \dint _0 ^1 (xy)^{2n-2}\d{x}\d{y} = \dint _0 ^1 \dint _0 ^1  \dsum _{n=1} 
^{+\infty}(xy)^{2n-2}\d{x}\d{y}
=\dint _0 ^1 \dint _0 ^1 \dfrac{\d{x}\d{y}}{1-x^2y^2},$$ as
claimed.\footnote{This exchange of integral and sum needs
justification. We will accept it for our purposes.}


\item If $x=\dfrac{\sin u}{\cos v}$, $y=\dfrac{\sin
v}{\cos u}$, then
$$\d{x} = (\cos u)(\sec v)\d{u} + (\sin u)(\sec v)(\tan v)\d{v}, \quad \d{y} = 
(\sec u)(\tan u)(\sin v)\d{u}  + (\sec u)(\cos v)\d{v}, $$
from where
$$ \d{x}\wedge\d{y}= \d{u}\wedge\d{v} - (\tan^2u)(\tan^2v)\d{u}\wedge\d{v} = 
\left(1-(\tan^2u)(\tan^2v)\right)\d{u}\wedge\d{v}. $$
Also,
$$ 1-x^2y^2 = 1-\dfrac{\sin ^2v}{\cos^2v}\cdot \dfrac{\sin^2v}{\cos^2u} 
=1-(\tan^2u)(\tan^2v). $$
This gives
$$ \dfrac{\d{x}\d{y}}{1-x^2y^2}=\d{u}\d{v}. $$
We now have to determine the region that the transformation
$x=\dfrac{\sin u}{\cos v}$, $y=\dfrac{\sin v}{\cos u}$ forms in the
$uv$-plane. Observe that
$$ u=\arctan x\sqrt{\dfrac{1-y^2}{1-x^2}}, \qquad v =\arctan y 
\sqrt{\dfrac{1-x^2}{1-y^2}}.  $$
This means that the square in the $xy$-plane in figure
\ref{fig:change_of_var_4} is transformed into the triangle in the
$uv$-plane in figure \ref{fig:change_of_var_5}.



We deduce,
 $$\dint _0 ^1 \dint _0 ^1
\dfrac{\d{x}\d{y}}{1-x^2y^2} = \dint _0 ^{\pi /2}\dint _0 ^{\pi/2
-v}\d{u}\d{v} =\dint _0 ^{\pi /2}\left(\pi/2 -v\right)\d{v} =
\left(\dfrac{\pi}{2}v -\dfrac{v^2}{2}\right)\Big| _0 ^{\pi/2} =
\dfrac{\pi ^2}{4} - \dfrac{\pi ^2}{8} =\dfrac{\pi ^2}{8}.
$$
Finally,
$$ \dfrac{3}{4}S=\dfrac{\pi ^2}{8} \implies S =\dfrac{\pi ^2}{6}.$$
\end{enumerate}
\end{solu}





  
\section*{\psframebox{Homework}}
\begin{pro}
Let $D' = \{(u, v)\in\reals^2: u  \leq 1, -u \leq v \leq u\}$.
Consider $$\fun{\Phi}{(u, v)}{\left(\frac{u + v}{2}, \frac{u -
v}{2}\right)}{\reals^2}{\reals^2}.$$
\begin{dingautolist}{202}
\item Find the image of $\Phi$ on $D'$, that is, find $D = \Phi
(D')$. \item Find $$\dint_D (x + y)^2e^{x^2 - y^2}\d{A}.$$
\end{dingautolist} \begin{answer}
\begin{dingautolist}{202}
\item  Put $x = \frac{u + v}{2}$ and $y = \frac{u - v}{2}$ . Then
$x + y = u$ and  $x - y$. Observe that $D'$ is the triangle in the
$uv$ plane bounded by the lines $u = 0, u = 1, v = u, v = -u.$ Its
image under $\Phi$ is the triangle bounded by the equations $x = 0,
y = 0, x + y = 1.$ Clearly also
$$\d{x}\wedge \d{y} = \frac{1}{2}\d{u}\wedge\d{v}.$$

\item  From the above
$$\begin{array}{lll}\dint_D (x + y)^2e^{x^2 - y^2}\d{A}
& = & \frac{1}{2}\dint_{D'} u^2e^{uv}\d{u}{\rm d}v \\
& = & \frac{1}{2}\dint _0 ^1 \dint _{-u} ^u u^2 e^{uv} \d{u}\d{v} \\
& = & \frac{1}{2}\dint _0 ^1 u(e^{u^2} - e^{-u^2})\d{u}\\
& = & \frac{1}{4}(e + e^{-1} - 2).
\end{array}$$
\end{dingautolist}
\end{answer}
\end{pro}
\begin{pro}
Using the change of variables $x=u^2-v^2$, $y=2uv$, $u \geq 0$,
$v\geq 0$, evaluate $\dint _R \sqrt{x^2+y^2}\d{A}$, where
$$R=\{(x,y)\in\reals^2: -1\leq x \leq 1, 0 \leq y \leq 2\sqrt{1-|x|}\}$$
\end{pro}
\begin{pro}Using  the change of variables $u = x-y$ and $v = x + y$,
evaluate $\dint _R \dfrac{x-y}{x+y}\d{A}$, where
 $R$ is the square with vertices at  $(0, 2)$, $(1, 1)$, $ (2, 2)$,
$(1, 3)$.
\end{pro}
\begin{pro}
Find $\dint \limits_D f(x, y) \d{A}$ where $$D = \{(x,
y)\in\reals^2|a \leq xy \leq b, y \geq x \geq 0, y^2 - x^2 \leq 1,
(a, b) \in\reals^2, 0 < a < b\}$$ and $f(x, y) = y^4 - x^4$ by using
the change of variables $u = xy, v = y^2 - x^2$.
\begin{answer} Here we argue that
$$\d{u}= y\d{x}+ x\d{y},$$
$${\rm d}v = -2x\d{x}+ 2y\d{y}.$$
Taking the wedge product of differential forms,
$$\d{u}\wedge\d{v} = 2(y^2  + x^2) \d{x}\wedge \d{y}. $$
Hence
$$\begin{array}{lll}f(x, y) \d{x}\wedge \d{y} & = & (y^4 - x^4) \dfrac{1}{2(y^2 
+ x^2)}\d{u}\wedge{\rm
d}v\\
& = &  \dfrac{1}{2}(y^2 - x^2)\d{u}\wedge{\rm d}v \\
& = &  \dfrac{v}{2}\d{u}\wedge{\rm d}v\end{array}$$ The region
transforms into
$$\Delta = [a; b]\times[0;1].$$The integral becomes
$$\begin{array}{lll}
\dint \limits_D f(x, y) \d{x}\wedge\d{y} & = & \dint \limits_\Delta
v \ \d{u}\wedge{\rm d}v \\
& = & \dfrac{1}{2}\left(\dint _a ^b \ \d{u}\right)\left(\dint _0 ^1 v\ 
\d{v}\right) \\
& = & \dfrac{b - a}{4}.
\end{array}$$
\end{answer}
\end{pro}

\begin{pro}Use the following steps (due to Tom Apostol) in order to
prove that
$$\dsum _{n = 1} ^\infty \frac{1}{n^2}  = \frac{\pi ^2}{6}.$$
\begin{dingautolist}{202}
\item  Use the series expansion
$$ \frac{1}{1 - t} = 1 + t + t^2 + t^3 + \cdots  \ \ \ \ |t| <
1,$$in order to prove (formally) that
$$\dint _0 ^1 \dint _0 ^1 \frac{\d{x}\d{y}}{1 - xy} = \dsum _{n = 1} ^\infty 
\frac{1}{n^2}.$$

\item  Use the change of variables $u = x + y, v = x - y$ to shew
that
$$\dint _0 ^1 \dint _0 ^1 \frac{\d{x}\d{y}}{1 - xy}  =
2\dint _0 ^1 \left(\dint _{-u} ^u\frac{{\rm d}v}{4 - u^2 + v^2}\right)
\d{u}+ 2\dint _1 ^2 \left(\dint _{u - 2} ^{2 - u }\frac{{\rm d}v}{4 -
u^2 + v^2}\right) \d{u}.$$

\item  Shew that the above integral reduces to
$$2\dint _0 ^1 \frac{2}{\sqrt{4 - u^2}} \arctan \frac{u}{\sqrt{4 - u^2}} \ \d{u}
+ 2\dint _1 ^2 \frac{2}{\sqrt{4 - u^2}} \arctan \frac{ 2 - u}{\sqrt{4
- u^2}} \ \d{u}.$$ \item Finally, prove that the above integral is
$\frac{\pi^2}{6}$ by using the substitution $\theta = \arcsin
\frac{u}{2}$.
\end{dingautolist}

\begin{answer}\begin{dingautolist}{202} \item  Formally,
$$\begin{array}{lll}
\dint _0 ^1 \dint _0 ^1 \frac{\d{x}\d{y}}{1 - xy} & = & \dint _0 ^1
\dint _0 ^1 (1 + xy + x^2y^2 + x^3y^3 + \cdots )\d{x}\d{y} \\
& = & \dint _0 ^1 \left(y + \frac{xy^2}{2} + \frac{x^2y^3}{3}   +
\frac{x^3y^4}{4} + \cdots \right)_0 ^1 \ \d{x}\\ & = & \dint _0 ^1 (1
+ \frac{x}{2} + \frac{x^2}{3}  + \frac{x^3}{4} + \cdots )\d{x}\\
& = & 1 + \frac{1}{2^2} + \frac{1}{3^2} + \frac{1}{4^2}  + \cdots
\end{array}$$
\item This change of variables transforms the square $[0; 1]\times
[0;1]$ in the $xy$ plane into the square with vertices at $(0, 0)$,
$(1, 1)$, $(2, 0)$, and $(1, -1)$ in the $uv$ plane. We will split
this region of integration into two disjoint triangles: $T_1$ with
vertices at  $(0, 0)$, $(1, 1)$, $(1, -1)$, and $T_2$ with vertices
at $(1, -1)$, $(1, 1)$, $(2, 0)$. Observe that
$$\d{x}\wedge \d{y} = \frac{1}{2}\d{u}\wedge\d{v},$$
and that $u + v = 2x, u - v = 2y$ and so $4xy = u^2 - v^2$. The
integral becomes
$$\begin{array}{lll}
\dint _0 ^1 \dint _0 ^1 \frac{\d{x}\d{y}}{1 - xy}  & = &
\frac{1}{2}\dint\limits_{T_1\cup T_2} \frac{\d{u}\wedge {\rm d
}v}{1 - \frac{u^2 - v^2}{4}} \\
& = & 2\dint _0 ^1 \left(\dint _{-u} ^u\frac{{\rm d}v}{4 - u^2 +
v^2}\right) \d{u}+ 2\dint _1 ^2 \left(\dint _{u - 2} ^{2 - u
}\frac{{\rm d}v}{4 - u^2 + v^2}\right) \d{u},
\end{array}$$as desired.

\item  This follows by using the identity
$$\dint_0 ^t \frac{\d{\omega}}{1 + \Omega^2} = \arctan t.$$
\item  This is straightforward but tedious!
\end{dingautolist}

\end{answer}
\end{pro}

\section{Change to Polar Coordinates}
One of the most common changes of variable is the passage to polar
coordinates  where
$$x = \rho\cos\theta \implies \d{x} = \cos\theta\d{\rho} - 
\rho\sin\theta\d{\theta}, $$
$$y = \rho\sin\theta \implies \d{y} = \sin\theta\d{\rho} + 
\rho\cos\theta\d{\theta}, $$
whence
$$\d{x} \wedge \d{y} = (\rho\cos^2\theta + 
\rho\sin^2\theta)\d{\rho}\wedge\d{\theta} = \rho\d{\rho}\wedge\d{\theta}. $$
\begin{exa}\label{exa:pol-coor1}
Find $$\dint \limits_D  xy\sqrt{x^2 + y^2}\d{A} $$ where
$$D = \{(x, y)\in\reals^2|x \geq 0, y \geq 0, y \leq x, x^2 + y^2
\leq 1\}.$$
\end{exa}
\begin{solu} We use polar coordinates. The region $D$ transforms into the
region $$\Delta = [0; 1]\times \left[0; \dfrac{\pi}{4}
\right].$$Therefore the integral becomes
$$\begin{array}{lll}
\dint \limits_\Delta \rho ^4 \cos\theta\sin\theta \ \d{\rho}
\d{\theta} & = & \left(\dint _0 ^{\pi/4} \cos\theta\sin\theta \ 
\d{\theta}\right)\left(\dint _0 ^1 \rho ^4 \ \d{\rho}\right) \\
& = & \dfrac{1}{20}.
\end{array}$$
\end{solu}
\vspace{2cm}
\begin{figure}[!hptb]
\begin{minipage}{4cm}
\centering \psset{unit=1pc} \pswedge*[linecolor=red](0,0){2}{0}{45}
\psaxes[labels=none,linewidth=2pt](0,0)(-4,-4)(4,4)
\vspace{2cm}\footnotesize\hangcaption{Example
\ref{exa:pol-coor1}.}\label{fig:pol-coor1}
\end{minipage}
\hfill
\begin{minipage}{4cm}
\centering \psset{unit=1pc}
\pswedge*[linecolor=red](0,0){2}{0}{90}\pswedge*[linecolor=white](0,1){1}{-90}{
90}
\psaxes[labels=none,linewidth=2pt](0,0)(-4,-4)(4,4)
\vspace{2cm}\footnotesize\hangcaption{Example
\ref{exa:pol-coor2}.}\label{fig:pol-coor2}
\end{minipage}
\hfill
\begin{minipage}{4cm}
\centering \psset{unit=1pc}
\psaxes[labels=none,linewidth=2pt](0,0)(-4,-4)(4,4)
\parametricplot[algebraic,linewidth=2pt,linecolor=red]{0}{6.29}{
-sqrt(3)*sin(t)/3+cos(t)|2*sqrt(3)*sin(t)/3}
\vspace{2cm}\footnotesize\hangcaption{Example
\ref{exa:pol-coor3}.}\label{fig:pol-coor3}
\end{minipage}
\hfill
\begin{minipage}{4cm}
\centering \psset{unit=1pc}
\psaxes[labels=none,linewidth=2pt](0,0)(-4,-4)(4,4)
\pstGeonode[PointName=none,PointSymbol=none](0,0){O}(2,0){A}(-1,1){C}(1,1){D}
\pstInterLC[PointName=none]{C}{D}{O}{A}{L}{L'}
\pscustom[fillstyle=solid,fillcolor=red]{\psline(L)(L')\pstArcOAB{O}{L'}{L}}
\vspace{2cm}\footnotesize\hangcaption{Example
\ref{exa:pol-coor4}.}\label{fig:pol-coor4}
\end{minipage}
\end{figure}

\begin{exa}\label{exa:pol-coor2}
Evaluate $\dint _R xd{A}$, where $R$ is the region bounded by the
circles $x^2+y^2=4$ and $x^2+y^2=2y$.
\end{exa}
\begin{solu}
Observe that this is problem \ref{pro:bet-circles1}. Since
$x^2+y^2=r^2$, the radius sweeps from  $r^2=2r\sin \theta$ to
$r^2=4$, that is, from $2\sin \theta$ to  $2$. The angle clearly
sweeps from $0$ to $\dfrac{\pi}{2}$. Thus the integral becomes
$$\begin{array}{lll}\dint _R xd{A} & = & \dint _0 ^{\pi/2}\dint _2 
^{2\sin\theta} r^2\cos\theta \d{r}\d{\theta} \\
& = & \dfrac{1}{3}\dint _0 ^{\pi/2}(8\cos\theta 
-8\cos\theta\sin^3\theta)\d{\theta} \\
& =& 2.
 \end{array} $$
\end{solu}
\begin{exa}\label{exa:pol-coor3}
Find $\dint _D e^{-x^2-xy-y^2} \ \d{A}$, where
$$D = \{(x,y)\in \reals^2: x^2 + xy + y^2 \leq 1\}.
$$ \end{exa} \begin{solu}
 Completing squares $$ x^2 + xy + y^2 =  \left(x + \frac{y}{2}\right)^2 + 
\left(\frac{\sqrt{3}y}{2}\right)^2.  $$ Put $U= x + \frac{y}{2}$, $V= 
\frac{\sqrt{3}y}{2}$.
The integral becomes $$\begin{array}{lll}\dint _{\{x^2 + xy + y^2
\leq 1\}} e^{-x^2-xy-y^2}\d{x} \d{y} & = & \frac{2}{\sqrt{3}}\dint
_{\{U^2 +V^2 \leq 1\}} e^{-(U^2 + V^2)}\d{U} \d{V}. \end{array}
$$Passing to polar coordinates, the above equals $$ \frac{2}{\sqrt{3}}\dint _0 
^{2\pi}\dint _{0} ^1  \rho e^{-\rho^2}\d{\rho}\d{\theta} =  
\dfrac{2\pi}{\sqrt{3}} (1 - e^{-1}).   $$
\end{solu}
\begin{exa}\label{exa:pol-coor4}
Evaluate $\dint _{\cal R} \dfrac{1}{(x^2+y^2)^{3/2}}\d{A}$ over the
region $\left\{(x,y)\in\reals^2:x^2+y^2\leq 4, y\geq 1\right\}$
(figure \ref{fig:pol-coor4}).
\end{exa}
\begin{solu}
The radius sweeps from $r=\dfrac{1}{\sin\theta} $ to $r=2$. The
desired integral is $$\begin{array}{lll}\dint _{\cal R}
\dfrac{1}{(x^2+y^2)^{3/2}}\d{A} & = & \dint _{\pi/6} ^{5\pi /6} \dint
_{\csc \theta} ^{2}\dfrac{1}{r^2} \d{r}\d{\theta}\\
& = & \dint _{\pi/6} ^{5\pi /6} \left(\sin\theta-\dfrac{1}{2}\right)\d{\theta}\\
& = &\sqrt{3} -\dfrac{\pi}{3}.
\end{array}$$
\end{solu}
\begin{exa}
Evaluate $\dint _R (x^3+y^3)\d{A}$ where $R$ is the region bounded by
the ellipse $\dfrac{x^2}{a^2}+\dfrac{y^2}{b^2}=1$ and the first
quadrant, $a>0$ and $b>0$.
\begin{solu}
Put $x=ar\cos \theta$, $y=br\sin \theta$. Then
$$x = ar\cos\theta \implies \d{x} = a\cos\theta\d{r} - ar\sin\theta\d{\theta}, 
$$
$$y = br\sin\theta \implies \d{y} = b\sin\theta\d{r} + br\cos\theta\d{\theta}, 
$$
whence
$$\d{x} \wedge \d{y} = (abr\cos^2\theta + abr\sin^2\theta)\d{r}\wedge\d{\theta} 
= abr\d{r}\wedge\d{\theta}. $$
Observe that on the ellipse
$$\dfrac{x^2}{a^2}+\dfrac{y^2}{b^2}=1 \implies 
\dfrac{a^2r^2\cos^2\theta}{a^2}+\dfrac{b^2r^2\sin^2\theta}{b^2}=1\implies r=1.  
$$
Thus the required integral is
$$\begin{array}{lll}\dint _R (x^3+y^3)\d{A} & = & \dint _0 ^{\pi/2} \dint _0 ^1 
abr^4(\cos ^3\theta + \sin ^3\theta)\d{r}\d{\theta} \\
& = & ab\left(\dint _0 ^1 r^4\d{r}\right)\left( \dint _0 ^{\pi/2}
(a^3\cos ^3\theta + b^3\sin ^3\theta)\d{\theta}\right)\\
& = & ab\left(\dfrac{1}{5}\right)\left(\dfrac{2a^3+2b^3}{3}\right)\\
& = & \dfrac{2ab(a^3+b^3)}{15}.
 \end{array}$$

\end{solu}
\end{exa}
\section*{\psframebox{Homework}}

\begin{pro}\label{pro:inside-circles}
Evaluate $\dint _{\cal R} xy\d{A}$ where ${\cal R}$ is the region
$${\cal R} =\left\{(x,y)\in\reals^2: x^2+y^2\leq 16, x\geq 1, y \geq
1\right\},$$as in the figure \ref{fig:inside-circles1}. Set up the
integral in both Cartesian and polar coordinates.
\begin{answer}
The integral in Cartesian coordinates is
$$\begin{array}{lll}  \dint _{1} ^{\sqrt{15}} \dint _{1} 
^{\sqrt{16-y^2}}xy\d{x}\d{y}
& = & \dfrac{1}{2}\dint \dint _{1} ^{\sqrt{15}} 15y-y^3 \d{y}\\
& = & \dfrac{49}{2}.
\end{array}$$



The integral in polar coordinates is $$\begin{array}{lll}\dint
_{\arcsin \frac{1}{4}} ^{\frac{\pi}{4}} \dint _{1/\sin \theta}
^4r^3\sin\theta\cos\theta  \d{r}\d{\theta}
  + \dint  _{\frac{\pi}{4}} ^{\arccos \frac{1}{4}}\dint _{1/\cos \theta} 
^4r^3\sin\theta\cos\theta  \d{r}\d{\theta} & =
& \dfrac{1}{4}\dint _{\arcsin \frac{1}{4}} ^{\frac{\pi}{4}}
\left(4^4-\dfrac{1}{\sin^4\theta}\right)\sin\theta\cos\theta
\d{\theta}\\
& & \quad
  + \dfrac{1}{4}\dint  _{\frac{\pi}{4}} ^{\arccos 
\frac{1}{4}}\left(4^4-\dfrac{1}{\cos^4\theta}\right)\sin\theta\cos\theta
  \d{\theta}\\
  & = & \dfrac{4^4}{4}\dint  _{\arcsin \frac{\pi}{4}} ^{\arccos
  \frac{1}{4}} \sin\theta\cos \theta\d{\theta} \\ & & \quad -  \dfrac{1}{4}\dint 
 ^{\frac{\pi}{4}}
  _{\arcsin
  \frac{1}{4}} (\cot \theta)(\csc^2\theta) \d{\theta} \\ & & \quad 
-\dfrac{1}{4}\dint  _{\frac{\pi}{4}} ^{\arccos
  \frac{1}{4}} (\tan\theta)(\sec^2\theta)\d{\theta}\\
  & = & 28-\dfrac{7}{4}-\dfrac{7}{4}\\
  & = & \dfrac{49}{2}
\end{array}$$

\end{answer}
\end{pro}
\vspace{2cm}
\begin{figure}[!hptb]
\centering \psset{unit=1pc}
\psaxes[labels=none,linewidth=2pt](0,0)(-4,-4)(4,4)
\pstGeonode[PointName=none,PointSymbol=none](0,0){O}(1,1){A}(1.732050808,1){B}(1
,1.732050808){C}(4,0){D}
\pstInterLC[PointName=none,PointSymbol=none]{A}{B}{O}{D}{M}{M'}
\pstInterLC[PointName=none,PointSymbol=none]{A}{C}{O}{D}{N}{N'}
\pscustom[fillcolor=red,fillstyle=solid]{
\psline(A)(M')\pstArcOAB{O}{M'}{N'}\psline(N')(A) }
\vspace{2cm}\footnotesize\hangcaption{Problem
\ref{pro:inside-circles}.}\label{fig:inside-circles1}
\end{figure}


\begin{pro}

Find $$\dint \limits_D (x^2 - y^2) \d{A} $$ where
$$D = \{(x, y)\in\reals^2|(x - 1)^2 + y^2 \leq 1\}.$$
\begin{answer} Using polar coordinates,
$$\begin{array}{lll}
\dint \limits_D x^2 - y^2 \d{x}\d{y} & = & \dint _{-\pi/2}
^{\pi/2}\left(\dint ^{2\cos\theta} _0 \rho^3 \ \d{\rho}\right)
(\cos^2\theta - \sin^2\theta) \d{\theta} \\
& = & 8 \dint _0 ^{\pi/2} \cos^4\theta(\cos^2\theta - \sin^2\theta) \
\d{\theta} \\
& = & \pi.
\end{array}$$
\end{answer}
\end{pro}
\begin{pro}
Find $$\dint \limits_D \sqrt{xy} \d{A}$$ where
$$D = \{(x, y)\in\reals^2|(x^2 + y^2)^2 \leq 2xy\}.$$
\begin{answer} Using polar coordinates,
$$\begin{array}{lll}
\dint \limits_D \sqrt{xy} \d{x}\d{y} & = & 4\dint _{0}
^{\pi/4}\left(\dint ^{\sqrt{\sin 2\theta}} _0 \rho\sqrt{\rho
^2\cos\theta\sin\theta} \ \d{\rho}\right) \d{\theta} \\
& = & \dfrac{4}{3} \dint _0 ^{\pi/4} (\sqrt{\sin
2\theta})^3\sqrt{\cos\theta\sin\theta} \ \d{\theta} \\
& = & \dfrac{4}{3\sqrt{2}} \dint _0 ^{\pi/4} \sin^2 2\theta \ \d{\theta} \\
& = & \dfrac{\pi\sqrt{2}}{12}.
\end{array}$$
\end{answer}
\end{pro}
\begin{pro}Find $\dint _D f \d{A}$ where $$D =
\{(x, y)\in\reals^2:b^2x^2 + a^2y^2 = a^2b^2, (a, b)\in ]0;+\infty[
\ {\rm fixed}\}$$ and $f(x, y) = x^3 + y^3$. \begin{answer} Using $x
= a\rho \cos \theta$, $y = b\rho\sin\theta,$ the integral becomes
$$
(ab)\left(\dint _0 ^{2\pi} a^3\cos^3\theta +
b^3\sin^3\theta\d{\theta}\right)\left(\dint _0 ^{1} \rho^4
\d{\rho}\right)  = \frac{2}{15}(ab)(a^3 + b^3). $$
\end{answer}
\end{pro}
\begin{pro}
Let $a>0$ and $b>0$. Prove that
$$ \dint _R \sqrt{\dfrac{a^2b^2-a^2y^2-b^2x^2}{a^2b^2+a^2y^2+b^2x^2}}\d{A} 
=\dfrac{\pi ab(\pi -2)}{8},$$
where $R$ is the region bounded by the ellipse
$\dfrac{x^2}{a^2}+\dfrac{y^2}{b^2}=1$ and the first quadrant.
\end{pro}
\begin{pro}
Prove that
$$ \dint _R \dfrac{y}{\sqrt{x^2+y^2}}\d{x}\d{y}=\sqrt{2}-1, $$
where $$R=\{(x,y)\in \reals^2:0<x<1, o<y<x^2\}.$$
\end{pro}
\begin{pro}
Prove that the ellipse $$(x - 2y + 3)^2 + (3x + 4y - 1)^2 = 4$$
bounds an area of $\dfrac{2\pi}{5}$.
\end{pro}
\begin{pro} Find $$\dint \limits_D \sqrt{x^2 + y^2} \d{A} $$ where $$D =
\{(x, y)\in\reals^2|x \geq 0, y \geq 0, x^2 + y^2 \leq 1, x^2 + y^2
- 2y \geq
 0\}.$$
\begin{answer} Using polar coordinates,
$$\begin{array}{lll}
\dint \limits_D f(x, y) \d{A} & = & \dint _{0} ^{\pi/6}\left(\dint
_{2\sin\theta} ^1 \rho^2 \ \d{\rho}\right)
\d{\theta} \\
& = & \dfrac{1}{3} \dint _0 ^{\pi/6} (1 - 8\sin^3\theta) \ \d{\theta} \\
& = & \dfrac{\pi}{18} - \dfrac{16}{9} + \sqrt{3}.
\end{array}$$
\end{answer}
\end{pro}
\begin{pro} Find $\dint _D f \d{A}$ where
$$D =  \{(x, y)\in\reals^2|y \geq 0, x^2 + y^2 - 2x \leq 0\}$$ and
$f(x, y) = x^2y$.  \begin{answer} Using polar coordinates the
integral becomes
$$\dint _0 ^{\pi/2}\left(\dint _0 ^{2\cos\theta} \rho^4 \d{\rho}\right) 
\cos^2\theta\sin\theta \d{\theta} = \frac{4}{5}.$$

\end{answer}
\end{pro}
\begin{pro}
Let $D=\{(x,y)\in \reals^2: x\geq 1, x^2+y^2 \leq 4\}$. Find $\dint
_D x\d{A}$.
\end{pro}

\begin{pro} Find $\dint _D f \d{A}$ where
$$D =  \{(x, y)\in\reals^2|x \geq 1, x^2 + y^2 - 2x \leq 0\}$$ and
$f(x, y) = \frac{1}{(x^2 + y^2)^2}$. \begin{answer} Using polar
coordinates the integral becomes
$$\dint _{-\pi/4} ^{\pi/4}\left(\dint_{1/\cos\theta} ^{2\cos\theta} 
\frac{1}{\rho^3}\d{\rho}\right)\d{\theta}  =
\dint_0 ^{\pi/4} \left(\cos^2\theta -
\frac{\sec^2\theta}{4}\right)\d{\theta} = \frac{\pi}{8}.
$$
\end{answer}
\end{pro}

\begin{pro}
 Let $$D = \{(x, y) \in \reals^2: x^2 + y^2 - y \leq 0, x^2 + y^2
- x \leq 0\}.$$ Find the integral $$\dint_D (x +
y)^2\d{A}.$$\begin{answer} Put  $$D' = \{(x, y) \in \reals^2:
 y \geq x, x^2 + y^2 - y \leq 0, x^2 + y^2 - x \leq 0\}.$$ Then
 the integral equals$$2\dint_{D'} (x + y)^2\d{x}\d{y}.$$Using polar coordinates 
the integral equals
$$\begin{array}{lll}2\dint _{\pi/4} ^{\pi/2}(\cos\theta + 
\sin\theta)^2\left(\dint_0 ^{\cos\theta} \rho^3{\rm d\rho}\right) \d{\theta}
& = & \frac{1}{2}\dint _{\pi/4} ^{\pi/2} \cos^4\theta (1 +
2\sin\theta\cos\theta)\d{\theta} \\ &  = &  \frac{3\pi}{64} -
\frac{5}{48}. \end{array}$$
\end{answer}
\end{pro}
\begin{pro}
 Let $D = \{(x, y)\in\reals^2| y \leq x^2
+ y^2 \leq 1\}$. Compute
$$\dint _{D} \frac{\d{A}}{(1 + x^2 + y^2)^2}.$$
\begin{answer} Observe that $D = D_2 \setminus D_1$ where $D_2$ is the disk
limited by the equation $x^2 + y^2 = 1$ and $D_1$ is the disk
limited by the equation $x^2 + y^2 = y.$ Hence
$$\dint _{D} \frac{\d{x}\d{y}}{(1 + x^2 + y^2)^2} = \dint _{D_2} 
\frac{\d{x}\d{y}}{(1 + x^2 + y^2)^2}
- \dint _{D_1} \frac{\d{x}\d{y}}{(1 + x^2 + y^2)^2}.$$ Using polar
coordinates we have
$$\dint _{D_2} \frac{\d{x}\d{y}}{(1 + x^2 + y^2)^2} =
\dint _{0} ^{2\pi}\dint_0 ^1 \frac{\rho}{(1 + \rho^2)^2} \d{\rho}
\d{\theta} =  \frac{\pi}{2}$$and
$$\begin{array}{lll}\dint _{D_1} \frac{\d{x}\d{y}}{(1 + x^2 + y^2)^2} &
=& 2\dint _{0} ^{\pi/2}\dint_0 ^{\sin\theta} \frac{\rho}{(1 +
\rho^2)^2} \d{\rho} \d{\theta} =  \dint _0 ^{\pi/2}
\frac{\sin^2\theta \d{\theta}}{1 + \sin^2\theta} \\ & = & \dint _0
^{+\infty}\frac{\d{t}}{t^2 + 1} -  \frac{{\rm d}t}{2t^2 + 1} =
\frac{\pi}{2} - \frac{\pi\sqrt{2}}{4}. \\ \end{array}
$$(We evaluated this last integral using $t = \tan\theta$) Finally, the integral 
equals
$$\frac{\pi}{2} - \left(\frac{\pi}{2} -
\frac{\pi\sqrt{2}}{4}\right) = \frac{\pi\sqrt{2}}{4}.$$



\end{answer}
\end{pro}


\begin{pro}
Evaluate $$\dint \limits_{\{(x,y)\in\reals^2: x \geq 0, y \geq 0,
x^4 + y^4 \leq 1\}} \ \ x^3y^3 \sqrt{1 - x^4-y^4} \ \d{A}
$$using $x^2 = \rho\cos \theta$, $y^2 = \rho\sin\theta$.
\begin{answer} We have $$2x\d{x} = \cos\theta\d{\rho} -
\rho\sin\theta\d{\theta}, \ \ \ 2y\d{y} = \sin\theta\d{\rho} +
\rho\cos\theta\d{\theta},
$$whence $$ 4xy\d{x}\wedge \d{y} = \rho\d{\rho}\wedge\d{\theta}.    $$
It follows that $$\begin{array}{lll} x^3y^3 \sqrt{1 - x^4-y^4} \
\d{x}\wedge\d{y} & = & \frac{1}{4}(x^2y^2)(\sqrt{1 - x^4 -
y^4})(4xy\ \d{x}\wedge \d{y}) \\ & = &
\frac{1}{4}(\rho^{3}\cos\theta\sin\theta\sqrt{1-\rho^2})\d{\rho}\wedge\d{\theta}
\end{array}$$ Observe that $$x^4 + y^4 \leq 1 \implies \rho^2\cos^2\theta +
\rho^2\sin^2\theta \leq 1 \implies \rho \leq 1.$$Since the
integration takes place on the first quadrant, we have $0 \leq
\theta \leq \pi/2$. Hence the integral becomes
$$\begin{array}{lll} \dint _0 ^{\pi/2} \dint _0 ^1  
\frac{1}{4}(\rho^{3}\cos\theta\sin\theta\sqrt{1-\rho^2})\d{\rho}\d{\theta} & = &
\frac{1}{4}\left(\dint _0
^{\pi/2}\cos\theta\sin\theta\d{\theta}\right)\left(\dint _0
^1\rho^{3}\sqrt{1 - \rho^2}\d{\rho}\right) \\
& = & \frac{1}{4}\cdot\frac{1}{2}\cdot\frac{2}{15} \\ & = &
\frac{1}{60}.
\end{array} $$


\end{answer}
\end{pro}


\begin{pro}
William Thompson (Lord Kelvin) is credited to have said: ``A
mathematician is someone to whom
$$\dint _0 ^{+\infty} e^{-x^2} \d{x}= \frac{\sqrt{\pi}}{2}$$is
as obvious as twice two is four to you. Liouville was a
mathematician.'' Prove that $$\dint _0 ^{+\infty} e^{-x^2} \d{x}=
\frac{\sqrt{\pi}}{2}$$ by following these steps.
\begin{dingautolist}{202}
\item  Let $a > 0$ be a real number and put $D_a = \{(x,
y)\in\reals^2| x^2 + y^2 \leq a^2\}$. Find
$$I_a = \dint _{D_a} e^{-(x^2 + y^2)} \ \d{x} \d{y}.$$
\item  Let $a > 0$ be a real number and put $\Delta_a = \{(x,
y)\in\reals^2| |x| \leq a, |y| \leq a\}$. Let
$$J_a = \dint _{\Delta_a} e^{-(x^2 + y^2)} \ \d{x} \d{y}.$$Prove that
$$I_a \leq J_a \leq I_{a\sqrt{2}}.$$
\item  Deduce that
$$\dint _0 ^{+\infty} e^{-x^2} \d{x}= \frac{\sqrt{\pi}}{2}.$$

\end{dingautolist}
\begin{answer}
\begin{dingautolist}{202}
\item   Using polar coordinates
$$I_a = \dint_0 ^{2\pi}\left(\dint _0 ^a\rho 
e^{-\rho^2}\d{\rho}\right)\d{\theta} = \pi(1 - e^{-a^2}).$$
\item   The domain of integration of $J_a$ is a square of side
$2a$ centred at the origin. The respective domains of integration of
$I_a$ and $I_{a\sqrt{2}}$ are the inscribed and the exscribed
circles to the square. \item  First observe that
$$J_a = \left(\dint_{-a} ^a e^{-x^2}\d{x}\right)^2.$$ Since both
$I_a$ and $I_{a\sqrt{2}}$ tend to $\pi$ as $a\rightarrow +\infty$,
we deduce that $J_a \rightarrow \pi.$ This gives the result.

\end{dingautolist}
\end{answer}
\end{pro}
\begin{pro}
Let $D = \{(x,y)\in \reals^2: 4 \leq x^2 + y^2 \leq 16\}$ and
$f(x,y) = \dfrac{1}{x^2 + xy + y^2}$. Find $\dint \limits_D
f(x,y)\d{A}$.
\begin{answer}
$$\begin{array}{lll}
\dint \limits_{4 \leq x^2 + y^2 \leq 16} \dfrac{1}{x^2 + xy +
y^2}\d{A} & = & \dint _0 ^{2\pi} \dint _2 ^4 \dfrac{r}{r^2 +
r^2\sin\theta\cos\theta}\d{r}\d{\theta} \\
& = & \dint _0 ^{2\pi} \dint _2 ^4 \dfrac{1}{r(1 +
\sin\theta\cos\theta)}\d{r}\d{\theta} \\
& = & \left(\dint _0 ^{2\pi} \dfrac{\d{\theta}}{1 +
\sin\theta\cos\theta}\right)\left(\dint _2 ^4 \dfrac{\d{r}}{r}\right) \\
& = &  \left(\dint _0 ^{2\pi} \dfrac{\d{\theta}}{1 +
\sin\theta\cos\theta}\right)\log 2\\
& = &  2\left(\dint _0 ^{2\pi} \dfrac{\d{\theta}}{2 + \sin
2\theta}\right)\log 2 \\
& = &  4\left(\dint _0 ^{\pi} \dfrac{\d{\theta}}{2 + \sin
2\theta}\right)\log 2 \\
& = & 4I\log 2,
\end{array}$$so the problem reduces to evaluate $I = \dint _0 ^{\pi} 
\dfrac{\d{\theta}}{2 +
\sin 2\theta}$. To find this integral, we now use what has been
dubbed as ``the world's sneakiest substitution''\footnote{by Michael
Spivak, whose {\em Calculus} book I recommend greatly.}: we put
$\tan \theta = t$. In so doing we have to pay attention to the fact
that $\theta \mapsto \tan \theta $ is not continuous on $[0;\pi]$,
so we split the interval of integration into two pieces, $[0;\pi] =
[0;\frac{\pi}{2}] \cup ]\frac{\pi}{2};\pi]$. Then $\sin 2\theta =
\dfrac{2t}{1+t^2}$, $\cos 2\theta = \dfrac{1-t^2}{1+t^2}$,
$\d{\theta} = \dfrac{\d{t}}{1+t^2}$. Hence

$$\begin{array}{lll} \dint _0 ^{\pi} \dfrac{\d{\theta}}{2 +
\sin 2\theta}  & = &  \dint _0 ^{\pi/2} \dfrac{\d{\theta}}{2 + \sin
2\theta} +  \dint _{\pi/2} ^{\pi} \dfrac{\d{\theta}}{2 +
\sin 2\theta}\\
&  = & \dint _0 ^{+\infty} \dfrac{\frac{\d{t}}{1+t^2}}{2 +
\frac{2t}{1+t^2}} + \dint ^0 _{-\infty} \dfrac{\frac{\d{t}}{1+t^2}}{2
+ \frac{2t}{1+t^2}} \\ & =  & \dint _0 ^{+\infty}
\dfrac{\d{t}}{2(t^2+t+1)} + \dint ^0 _{-\infty}
\dfrac{\d{t}}{2(t^2+t+1)} \\ & = & \dfrac{2}{3}\dint _0 ^{+\infty}
\dfrac{\d{t}}{(\frac{2t}{\sqrt{3}}+\frac{1}{\sqrt{3}})^2+1} +
\dfrac{2}{3}\dint ^0 _{-\infty}
\dfrac{\d{t}}{(\frac{2t}{\sqrt{3}}+\frac{1}{\sqrt{3}})^2+1}
\\ & = & \dfrac{\sqrt{3}}{3}\Big| _0 ^{+\infty}\arctan
\left(\dfrac{2t\sqrt{3}}{3}+\frac{\sqrt{3}}{3}\right) +
\dfrac{\sqrt{3}}{3}\Big| ^0 _{-\infty}\arctan
\left(\dfrac{2t\sqrt{3}}{3}+\frac{\sqrt{3}}{3}\right) \\ & = &
\dfrac{\sqrt{3}}{3}\left(\dfrac{\pi}{2}-\dfrac{\pi}{6}\right) +
\dfrac{\sqrt{3}}{3}\left(\dfrac{\pi}{6}-\left(-\dfrac{\pi}{2}\right)\right)
\\
& = & \dfrac{\pi\sqrt{3}}{3}.\end{array}$$ We conclude that
$$\dint \limits_{4 \leq x^2 + y^2 \leq 16} \dfrac{1}{x^2 + xy +
y^2}\d{A} = \dfrac{4\pi\sqrt{3}\log 2}{3}.   $$

\end{answer}
\end{pro}
\begin{pro}
 Prove that every closed convex region in the
plane of area $ \geq \pi$ has two points which are two units apart.
\begin{answer}Recall from formula \ref{eq:area_inside_curve} that the area
enclosed by a simple closed curve $\Gamma$ is given by $$
\frac{1}{2}\dint _\Gamma x\d{y} - y\d{x}.
$$Using polar  coordinates $$\begin{array}{lll}x\d{y} - y\d{x} & = &
(\rho\cos\theta)(\sin\theta\d{\rho} + \rho\cos\theta\d{\theta}) -
(\rho\sin\theta)(\cos\theta\d{\rho} - \rho\sin\theta\d{\theta}) \\
& = & \rho^2\d{\theta}.
\end{array}$$
 Parametrise the curve enclosing the region by polar
coordinates so that the region is tangent to the polar axis at the
origin. Let the equation of the curve be $\rho = f(\theta)$. The
area of the region is then given by
$$\frac{1}{2}\dint_0 ^\pi \rho^2 \d{\theta} = \frac{1}{2}\dint_0 ^\pi 
(f(\theta))^2 \d{\theta}
=  \frac{1}{2}\dint_0 ^{\pi/2}( (f(\theta))^2  +   (f(\theta +
\pi/2))^2 )\d{\theta} .$$ By the Pythagorean Theorem, the integral
above is the integral of the square of the chord in question. If no
two points are farther than 2 units, their squares are no farther
than 4 units, and so the area
$$< \frac{1}{2}\dint_0 ^{\pi/2} 4\d{\theta} = \pi,$$a
contradiction.
\end{answer}
\end{pro}
\begin{pro} In the $xy$-plane, if $R$ is the set of points
inside and on a convex polygon, let $D(x, y)$ be the distance from
$(x, y)$ to the nearest point $R$. Show that
$$\dint _{-\infty} ^{+\infty}\dint _{-\infty} ^{+\infty} e^{-D(x, y)} \ \d{x} 
\d{y} = 2\pi +  L +
A,$$where $L$ is the perimeter of $R$ and $A$ is the area of $R$.
\begin{answer} Let $I(S)$ denote the integral sought over a region $S$. Since
$D(x, y) = 0$ inside $R$, $I(R) = A$. Let ${\mathscr  L}$ be a side
of $R$ with length $l$ and let $S({\mathscr  L})$ be the half strip
consisting of the points of the plane having a point on ${\mathscr
L}$ as nearest point of $R$. Set up coordinates $uv$ so that $u$ is
measured parallel to ${\mathscr  L}$ and $v$ is measured
perpendicular to $L$. Then
$$I(S({\mathscr  L})) = \dint _{0} ^{l}\dint _{0} ^{+\infty} e^{-v} \ \d{u}\d{v} 
=
l.$$The sum of these integrals over all the sides of $R$ is $L$.


If ${\mathscr  V}$ is a vertex of $R$, the points that have
${\mathscr  V}$ as nearest from $R$ lie inside an angle $S({\mathscr
V})$ bounded by the rays from ${\mathscr  V}$ perpendicular to the
edges meeting at ${\mathscr  V}$. If $\alpha$ is the measure of that
angle, then using polar coordinates
$$I(S({\mathscr  V})) = \dint _{0} ^{\alpha}\dint _{0} ^{+\infty} \rho e^{-\rho} 
\ \d{\rho} \d{\theta} =
\alpha.$$The sum of these integrals over all the vertices of $R$ is
$2\pi$. Assembling all these integrals we deduce the result.
\end{answer}
\end{pro}

  \section{Triple integrals}

  Triple integrals are just like double integrals, except we integrate over 
regions in $\bbR^3$ instead of $\bbR^2$.
  Let $C$ be the cuboid $C = [a_1, b_1] \times [a_2, b_2] \times [a_3, b_3]$ and 
$f:C \to \bbR$ be a function.
  As before, define the Riemann sum
  \begin{equation*}
    \mathcal R(f, P, \Xi)
      = \dsum_{i = 0}^{N_1} \dsum_{j = 0}^{N_2} \dsum_{k = 0}^{N_3} f( \xi_{i, 
j, k} ) (x_{i+1} - x_i)(y_{i+1} - y_i)(z_{i+1} - z_i)
  \end{equation*}
  define the Riemann integral of $f$ by taking the limit of Riemann sums:
  \begin{equation*}
    \dint_U f \, dV = \lim_{\norm{P} \to 0} \mathcal R(f, P, \Xi).
  \end{equation*}
  Here we use $dV$ (or sometimes $dx \, dy \, dz$ to denote that the integral is 
a volume (or triple) integral.

  When dealing with unbounded functions over unbounded domains,%
  \footnote{%
    As before, we make the ``niceness'' assumption that the boundary of $U$ is a 
differentiable surface.
  }
  we use the same limiting procedure.
  If
  \begin{equation*}
    \lim_{R \to \infty} \dint_{U \cap B(0, R)} \chi*R \abs{f} \, dV
  \end{equation*}
  exists and is finite then we define
  \begin{equation*}
    \dint_U f \, dV = \lim_{R \to \infty} \dint_{U \cap B(0, R)} \chi*R f \, dV
  \end{equation*}

  We can break a volume integral into three iterated integrals, and Fubini's 
theorem is still true.
  Rather than restate everything, we do a few examples.

  \begin{example}
    Let $U = B(0, R) \subseteq \bbR^3$.
    Compute $\dint_U 1 \, dV$ and derive a formula for the volume.
  \end{example}
  \begin{proof}[Solution]
    Note
    \begin{align}
      \dint_U 1 \, dV
	&= \dint_{x = -R}^R \dint_{y = -\sqrt{R^2 - x^2}}^{\sqrt{R^2 - x^2}} 
\dint_{z = -\sqrt{R^2 - x^2 - y^2}}^{\sqrt{R^2 - x^2 - y^2}} 1 \, dz \, dy \, dx
	\\
	&= 2 \dint_{x = -R}^R \dint_{y = -\sqrt{R^2 - x^2}}^{\sqrt{R^2 - x^2}} 
\sqrt{R^2 - x^2 - y^2} dy \, dx
	\\
	&= 2 \dint_{x = -R}^R \dint_{\theta = -\frac{\pi}{2}}^{\frac{\pi}{2}} 
(R^2 - x^2)\cos^2 \theta \, d\theta \, dx
	= \pi \left(R^2 x - \frac{x^3}{3} \right)_{-R}^R
	= \frac{4}{3} \pi R^3
	\qed
    \end{align}
  \end{proof}

 
\begin{exa}
Find $$\dint \limits_{[0;1]^3} \ x^2ye^{xyz} \ \d{V}.$$
\end{exa}
\begin{solu}The integral is
$$\begin{array}{lll}
\dint_0 ^1 \left(\dint_0 ^1 \left(\dint_0 ^1 x^2ye^{xyz}\
\d{z}\right)\d{y}\right) \d{x}& = & \dint_0 ^1 \left(\dint_0 ^1
x(e^{xy} - 1)\ \d{y}\right) \d{x}\\
& = & \dint_0 ^1 (e^x - x - 1) \d{x}\\
& = & e  - \dfrac{5}{2}.
\end{array}$$
\end{solu}
\begin{exa}
Find $\dint \limits_R z \ \d{V}$ if
$$R = \{(x, y, z)\in\reals^3| x \geq 0, y \geq 0, z \geq 0, \sqrt{x}
+ \sqrt{y} + \sqrt{z} \leq 1\}.$$
\end{exa}
\begin{solu} The integral is
$$\begin{array}{lll}
\dint _R z\d{x}\d{y}\d{z} & =  &  \dint _0 ^1 z\left(\dint _0 ^{(1 -
\sqrt{z})^2}\left( \dint _0 ^{(1 - \sqrt{z} - \sqrt{x})^2} \d{y}
\right)\d{x}\right)\
\d{z} \\
& = &  \dint _0 ^1 z\left(\dint _0 ^{(1 - \sqrt{z})^2} (1 - \sqrt{z} -
\sqrt{x} )^2\d{x}\right)\
\d{z} \\
& = &  \dfrac{1}{6}\dint _0 ^1 z(1 - \sqrt{z})^4\
\d{z} \\
 & = & \dfrac{1}{840}.
\end{array}$$
\end{solu}
\begin{exa}
Prove that $$\dint \limits_{V} x \d{V} = \dfrac{a^2bc}{24},
$$where $V$ is the tetrahedron $$V = \left\{(x,y,z)\in\reals^3: x \geq 0, y \geq 
0, z \geq 0, \dfrac{x}{a} + \dfrac{y}{b} + \dfrac{z}{c} \leq
1\right\}.$$\end{exa} \begin{solu} We have
$$\begin{array}{lll}\dint \limits_{V} x \d{x}\d{y}\d{z} & = &
\dint_0 ^{c}  \dint _0 ^{b-bz/c}  \dint _0 ^{a - {ay}/{b}-{az}/{c} }         x 
\d{x}\d{y}\d{z} \\
& = & \frac{1}{2}\dint_0 ^{c}  \dint _0 ^{b-bz/c}  \left(a - 
\dfrac{ay}{b}-\dfrac{az}{c}\right)^2\d{y}\d{z} \\
&  = & \frac{1}{6} \dint _0 ^c \frac {{a}^{2} \left( -z+c \right) 
^{3}b}{{c}^{3}} \d{x} \\
& = & \dfrac{a^2bc}{24}
\end{array}$$
\end{solu}
\vspace{2cm}
\begin{figure}[htpb]
\begin{minipage}{7cm}
\centering \psset{unit=2pc}\pstThreeDCoor[IIIDticks=false,xMin=
0,xMax=3.5,yMin=0,yMax=4.5,zMin=0,zMax=4.5] \pstThreeDNode(0,0,0){O}
     \psdot[linecolor=red](A) \uput[35](O){$O$}
      \pstThreeDNode(3,2,0){A}
     \psdot[linecolor=red](A) \uput[-90](A){$A$}
      \pstThreeDNode(0,3,0){B}
     \psdot[linecolor=red](B) \uput[-90](B){$B$}
     \pstThreeDNode(0,0,2){C}
     \psdot[linecolor=red](C) \uput[0](C){$C$}
     \pstLineAB[linewidth=2pt]{O}{A}
      \pstLineAB[linewidth=2pt]{O}{B}
       \pstLineAB[linewidth=2pt]{O}{C}
       \pspolygon[linewidth=2pt](A)(B)(C)
       \vspace{2cm}\footnotesize\hangcaption{Problem
\ref{pro:vol-integral}.} \label{fig:vol-integral}
\end{minipage} \hfill
\begin{minipage}{7cm}\centering\psset{unit=1pc}
\psaxes[subticks=0,labels=none]{->}(0,0)(-5,-2)(5,5)
\pstGeonode[PosAngle={0,30,210},dotscale=2](3,2){A}(0,3){B}(0,0){C}
\pspolygon[linewidth=2pt](A)(B)(C)
\vspace{1cm}\footnotesize\hangcaption{$xy$-projection.}\label{fig:xyprojection}
\end{minipage}
\end{figure}

\begin{exa}
\label{pro:vol-integral} Evaluate the integral $\dint _{S} x\d{V}$
where $S$ is the (unoriented) tetrahedron with vertices $(0,0,0)$,
$(3,2,0)$, $(0,3,0)$, and $(0,0,2)$. See figure
\ref{fig:vol-integral}.
\end{exa}
\begin{solu}A short computation shews that the plane passing through
$(3,2,0)$, $(0,3,0)$, and $(0,0,2)$ has equation $2x+6y+9z=18$.
Hence, $0 \leq z \leq \dfrac{18-2x-6y}{9}.$ We must now figure out
the $xy$ limits of integration. In figure \ref{fig:xyprojection} we
draw the projection of the tetrahedron on the $xy$ plane. The line
passing through $AB$ has equation $y=-\dfrac{x}{3}+3$. The line
passing through $AC$ has equation $y=\dfrac{2}{3}x$.

We find, finally,
$$ \begin{array}{lll} \dint _{S} x\d{V} & = & \dint _0 ^3 \dint _{2x/3} ^{3-x/3} 
\dint _0 ^{(18-2x-6y)/9} x\d{z}\d{y}\d{x} \\
& = & \dint _0 ^3 \dint _{2x/3} ^{3-x/3} \dfrac{18x-2x^2-6yx}{9}
\d{y}\d{x}\\
& = & \dint _0 ^3   \dfrac{18xy-2x^2y-3y^2x}{9} \Big| _{2x/3}
^{3-x/3} \d{x}\\
& = &\dint _0 ^3 \left(\dfrac{x^3}{3}-2x^2+3x\right)\d{x} \\
& = & \dfrac{9}{4}
 \end{array}$$





\end{solu}

\begin{exa}
Evaluate $\dint _R xyz \d{V}$, where $R$ is the solid formed by the
intersection of the parabolic cylinder $z=4-x^2$, the planes $z=0$,
$y=x$, and $y=0$. Use the following orders of integration:
\begin{enumerate}
\item $\d{z}\d{x}\d{y}$
\item $\d{x}\d{y}\d{z}$
\end{enumerate}
\end{exa}
\begin{solu}
We must find the projections of the solid on the the coordinate
planes.
\begin{enumerate}
\item With the order $\d{z}\d{x}\d{y}$, the limits of integration of $z$ can 
only depend, if at all, on $x$ and $y$.
Given an arbitrary point in the solid,  its lowest $z$ coordinate is
$0$ and its highest one is on the cylinder, so the limits for $z$
are from $z=0$ to $z=4-x^2$. The projection of the solid on the
$xy$-plane is the area bounded by the lines $y=x$, $x=2$, and the
$x$ and $y$ axes.
$$\begin{array}{lll}\dint _0 ^2\dint _0 ^y\dint _0 ^{4-x^2}   xyz 
\d{z}\d{x}\d{y} &
= &  \dfrac{1}{2}\dint _0 ^2 \dint _0 ^y  xy(4-x^2)^2 \d{x}\d{y}\\
&
= &  \dfrac{1}{2}\dint _0 ^2 \dint _0 ^y  y(16x-8x^3+x^5) \d{x}\d{y}\\
&
= &  \dint _0 ^2 \left(4y^3-y^5+\dfrac{y^7}{12}\right)\d{y}\\
& = & 8.
\end{array}$$
\item With the order $\d{x}\d{y}\d{z}$, the limits of integration of $x$ can 
only depend, if at all, on $y$ and $z$.
Given an arbitrary point in the solid,   $x$ sweeps from the plane
to $x=2$, so the limits for $x$ are from $x=y$ to $x=\sqrt{4-z}$.
The projection of the solid on the $yz$-plane is the area bounded by
$z=4-y^2$, and the  $z$ and $y$ axes.
$$\begin{array}{lll}\dint _0 ^4\dint _0 ^{\sqrt{4-z}}\dint _y ^2   xyz 
\d{x}\d{y}\d{z} &
= &  \dfrac{1}{2}\dint _0 ^4 \dint _0 ^{\sqrt{4-z}}  (4y-y^3)z \d{y}\d{z}\\
&
= &  \dint _0 ^4 \left(2z-\dfrac{z^3}{8}\right)  \d{z}\\
& = & 8.
\end{array}$$
\end{enumerate}
\end{solu}

\section*{\psframebox{Homework}}
\begin{pro}
Compute $\dint _E z\d{V}$ where $E$ is the region in the first octant
bounded by the planes $y+z = 1$ and $x+z = 1$.
\begin{answer}
We have
$$\begin{array}{lll}
\dint _E z\d{V} & = & \dint _0 ^1 \dint _0 ^{1-y}\dint _0
^{1-z}z\d{x}\d{z}\d{y}
\\
& = & \dint _0 ^1 \dint _0 ^{1-y}z-z^2\d{z}\d{y}
\\
& = & \dint _0 ^1 \dfrac{(1-y)^2}{2}-\dfrac{(1-y)^3}{3}\d{y}
\\
& = &  \dfrac{(1-y)^4}{12}-\dfrac{(1-y)^3}{6}\Big| _0 ^1
\\
& = & =\dfrac{1}{6}.
\end{array}$$
\end{answer}
\end{pro}
\begin{pro}
Consider the solid $S$ in the first octant, bounded by the parabolic
cylinder $z=2-\dfrac{x^2}{2}$ and the planes $z=0$, $y=x$, and
$y=0$. Prove that $\dint _S xyz = \dfrac{2}{3}$ first by integrating
in the order $\d{z}\d{y}\d{x}$, and then by integrating in the order
$\d{y}\d{x}\d{z}$.
\end{pro}

\begin{pro}
 Evaluate the integrals $\dint _R 1\d{V}$ and  $\dint _R x\d{V}$, where $R$ is 
the
tetrahedron with vertices at $(0,0,0)$, $(1,1,1)$, $(1,0,0)$, and
$(0,0,1)$.
\begin{answer}
Let $A=(1,1,1)$, $B=(1,0,0)$,  $C=(0,0,1)$, and $O=(0,0,0)$. We have
four planes passing through each triplet of points:
$$\begin{array}{lll}P_1: & A, B, C, &  x-y+z=1\\
P_2: & A, B, O & z=y \\
P_3: & A,C, O & x=y \\
P_4: & B, C, O & y = 0. \\
  \end{array}$$
Using the order of integration $\d{z}\d{x}\d{y}$, $z$ sweeps from
$P_2$ to $P_1$, so the limits are $z=y$ to $z=1-x+y$. The projection
of the solid on the $xy$ plane produces the region bounded by the
lines $x=0$, $x=1$ and $x=y$ on the first quadrant of the
$xy$-plane. Thus
$$\begin{array}{lll} \dint _0 ^1\dint _0 ^{x} \dint _y ^{1-x+y} \d{z}\d{y}\d{x} 
& = &
\dint _0 ^1\dint _0 ^{x} (1-x)\d{y}\d{x} \\
& = & \dint _0 ^1 \left(x-x^2\right)\d{x}\\
& = & \left(\dfrac{x^2}{2}-\dfrac{x^3}{3}\right)\Big|_0 ^1\\
& = & \dfrac{1}{6}.
\end{array}$$


\bigskip


 We use the same limits of integration as in the previous integral. We have
$$\begin{array}{lll} \dint _0 ^1\dint _0 ^{x} \dint _y ^{1-x+y}x \d{z}\d{y}\d{x} 
& = &
\dint _0 ^1\dint _0 ^{x} (x-x^2)\d{y}\d{x} \\
& = & \dint _0 ^1 \left(x^2-x^3\right)\d{x}\\
& = & \left(\dfrac{x^3}{3}-\dfrac{x^4}{4}\right)\Big|_0 ^1\\
& = & \dfrac{1}{12}.
\end{array}$$

\end{answer}
\end{pro}

\begin{pro}
Compute $\dint _E x\d{V}$   where $E$ is the region in the first
octant bounded by the plane $y=3z$ and the cylinder $x^2+y^2 = 9$.
\begin{answer}
We have
$$
\dint _E x\d{V}  = \dint _0 ^3 \dint _0 ^{\sqrt{9-x^2}}\dint _0
^{y/3}x\d{z}\d{y}\d{x}=\dfrac{27}{8}.
$$
\end{answer}
\end{pro}
\begin{pro}
Find $\dint \limits_D \dfrac{\d{V}}{(1+x^2z^2)(1+y^2z^2)}$ where
$$D=\{(x,y,z)\in \reals^3: 0 \leq x \leq 1, 0 \leq y \leq 1, z \geq 0\}.
$$\begin{answer} The desired integral is
$$\begin{array}{lll} \dint_0 ^1 \dint_0 ^1 \dint_0 ^\infty 
\dfrac{\d{x}\d{y}\d{z}}{(1+x^2z^2)(1+y^2z^2)}
& = & \dint_0 ^1 \dint_0 ^1 \dint_0 ^\infty
\dfrac{1}{x^2-y^2}\left(\dfrac{x^2}{1+x^2z^2}-\dfrac{y^2}{1+y^2z^2}\right)\d{x}
\d{y}\d{z}\\
& = & \dint_0 ^1 \dint_0 ^1 \dfrac{1}{x^2-y^2}\left(x\arctan (xz) -
y\arctan(yz)\right)\Bigg|_{z=0} ^{z=\infty}\d{x}\d{y}\\
& = & \dint_0 ^1 \dint_0 ^1 \dfrac{\pi
(x-y)}{2(x^2-y^2)}\d{x}\d{y}\\
& = & \dint_0 ^1 \dint_0 ^1 \dfrac{\pi}{2(x+y)}\d{x}\d{y}\\
& = & \dfrac{\pi}{2}\dint _0 ^1 \log (y + 1)  - \log y\d{y} \\
& = & \dfrac{\pi}{2}\cdot( (y + 1)\log (y + 1) - (y + 1) -y\log y + y)\Bigg|_0 ^1 \\
& = & \pi \log 2.
\end{array}$$
\end{answer}
\end{pro}

 \section{Change of Variables}
\begin{exa}
Find $$\dint\limits_R (x + y + z)(x + y - z)(x - y - z) \d{V},
$$where $R$ is the tetrahedron bounded by the planes $x + y + z =
0$, $x + y - z= 0$, $x - y - z = 0$, and $2x - z = 1$.
\end{exa}
\begin{solu} We make the change of variables $$  u  = x + y + z \implies
\d{u} = \d{x} + \d{y} + \d{z},$$
$$  v = x + y - z
\implies \d{v} = \d{x} + \d{y} - \d{z},$$
$$  w  = x - y - z
\implies \d{w} = \d{x} - \d{y} -\d{z}.$$ This gives $$\d{u} \wedge
\d{v} \wedge \d{w} = -4\d{x} \wedge \d{y} \wedge \d{z}.
$$ These forms have opposite orientations, so we choose, say,  $$\d{u} \wedge
\d{w} \wedge \d{v} = 4\d{x} \wedge \d{y} \wedge \d{z}
$$ which have the same orientation. Also,
$$ 2x - z = 1 \implies u + v + 2w = 2.$$ The tetrahedron in the
$xyz$-coordinate frame is mapped into a tetrahedron bounded by $u =
0$, $v = 0$, $u + v + 2w = 1 $ in the $uvw$-coordinate frame. The
integral becomes
$$\frac{1}{4}\dint _0 ^2 \dint _0 ^{1 - v/2} \dint _0 ^{2 - v - 2w} uvw\  \d{u}
\d{w}  \d{v} = \dfrac{1}{180}. $$

Consider a transformation to cylindrical coordinates $$(x, y, z) =
(\rho\cos\theta, \rho\sin\theta, z).$$From what we know about polar
coordinates
$$\d{x} \wedge \d{y} = \rho \d{\rho}\wedge \d{\theta}.   $$
Since the wedge product of forms is associative,
$$\d{x} \wedge \d{y} \wedge \d{z} = \rho \d{\rho}\wedge \d{\theta} \wedge\d{z}.   $$
\end{solu}
\begin{exa}
Find $\dint _R z^2\d{x}\d{y}\d{z}$ if
$$R = \{(x, y, z)\in\reals^3| x^2 + y^2 \leq 1, 0 \leq z \leq 1\}.$$

\end{exa}
\begin{solu}  The region of integration is mapped into
$$\Delta = [0; 2\pi]\times[0; 1]\times[0 ; 1]$$ through a
cylindrical coordinate change. The integral is therefore
$$\begin{array}{lll}
\dint _R f(x, y, z)\d{x}\d{y}\d{z}  & = & \left(\dint _0 ^{2\pi} \
\d{\theta}\right)\left(\dint _0 ^{1} \rho\ \d{\rho}\right)\left(\dint
_0 ^{1} z^2\ \d{z}\right)
\\
& = & \dfrac{\pi}{3}.
\end{array}$$
\end{solu}
\begin{exa}
Evaluate $\dint _D (x^2 + y^2)\d{x}\d{y}\d{z}$ over the first octant
region bounded by the cylinders $x^2 + y^2 =1$ and $x^2 + y^2 = 4$
and the planes $z = 0, z = 1,$ $x = 0, x = y$.
\end{exa} \begin{solu} The integral is $$\dint _0 ^1  \dint _{\pi /4}
^{\pi /2} \dint _1 ^2 \rho^3\d{\rho}\d{\theta}\d{z} =
\dfrac{15\pi}{16}.
$$
\end{solu}


\begin{exa}\label{exa:3cylinders}
Three long cylinders of radius $R$ intersect at right angles. Find
the volume of their intersection.
\end{exa}
\begin{solu} Let $V$ be the desired volume. By symmetry, $V = 2^4V'$, where
$$V' = \dint _{D'} \d{x} \d{y}  \d{z},$$
$$D' = \{(x, y, z)\in \reals^3: 0 \leq y \leq x, 0 \leq z, x^2 + y^2 \leq R^2, y^2 + z^2 \leq R^2, z^2 + x^2 \leq R^2\}.$$
In this case it is easier to integrate with respect to $z$ first.
Using cylindrical coordinates
$$\Delta' = \left\{(\theta, \rho, z)\in \left[0;\dfrac{\pi}{4}\right]\times [0; R]\times [0; +\infty[,
0 \leq z \leq \sqrt{R^2 - \rho^2\cos^2\theta}\right\}.$$ Now,
$$\begin{array}{lll}
V' & = & \dint _0 ^{\pi/4}\left(\dint _0 ^R \left( \dint _0 ^{\sqrt{R^2
- \rho^2\cos^2\theta}} \d{z}\right)\rho \d{\rho}
\right) \d{\theta} \\
& = & \dint _0 ^{\pi/4}\left(\dint _0 ^R \rho\sqrt{R^2 -
\rho^2\cos^2\theta} \d{\rho}
\right) \d{\theta} \\
& = & \dint _0 ^{\pi/4}-\dfrac{1}{3\cos^2\theta}\left[ (R^2 -
\rho^2\cos^2\theta)^{3/2}
\right] _0 ^R \d{\theta} \\
& =  & \dfrac{R^3}{3}\dint _0 ^{\pi/4} \dfrac{1 - \sin^3\theta}{\cos^2\theta} \d{\theta} \\
& \stackrel{=}{u = \cos\theta} &
\dfrac{R^3}{3}\left([\tan\theta]_0 ^{\pi/4} + \dint _1
^{\frac{\sqrt{2}}{2}} \dfrac{1 - u^2}{u^2}\d{u}\right) \\
& = &  \dfrac{R^3}{3}\left(1 - \left[u^{-1} + u\right] _1
^{\frac{\sqrt{2}}{2}}\right)
\\
& = & \dfrac{\sqrt{2} - 1}{\sqrt{2}}R^3.
\end{array}$$
Finally
$$V = 16V' = 8(2 - \sqrt{2})R^3.$$
\end{solu}



Consider now a change to spherical coordinates
$$ x=\rho\cos\theta\sin\phi, \ \  y=\rho\sin\theta\sin\phi, \ \
z=\rho\cos\phi.
$$ We have

$$\begin{array}{lll} \d{x} & = &
\cos\theta\sin\phi\d{\rho} - \rho\sin\theta\sin\phi\d{\theta} +
\rho\cos\theta\cos\phi\d{\phi},    \\
 \d{y} & = & \sin\theta\sin\phi\d{\rho} +
\rho\cos\theta\sin\phi\d{\theta} + \rho\sin\theta\cos\phi\d{\phi},
\\
  \d{z} & = & \cos\phi\d{\rho} - \rho\sin\phi\d{\phi}.  \end{array}   $$ This
gives
$$\d{x}\wedge\d{y}\wedge\d{z}  = -\rho^2\sin\phi\d{\rho}\wedge\d{\theta}\wedge\d{\phi}.  $$
From this derivation, the form
$\d{\rho}\wedge\d{\theta}\wedge\d{\phi}$ is negatively oriented,
and so we choose
$$\d{x}\wedge\d{y}\wedge\d{z}  = \rho^2\sin\phi\d{\rho}\wedge\d{\phi}\wedge\d{\theta} $$
instead.

\begin{exa}
Let $(a, b, c)\in ]0;+\infty[^3$ be fixed. Find $\dint _R xyz\
\d{V}$ if
$$R = \left\{(x, y, z)\in\reals^3:  \dfrac{x^2}{a^2} + \dfrac{y^2}{b^2} +
\dfrac{z^2}{c^2}
 \leq 1, x \geq 0, y \geq 0, z \geq 0\right\}.$$

\end{exa}\begin{solu}We use spherical coordinates, where $$(x, y, z) = (a\rho\cos\theta\sin\phi,
b\rho\sin\theta\sin\phi, c\rho\cos\phi).$$ We have
$$ \d{x}\wedge\d{y}\wedge\d{z} = abc\rho^2\sin\phi \d{\rho}\wedge\d{\phi}\wedge\d{\rho}. $$ The integration region is mapped into
$$\Delta = [0 ; 1] \times [0 ; \dfrac{\pi}{2}]\times [0; \dfrac{\pi}{2}].$$
The integral becomes
$$ (abc)^2\left(\dint _0 ^{\pi/2}\cos\theta\sin\theta \ \d{\theta}\right)\left(\dint _0 ^{1} \rho^5\ \d{\rho}\right)\left(\dint
_0 ^{\pi/2} \cos^3\phi\sin\phi\ \d{\phi}\right) \\
= \dfrac{(abc)^2}{48}.
$$
\end{solu}
 \begin{exa} Let $V = \{(x,y,z)\in\reals^3: x^2 + y^2 + z^2 \leq 9, 1 \leq z \leq
2\}$. Then  $$\begin{array}{lll} \dint \limits_{V} \d{x}\d{y}\d{z} &
= & \dint _{0} ^{2\pi}\dint _{\pi/2 - \arcsin 2/3} ^{\pi/2 - \arcsin
1/3}   \dint _{1/\cos\phi}
^{2/\cos\phi} \rho^2\sin\phi \ \d{\rho} \d{\phi}\d{\theta}\\
& = & \dfrac{63\pi}{4}. \end{array}$$
\end{exa}


\section*{\psframebox{Homework}}
\begin{pro}
Consider the region ${\cal R}$  below the cone $z=\sqrt{x^2+y^2}$
and above the paraboloid $z=x^2+y^2$ for $0 \leq z \leq 1$. Set up
integrals for the volume of this region in Cartesian, cylindrical
and spherical coordinates. Also, find this volume.
\begin{answer}Cartesian:
$$\dint _{-1} ^1\dint _{-\sqrt{1-y^2}} ^{\sqrt{1-y^2}} \dint
_{x^2+y^2} ^{\sqrt{x^2+y^2}} \d{z}\d{x}\d{y} .$$ Cylindrical:
$$\dint _{0} ^1\dint _{0} ^{2\pi} \dint _{r^2} ^{r}
r\d{z}\d{\theta}\d{r} .$$ Spherical: $$\dint _{\pi/4} ^{\pi/2}\dint
_{0} ^{2\pi} \dint _{0} ^{(\cos \phi)/(\sin \phi)^2} r^2\sin\phi
\d{r}\d{\theta}\d{\phi} . $$ The volume is $\dfrac{\pi}{3}$.
\end{answer}
\end{pro}

\begin{pro}
Consider the integral $\dint _{{\cal R}} x \d{V}$, where ${\cal R}$
is the region above the paraboloid $z=x^2+y^2$ and under the sphere
$x^2+y^2+z^2 =4$. Set up integrals for the volume of this region in
Cartesian, cylindrical and spherical coordinates. Also, find this
volume.
\begin{answer}
\noindent
\begin{enumerate}
\item Since $x^2+y^2\leq z\leq \sqrt{4-x^2-y^2}$, we start our
integration with the $z$-variable. Observe that if $(x,y,z)$ is on
the intersection of the surfaces then  $$ z^2+z=4 \implies
z=\dfrac{-1\pm \sqrt{17}}{2}.
$$Since $x^2+y^2+z^2 =4 \implies -2 \leq z \leq 2$, we must have
$z=\dfrac{\sqrt{17}-1}{2}$ only. The projection of the circle of
intersection of the paraboloid and the sphere onto the $xy$-plane
satisfies the equation
$$z^2+z=4 \implies x^2+y^2+(x^2+y^2)^2=4 \implies x^2+y^2=\dfrac{\sqrt{17}-1}{2},  $$a circle of radius
$\sqrt{\frac{\sqrt{17}-1}{2}}$. The desired integral is thus
$$\dint _{-\sqrt{\frac{\sqrt{17}-1}{2}}} ^{\sqrt{\frac{\sqrt{17}-1}{2}}} \dint _{-\sqrt{\frac{\sqrt{17}-1}{2}-x^2}}  ^{\sqrt{\frac{\sqrt{17}-1}{2}-x^2}} \dint _{x^2+y^2} ^{\sqrt{4-x^2-y^2}} x\d{z}\d{y}\d{x}.  $$
\item  The $z$-limits remain the same as in the Cartesian coordinates, but translated into cylindrical coordinates, and so
$r^2\leq z\leq \sqrt{4-r^2}$. The projection of the intersection
circle onto the $xy$-plane is again a circle with centre at the
origin and radius $\sqrt{\frac{\sqrt{17}-1}{2}}$. The desired
integral
$$\dint _{0} ^{2\pi} \dint _{0}  ^{\sqrt{\frac{\sqrt{17}-1}{2}}} \dint _{r^2} ^{\sqrt{4-r^2}} r^2\cos \theta \d{z}\d{r}\d{\theta}.  $$


\item  Observe that $$ z=x^2+y^2 \implies r\cos \phi=r^2(\cos \theta)^2(\sin\phi)^2+r^2(\sin\theta)^2(\sin\phi)^2 \implies
r \in \{0, (\csc \phi)(\cot \phi)\}. $$It is clear that the limits
of the angle $\theta$ are from $\theta =0$ to $\theta =2\pi$. The
angle $\phi$ starts at $\phi=0$.   Now,
$$z=r\cos\phi \implies \cos \phi = \dfrac{\dfrac{\sqrt{17}-1}{2}}{2} \implies \phi = \arccos \left(\dfrac{\sqrt{17}-1}{4}\right)$$

The desired integral
$$\dint _{0} ^{2\pi} \dint _{0}  ^{\arccos \left(\dfrac{\sqrt{17}-1}{4}\right)} \dint _{(\csc \phi)(\cot \phi)} ^{2} r^3\cos \theta\sin ^2 \phi \d{r}\d{\phi}\d{\theta}.  $$

\end{enumerate}


Perhaps it is easiest to evaluate the integral using cylindrical
coordinates. We obtain
$$\dint _{0} ^{2\pi} \dint _{0}  ^{\sqrt{\dfrac{\sqrt{17}-1}{2}}} \dint _{r^2} ^{\sqrt{4-r^2}} r^2\cos \theta \d{z}\d{r}\d{\theta} = 0,  $$
a conclusion that is easily reached, since the integrand is an odd
function of $x$ and the domain of integration is symmetric about the
origin in $x$.
\end{answer}

\end{pro}
\begin{pro}
Consider the region ${\cal R}$  bounded by the sphere
$x^2+y^2+z^2=4$ and the plane $z=1$. Set up integrals for the volume
of this region in Cartesian, cylindrical and spherical coordinates.
Also, find this volume.
\begin{answer}Cartesian:
$$\dint _{-\sqrt{3}} ^{\sqrt{3}}\dint _{-\sqrt{3-y^2}} ^{\sqrt{3-y^2}} \dint
_{1} ^{\sqrt{4-x^2-y^2}} \d{z}\d{x}\d{y} .$$ Cylindrical:
$$\dint _{0} ^{\sqrt{3}}\dint _{0} ^{2\pi} \dint _{1} ^{\sqrt{4-r^2}}
r\d{z}\d{\theta}\d{r} .$$ Spherical: $$\dint _{0} ^{\pi/3}\dint _{0}
^{2\pi} \dint _{1/\cos\phi} ^{2} r^2\sin\phi \d{r}\d{\theta}\d{\phi}
.
$$ The volume is $\dfrac{5\pi}{3}$.

\end{answer}
\end{pro}
\begin{pro}
Prove that the volume enclosed by the ellipsoid
$$\dfrac{x^2}{a^2}+\dfrac{y^2}{b^2}+\dfrac{z^2}{c^2}=1
$$is $\dfrac{4\pi abc}{3}$. Here $a>0$, $b>0$, $c>0$.
\end{pro}

\begin{pro}
Compute $\dint _E y\d{V}$   where $E$ is the region between the
cylinders  $x^2+y^2=1$ and  \mbox{$x^2+y^2 = 4$}, below the plane
$x-z=-2$ and above the $xy$-plane.
\begin{answer}
We have
$$\dint _E y\d{V} = \dint _0 ^{2\pi}\dint _1 ^2 \dint _0 ^{2+r\cos\theta}r^2\sin\theta \d{z}\d{r}\d{\theta} =0. $$
\end{answer}
\end{pro}
\begin{pro}
Prove that $$ \dint \limits _{\stackrel{x\geq 0, y \geq
0}{x^2+y^2+z^2\leq R^2}} e^{-\sqrt{x^2+y^2+z^2}}\d{V}=
\pi\left(2-2e^{-R}-2Re^{-R}-R^2e^{-R}\right).$$
\end{pro}

\begin{pro}
Compute $\dint _E y^2z^2\d{V}$   where $E$ is bounded by the
paraboloid \mbox{$x=1-y^2-z^2$} and the plane $x=0$.
\begin{answer}
$\dfrac{\pi}{96}$
\end{answer}
\end{pro}

\begin{pro}
Compute $\dint _E z\sqrt{x^2+y^2+z^2}\d{V}$   where $E$ is is the
upper solid hemisphere bounded by the $xy$-plane and the sphere of
radius $1$ about the origin.
\begin{answer}
$\dfrac{\pi}{14}$
\end{answer}
\end{pro}




\begin{pro}
Compute the $4$-dimensional integral
$$\iiiint\limits_{x^2 + y^2 + u^2 + v^2 \leq 1} e^{x^2 + y^2 + u^2 + v^2 } \d{x}\d{y}\d{u}\d{v}.   $$
\begin{answer} We put $$x = \rho\cos \theta\sin\phi\sin t; y
=\rho\sin\theta\sin\phi\sin t; u =\rho\cos \phi\sin t; v = \rho\cos
t.
$$ Upon using $\sin^2a+\cos^2a = 1$ three times,
$$\begin{array}{lll}x^2+y^2+u^2+v^2 & =
&r^2\cos^2\theta\sin^2\phi\sin^2 t+r^2\sin^2\theta\sin^2\phi\sin^2t
+ r^2\cos^2\phi\sin^2 t + r^2\cos^2 t \\
& = &r^2\cos^2\theta\sin^2\phi+r^2\sin^2\theta\sin^2\phi
+ r^2\cos^2\phi \\
& = &r^2\cos^2\theta+r^2\sin^2\theta \\
& = & r^2.
\end{array}$$
Now,
$$\begin{array}{lll}\d{x} & = & \cos\theta\sin\phi\sin t\d{r} -\rho\sin\theta\sin\phi\sin t\d{\theta} + \rho\cos \theta\cos\phi\sin t\d{\phi}
+ \rho\cos \theta\sin\phi\cos t\d{t} \\
\d{y} & = & \sin\theta\sin\phi\sin t\d{r} +\rho\cos
\theta\sin\phi\sin t\d{\theta} + \rho\sin\theta\cos\phi\sin
t\d{\phi} +
\rho\sin\theta\sin\phi\cos t\d{t} \\
\d{u} & = & \cos\phi\sin t\d{r} -\rho\sin\phi\sin t\d{\phi} + \rho\cos \phi\cos t\d{t}\\
\d{v}  & = & \cos t\d{r} - \rho\sin t\d{t}   \end{array}.$$After
some calculation,
$$ \d{x}\wedge\d{y}\wedge\d{u}\wedge\d{v} = r^3\sin\phi\sin^2t\d{r}\wedge\d{\phi}\wedge\d{\theta}\wedge\d{t}. $$
Therefore
$$\begin{array}{lll}\iiiint\limits_{x^2 + y^2 + u^2 + v^2 \leq 1} e^{x^2 + y^2 + u^2 + v^2 } \d{x}\d{y}\d{u}\d{v} & = &
\dint _{0} ^{\pi}\dint _0 ^{2\pi}\dint _0 ^\pi\dint _0 ^1
r^3e^{r^2}\sin\phi\sin^2t\d{r}\d{\phi}\d{\theta}\d{t} \\ & = &
\left(\dint _{0} ^{1} r^3e^{r^2}\d{r}\right)\left(\dint _{0} ^{2\pi}
\d{\theta}\right)\left(\dint _{0} ^{\pi}\sin\phi
\d{\phi}\right)\left(\dint _{0} ^{\pi}\sin^2t
\d{t}\right)\\
& = &
\left(\dfrac{1}{2}\right)(2\pi)(2)\left(\dfrac{\pi}{2}\right) \\
& = & \pi^2.
\end{array}$$

\end{answer}
\end{pro}

\begin{pro}[Putnam Exam 1984] Find $$\dint \limits_R x^1y^9z^8(1 -x-y-z)^4\  \d{x}\d{y}\d{z},   $$
where $$R = \{(x,y,z)\in\reals^3: x \geq 0, y \geq 0, z \geq 0, x +
y + z \leq 1\}.
$$

\begin{answer} We make the change of variables $$u = x + y + z \implies \d{u}
= \d{x} + \d{y} + \d{z},
$$
$$uv = y + z \implies u\d{v} + v\d{u} = \d{y} + \d{z},  $$
$$uvw = z \implies uv\d{w} + uw\d{v} + vw\d{u} = \d{z}.  $$This gives
$$x = u(1- v),     $$
$$y = uv(1- w),$$
$$z = uvw,  $$
$$u^2v\ \d{u}\wedge\d{v}\wedge\d{w}
=\d{x}\wedge\d{y}\wedge\d{z}.$$ To find the limits of integration we
observe that the limits of integration using $\d{x}\wedge \d{y}
\wedge \d{z}$ are
$$ 0 \leq z \leq 1, \ 0 \leq y \leq 1 - z, \ 0 \leq x \leq 1 - y - z.   $$
This translates into
$$ 0 \leq uvw \leq 1, \ 0 \leq uv-uvw \leq 1 - uvw, \ 0\leq u-uv \leq 1 - uv+uvw - uvw.   $$
Thus
$$ 0 \leq uvw \leq 1, \ 0 \leq uv \leq 1 , \ 0\leq u\leq 1 ,
$$which finally give
$$ 0 \leq u \leq 1, \ 0 \leq v \leq 1 , \ 0\leq w\leq 1.
$$
The integral sought is then, using the fact that for positive
integers $m, n$ one has $$\dint _0 ^1 x^m(1-x)^n\ \d{x} =
\dfrac{m!n!}{(m+n+1)!},   $$ we deduce,
$$\dint _0 ^1\dint _0 ^1\dint _0 ^1
u^{20}v^{18}w^8(1-u)^4(1-v)(1-w)^9 \ \d{u}\d{v}\d{w},
$$which in turn is
$$
\left(\dint_0^1 u^{20}(1-u)^4\d{u}\right)\left(\dint_0^1
v^{18}(1-v)\d{v}\right)\left(\dint_0^1 w^8(1-w)^9\d{w}\right) =
\frac{1}{265650}\cdot\frac{1}{380}\cdot\frac{1}{437580} $$which is
$$ =  \frac{1}{44172388260000}.$$
\end{answer}
\end{pro}

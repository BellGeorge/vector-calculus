\begin{Answer}{2.2}
 Since polynomials are continuous
functions and the image of a connected set is connected for a
continuous function, the image must be an interval of some sort. If
the image were a finite interval, then $f(x,kx)$ would be bounded
for every constant $k$, and so the image would just be the point
$f(0,0).$ The possibilities are thus
\begin{enumerate}\item  a single point (take for example,
$p(x,y)= 0$), \item  a semi-infinite interval with an endpoint (take
for example $p(x,y) = x^2$ whose image is $[0; +\infty[$),
\item  a semi-infinite interval with no endpoint (take for example
$p(x,y) = (xy - 1)^2 + x^2$ whose image is $]0; +\infty[$),
\item  all real numbers (take for example $p(x,y) =
x$).
\end{enumerate}
\end{Answer}
\begin{Answer}{2.8}
 $0$ 
\end{Answer}
\begin{Answer}{2.9}
 $2$ 
\end{Answer}
\begin{Answer}{2.10}
$c=0$.
\end{Answer}
\begin{Answer}{2.11}
 $0$ 
\end{Answer}
\begin{Answer}{2.14}
By AM-GM, $$\dfrac{x^2y^2z^2}{x^2+y^2+z^2} \leq
\dfrac{(x^2+y^2+z^2)^3}{27(x^2+y^2+z^2)} =
\dfrac{(x^2+y^2+z^2)^2}{27}\to 0$$ as $(x,y,z)\to (0,0,0)$.
\end{Answer}
\begin{Answer}{2.21}
 $0$ 
\end{Answer}
\begin{Answer}{2.22}
 $2$ 
\end{Answer}
\begin{Answer}{2.23}
$c=0$.
\end{Answer}
\begin{Answer}{2.24}
 $0$ 
\end{Answer}
\begin{Answer}{2.27}
By AM-GM, $$\dfrac{x^2y^2z^2}{x^2+y^2+z^2} \leq
\dfrac{(x^2+y^2+z^2)^3}{27(x^2+y^2+z^2)} =
\dfrac{(x^2+y^2+z^2)^2}{27}\to 0$$ as $(x,y,z)\to (0,0,0)$.
\end{Answer}
\begin{Answer}{3.4}
 We have
$$\renewcommand{\arraystretch}{1.7}
{\everymath{\displaystyle}\begin{array}{lll} F(\vector{x} +
\vector{h})
- F(\vector{x}) & = & (\vector{x} + \vector{h})\cross L(\vector{x} + \vector{h}) - \vector{x}\cross L(\vector{x})\\
& = & (\vector{x} + \vector{h})\cross (L(\vector{x}) +
L(\vector{h})) - \vector{x}\cross L(\vector{x})\\
& = & \vector{x}\cross L(\vector{h}) + \vector{h}\cross
L(\vector{x}) + \vector{h}\cross
L(\vector{h}) \\
\end{array}}
$$\renewcommand{\arraystretch}{1}
Now, we will prove that  $|| \vector{h}\cross L(\vector{h})|| =
\smallosans{\norm{\vector{h}}}$ as $\vector{h} \rightarrow
\vector{0}$. For let $$\vector{h} = \sum_{k = 1} ^n h_k
\vector{e}_k,$$where the $\vector{e}_k$ are the standard basis for
$\reals^n$. Then
$$L(\vector{h}) = \sum_{k = 1} ^n h_k L(\vector{e}_k),$$and hence
by the triangle inequality, and by the Cauchy-Bunyakovsky-Schwarz
inequality,
$$\renewcommand{\arraystretch}{1.7}\begin{array}{ccc}||L(\vector{h})|| & \leq & \sum_{k = 1} ^n |h_k| ||L(\vector{e}_k)|| \\ &
\leq&  \left(\sum_{k = 1} ^n |h_k|^2\right)^{1/2}\left(\sum_{k = 1}
^n ||L(\vector{e}_k)||^2\right)^{1/2} \\ &  = &
\norm{\vector{h}}(\sum_{k = 1} ^n
||L(\vector{e}_k)||^2)^{1/2}, \\
\end{array}$$whence, again by the  Cauchy-Bunyakovsky-Schwarz
Inequality, $$|| \vector{h}\cross L(\vector{h})|| \leq||
\vector{h}||||L(\vector{h})|\leq
||\vector{h}||^2|||L(\vector{e}_k)||^2)^{1/2} =
\smallosans{||\vector{h}||},$$ as it was to be shewn.
\end{Answer}
\begin{Answer}{3.5}
 Assume that $\vector{x} \neq
\vector{0}.$ We use the fact that $(1 + t)^{1/2} = 1 + \dfrac{t}{2}
+ \smallosans{t}$ as $t \rightarrow 0$. We have
$$\renewcommand{\arraystretch}{1.7}
{\everymath{\displaystyle}\begin{array}{lll} \funvect{f}(\vector{x} +
\vector{h}) - \funvect{f}(\vector{x}) & = & ||\vector{x}+\vector{h}|| - \norm{\vector{x}}  \\
& = & \sqrt{(\vector{x} + \vector{h})\bp (\vector{x} + \vector{h})} - \norm{\vector{x}} \\
& = & \sqrt{\norm{\vector{x}}^2 + 2\vector{x}\bp\vector{h}+
\norm{\vector{h}}^2} - \norm{\vector{x}}
\\
& = & \dfrac{2\dotprod{x}{h}+
\norm{\vector{h}}^2}{\sqrt{\norm{\vector{x}}^2 +
2\vector{x}\bp\vector{h}+ \norm{\vector{h}}^2}  + \norm{\vector{x}}}.\\
\end{array}}$$\renewcommand{\arraystretch}{1}
As $\vector{h} \rightarrow \vector{0}$,
$$\sqrt{\norm{\vector{x}}^2 +
2\vector{x}\bp\vector{h}+ \norm{\vector{h}}^2}  + \norm{\vector{x}}
\rightarrow 2\norm{\vector{x}}.$$Since $\norm{\vector{h}}^2 =
\smallosans{\norm{\vector{h}}}$ as $\vector{h} \rightarrow
\vector{0}$, we have
$$\dfrac{2\dotprod{x}{h}+ \norm{\vector{h}}^2}{\sqrt{\norm{\vector{x}}^2 +
2\vector{x}\bp\vector{h}+ \norm{\vector{h}}^2}  + \norm{\vector{x}}}
\rightarrow \dfrac{\vector{x}\bp\vector{h}}{\norm{\vector{h}}} +
\smallosans{\norm{\vector{h}}},$$proving the first assertion.



      To prove the second assertion, assume that there is a linear
transformation $\deriv{0}{f} = L$,   $L:\reals^n \rightarrow \reals$
such that
$$||\funvect{f}(\vector{0} + \vector{h}) - \funvect{f}(\vector{0}) - L(\vector{h})|| = \smallosans{\norm{\vector{h}}},$$as $\norm{\vector{h}}\rightarrow
0$.  Recall that by theorem \ref{thm:lipschitzlin}, $L(\vector{0}) =
\vector{0}$, and so by example \ref{exa:derivlintran},
$\deriv{0}{L}(\vector{0}) = L(\vector{0}) = \vector{0}$. This
implies that $\dis{\dfrac{L(\vector{h})}{\norm{\vector{h}}}}
\rightarrow \deriv{0}{L}(\vector{0}) = \vector{0}$, as
$\norm{\vector{h}}\rightarrow 0$. Since $\funvect{f}(\vector{0}) = \norm{0} =
0, \funvect{f}(\vector{h}) = \norm{\vector{h}}$ this would imply that
$$\left|\left|\norm{\vector{h}} - L(\vector{h})\right|\right| = \smallosans{\norm{\vector{h}}},$$or
$$\left|\left| 1 -  \dfrac{L(\vector{h})}{\norm{\vector{h}}}\right|\right| = \smallosans{1}.$$
But the left side $\rightarrow 1$ as $\vector{h} \rightarrow
\vector{0}$, and the right side $\rightarrow 0$ as $\vector{h}
\rightarrow \vector{0}$. This is a contradiction, and so, such
linear transformation $L$ does not exist at the point $\vector{0}$.
\end{Answer}
\begin{Answer}{3.7}
Observe that
$$\funvect{f}(x,y)=\left\{\begin{array}{ll} x & \mathrm{if}\ x \leq y^2 \\   y^2 & \mathrm{if}\ x > y^2 \\ \end{array}\right. $$
Hence
$$\dfrac{\partial}{\partial x}\funvect{f}(x,y)=\left\{\begin{array}{ll} 1 & \mathrm{if}\ x > y^2 \\   0 & \mathrm{if}\ x > y^2 \\ \end{array}  \right. $$ and
$$\dfrac{\partial}{\partial y}\funvect{f}(x,y)=\left\{\begin{array}{ll} 0 & \mathrm{if}\ x > y^2 \\   2y & \mathrm{if}\ x > y^2 \\ \end{array}\right. $$
\end{Answer}
\begin{Answer}{3.8}
Observe that
$$\funvect{g}(1,0,1) = \coord{3 \\ 0}, \qquad f'(x,y) = \begin{bmatrix} y^2 &
2xy \cr 2xy & x^2
\end{bmatrix}, \qquad g'(x,y) =\begin{bmatrix} 1 & -1 & 2  \cr y & x & 0
\cr
\end{bmatrix},  $$
and hence
$$ g'(1,0,1) = \begin{bmatrix} 1 & -1 & 2  \cr 0 & 1 & 0
\cr
\end{bmatrix},\qquad  f'(\funvect{g}(1,0,1)) =  f'(3,0) = \begin{bmatrix} 0  &  0  \cr 0 & 9
\cr
\end{bmatrix}. $$This gives, via the Chain-Rule,
$$(f\circ g)'(1,0,1)  = f'(\funvect{g}(1,0,1))g'(1,0,1) = \begin{bmatrix} 0  &  0  \cr 0 & 9
\cr
\end{bmatrix}\begin{bmatrix} 1 & -1 & 2  \cr 0 & 1 & 0
\cr
\end{bmatrix} = \begin{bmatrix} 0 & 0 & 0  \cr 0 & 9 & 0
\cr
\end{bmatrix}. $$
The composition $g\circ \funvect{f}$ is undefined. For, the output of $\funvect{f}$ is
$\reals^2$, but the input of $\funvect{g}$ is in $\reals^3$.

\end{Answer}
\begin{Answer}{3.9}
Since $\funvect{f}(0,1) = \coord{0\\ 1}$, the Chain Rule gives
$$(g\circ f)'(0,1) = (g'(\funvect{f}(0,1)))(f'(0,1)) = (g'(0,1))(f'(0,1)) =\begin{bmatrix} 1 & -1\\ 0 &
0 \\ 1 & 1 \\
\end{bmatrix}\begin{bmatrix} 1 & 0\\ 1 & 1
\end{bmatrix} = \begin{bmatrix} 0 & -1\\ 0 & 0 \\ 2 & 1 \\
\end{bmatrix}   $$

\end{Answer}
\begin{Answer}{3.13}
We have
$$  \dfrac{\partial }{\partial x}(x+z)^2+\dfrac{\partial }{\partial x}(y+z)^2= \dfrac{\partial }{\partial x}8
\implies 2(1+ \dfrac{\partial z}{\partial x})(x+z) +
2\dfrac{\partial z}{\partial x}(y+z)=0.  $$ At $(1,1,1)$ the last
equation becomes
$$ 4(1+ \dfrac{\partial z}{\partial x}) +
4\dfrac{\partial z}{\partial x}=0\implies \dfrac{\partial
z}{\partial x} = -\dfrac{1}{2}.  $$
\end{Answer}
\begin{Answer}{3.14}
a) Here $\nabla T = (y+z) \vector{ i} + (x+z) \vector{ j} + (y+x) \vector{ k}$.
The maximum rate of change at $(1,1,1)$ is
$|\nabla T(1,1,1)| = 2 \sqrt{3}$ and direction cosines are
\begin{eqnarray*}
\dfrac{\nabla T}{|\nabla T|} =
\dfrac{1}{\sqrt{3}} \vector{ i} +
\dfrac{1}{\sqrt{3}} \vector{ j} +
\dfrac{1}{\sqrt{3}} \vector{ k} =
\cos \alpha \vector{ i} + \cos \beta \vector{ j} + \cos \gamma \vector{ k}
\end{eqnarray*}


b) The required derivative is
\begin{eqnarray*}
\nabla T(1,1,1) \bp
\dfrac{3 \vector{ i} - 4 \vector{ k}}{|3 \vector{ i} - 4 \vector{ k}|} =
- \dfrac{2}{5}
\end{eqnarray*}

\end{Answer}
\begin{Answer}{3.15}

a) Here $\nabla \phi = \vector{ F}$ requires $\nabla \times \vector{ F} = 0$ which is
not the case here, so no solution.


b) Here $\nabla \times \vector{ F} = 0$ so that
\begin{eqnarray*}
\phi(x,y,z) = x^{2} y + y^{2} z + z + c
\end{eqnarray*}

\end{Answer}
\begin{Answer}{3.16}
$\nabla f(x,y,z)=\coord{e^{yz} ,xze^{yz},xye^{yz}}\implies
(\nabla f)(2,1,1) =  \coord{e,2e,2e} $.
\end{Answer}
\begin{Answer}{3.17}
$(\nabla \cross f)(x,y,z)=\coord{0,x, ye^{xy}} \implies (\nabla
\cross f)(2,1,1)=\coord{0,2 ,e^2}$.
\end{Answer}
\begin{Answer}{3.19}
 The vector $\coord{1 ,-7 ,0}$ is perpendicular to the plane. Put $\funvect{f}(x,y,z)=x^2+y^2-5xy+xz-yz+3$. Then $(\nabla f)(x,y,z)=\coord{2x-5y+z ,2y-5x-z \\
x-y}$. Observe that $\nabla f(x,y,z)$ is parallel to the vector
$\coord{1 ,-7 ,0}$, and hence there exists a constant $a$ such
that $$ \coord{2x-5y+z ,2y-5x-z \\
x-y}=a\coord{1 ,-7 ,0} \implies x=a, \quad y=a, \quad z=4a.$$
Since the point is on the plane $$x-7y=-6 \implies a-7a=-6 \implies
a=1.$$ Thus $x=y=1$ and $z=4$.
\end{Answer}
\begin{Answer}{3.22}
Observe that $$\funvect{f}(0,0) =1, \quad f_x(x,y) = (\cos 2y)e^{x\cos 2y}
\implies f_x(0,0)=1, $$ $$ f_y(x,y) = -2x\sin 2ye^{x\cos 2y}
\implies f_y(0,0) = 0.
$$Hence
$$\funvect{f}(x,y) \approx \funvect{f}(0,0) + f_x(0,0)(x-0) + f_y(0,0)(y-0) \implies \funvect{f}(x,y)\approx 1 + x. $$
This gives $\funvect{f}(0.1,-0.2)\approx 1+0.1 = 1.1$.
\end{Answer}
\begin{Answer}{3.23}
This is essentially the product rule: $\d{uv} = u\d{v}+v\d{u}$,
where $\nabla$ acts the differential operator and $\cross$ is the
product. Recall that when we defined the volume of a parallelepiped
spanned by the vectors $\vector{a},$ $\vector{b}$, $\vector{c}$, we
saw that
$$ \vector{a}\bullet (\crossprod{b}{c}) = (\crossprod{a}{b})\bullet \vector{c}.  $$
Treating $\nabla = \nabla _{\vector{u}}+ \nabla _{\vector{v}}$ as a
vector, first keeping $\vector{v}$ constant and then keeping
$\vector{u}$ constant  we then see that
$$ \nabla _{\vector{u}} \bullet (\crossprod{u}{v})= (\crossprod{\nabla}{u})\bullet \vector{v}, \qquad   \nabla  _{\vector{v}} \bullet (\crossprod{u}{v})=  -\nabla \bullet (\crossprod{v}{u})  = -(\crossprod{\nabla}{v})\bullet \vector{u}. $$
Thus
$$\nabla \bullet (\point{u}\cross \point{v}) = (\nabla _{\vector{u}} + \nabla _{\vector{v}}) \bullet (\point{u}\cross \point{v})
= \nabla _{\vector{u}} \bullet (\crossprod{u}{v}) + \nabla
_{\vector{v}} \bullet (\crossprod{u}{v}) =
(\crossprod{\nabla}{u})\bullet \vector{v}-
(\crossprod{\nabla}{v})\bullet \vector{u}.
   $$
\end{Answer}
\begin{Answer}{3.26}
An angle of $ \dfrac{\pi}{6}$ with the $x$-axis and $
\dfrac{\pi}{3}$ with the $y$-axis.
\end{Answer}
\begin{Answer}{5.1}
 Let $\colpoint{x\\ y\\ z}$ be a point on $S$. If this point were on
the $xz$ plane, it would be on the ellipse, and its distance to the
axis of rotation would be $|x| = \dfrac{1}{2}\sqrt{1 -
z^2}$. Anywhere else, the distance from $\colpoint{x\\ y\\
z}$ to the $z$-axis  is the distance of this point to the point
$\colpoint{0\\ 0\\ z}$ : $\sqrt{x^2 + y^2}$. This distance is the
same as the length of the segment on the $xz$-plane going from the
$z$-axis. We thus have
$$ \sqrt{x^2 + y^2} = \dfrac{1}{2}\sqrt{1 -
z^2},$$or
$$4x^2 + 4y^2 + z^2 = 1.$$
\end{Answer}
\begin{Answer}{5.2}
 Let $\colpoint{x\\ y\\ z}$ be a point on $S$. If this point were on the
$xy$ plane, it would be on the line, and its distance to the axis of
rotation would be $|x| = \dfrac{1}{3}|1 - 4y|$. Anywhere else, the
distance of $\colpoint{x\\ y\\ z}$ to the axis of rotation is the
same as the distance of $\colpoint{x\\ y\\ z}$ to $\colpoint{0\\
y \\ 0}$, that is $\sqrt{x^2 + z^2}$. We must have
$$\sqrt{x^2 + z^2} = \dfrac{1}{3}|1 - 4y|, $$which is to say
$$9x^2 + 9z^2 - 16y^2 + 8y - 1 = 0.$$
\end{Answer}
\begin{Answer}{5.3}
A spiral staircase.
\end{Answer}
\begin{Answer}{5.4}
A spiral staircase.
\end{Answer}
\begin{Answer}{5.6}
 The planes $A: x + z = 0$ and $B: y = 0$ are secant. The
surface has equation of the form $f(A, B) = e^{A^2 + B^2} - A = 0$,
and it is thus a cylinder.  The directrix has direction $\vector{i}
- \vector{k}$.
\end{Answer}
\begin{Answer}{5.7}
 Rearranging,
$$ (x^2 + y^2 + z^2)^2 - \dfrac{1}{2}((x + y + z)^2 - (x^2 + y^2 + z^2)) - 1 = 0,$$
and so we may take $A: x + y + z = 0, \superficie : x^2 + y^2 + z^2 = 0$,
shewing that the surface is of revolution. Its axis is the line in
the direction $\vector{i} + \vector{j} + \vector{k}$.
\end{Answer}
\begin{Answer}{5.8}
 Considering the planes $A: x - y = 0, B: y - z = 0$,
the equation takes the form
$$f(A, B) = \dfrac{1}{A} + \dfrac{1}{B} - \dfrac{1}{A + B} - 1 =
0,$$thus the equation represents a cylinder. To find its directrix,
we find the intersection of the planes $x = y$ and $y = z$. This
gives $\colvec{x \\ y \\ z} = t\colvec{1 \\ 1 \\ 1}$. The direction
vector is thus $\vector{i}+ \vector{j} + \vector{k}$.
\end{Answer}
\begin{Answer}{5.9}
 Rearranging,
$$(x + y + z)^2 - (x ^2 + y ^2 + z ^2) + 2(x + y + z) + 2 = 0,$$so we
may take $A: x + y + z = 0, \superficie : x ^2 + y ^2 + z ^2 = 0$ as our
plane and sphere. The axis of revolution is then in the direction of
$\vector{i} + \vector{j} + \vector{k}$.
\end{Answer}
\begin{Answer}{5.10}
 After rearranging, we obtain
$$(z - 1)^2 - xy = 0,$$or
$$-\dfrac{x}{z - 1}\dfrac{y}{z - 1} + 1 = 0.$$Considering the planes
$$A: x = 0, \ B: y = 0, \ \ C: z = 1,$$we see that our surface is
a cone, with apex at $(0,0,1)$.
\end{Answer}
\begin{Answer}{5.11}
 The largest circle has radius $b$. Parallel cross sections of
the ellipsoid are similar ellipses, hence we may increase the size
of these by moving towards the centre of the ellipse.  Every plane
through the origin which makes a circular cross section must
intersect the $yz$-plane, and the diameter of any such cross section
must be a diameter of the ellipse $x = 0, \dfrac{y^2}{b^2} +
\dfrac{z^2}{c^2} = 1.$ Therefore, the radius of the circle is at most
$b$. Arguing similarly on the $xy$-plane shews that the radius of
the circle is at least $b$. To shew that circular cross section of
radius $b$ actually exist, one may verify that the two planes given
by $a^2(b^2 - c^2)z^2 = c^2(a^2 - b^2)x^2$ give circular cross
sections of radius $b$.
\end{Answer}
\begin{Answer}{5.12}
Any hyperboloid oriented like the one on the figure has an equation
of the form
$$ \dfrac{z^2}{c^2}=\dfrac{x^2}{a^2}+\dfrac{y^2}{b^2}-1. $$ When $z=0$ we must have
$$4x^2+y^2=1 \implies a=\dfrac{1}{2}, \ b=1.   $$
Thus
$$ \dfrac{z^2}{c^2}= 4x^2+y^2-1. $$Hence, letting $z=\pm 2$,
$$  \dfrac{4}{c^2} =4x^2+y^2-1 \implies \dfrac{1}{c^2}= x^2+\dfrac{y^2}{4}-\dfrac{1}{4}=1-\dfrac{1}{4}=\dfrac{3}{4},$$
since at $z=\pm 2$, $x^2+\dfrac{y^2}{4}=1$. The equation is thus
$$ \dfrac{3z^2}{4}=  4x^2+y^2-1.$$
\end{Answer}
\begin{Answer}{13.1}
\noindent
\begin{enumerate}
\item  Let $L_1: y =x+1$, $L_2: -x+1$. Then
$$
\begin{array}{lll}
\dint _C x\d{x}+y\d{y} & = & \dint _{L_1} x\d{x}+y\d{y}+\dint _{L_2}
x\d{x}+y\d{y}\\
& = & \dint _{-1} ^1 x\d{x} (x+1)\d{x} + \dint _0 ^1 x\d{x}
-(-x+1)\d{x}\\
& = & 0.
\end{array}
$$
Also, both on $L_1$ and on $L_2$ we have
$\norm{\d{\point{x}}}=\sqrt{2}\d{x}$, thus
$$
\begin{array}{lll}
\dint _C xy \norm{\d{\point{x}}}& = & \dint _{L_1}xy
\norm{\d{\point{x}}}+\dint _{L_2}
xy \norm{\d{\point{x}}}\\
& = & \sqrt{2}\dint _{-1} ^1 x(x+1)\d{x} - \sqrt{2}\dint _0 ^1 x(-x+1)\d{x}\\
& = & 0.
\end{array}
$$

\item We put $x=\sin t$, $y = \cos t$, $t\in\lcrc{-\frac{\pi}{2}}{\frac{\pi}{2}}$. Then

$$
\begin{array}{lll}
\dint _C x\d{x}+y\d{y} & = & \dint _{-\pi/2} ^{\pi/2}
(\sin t)(\cos t)\d{t}-(\cos t)(\sin t)\d{t}\\
& = & 0.
\end{array}
$$
Also, $\norm{\d{\point{x}}}=\sqrt{(\cos t)^2+(-\sin
t)^2}\d{t}=\d{t}$, and thus
$$
\begin{array}{lll}
\dint _C xy \norm{\d{\point{x}}}& = & \dint _{-\pi/2} ^{\pi/2} (\sin
t)(\cos t) \d{t}\\
& = & \dfrac{(\sin t)^2}{2}\Big| _{-\pi/2} ^{\pi/2} \\
& = & 0.
\end{array}
$$
\end{enumerate}
\end{Answer}
\begin{Answer}{13.2}
Let $\Gamma _1$ denote the straight line segment path from $O$ to
$A=(2\sqrt{3},2)$ and $\Gamma _2$ denote the arc of the circle
centred at $(0,0)$ and radius $4$ going counterclockwise from
$\theta=\dfrac{\pi}{6}$ to  $\theta=\dfrac{\pi}{5}$.

\bigskip

Observe that the Cartesian equation of the line  $\line{OA}$ is $y
=\dfrac{x}{\sqrt{3}}$. Then on $\Gamma _1$
$$x\d{x} + y\d{y} = x\d{x}+ \dfrac{x}{\sqrt{3}}\d{\dfrac{x}{\sqrt{3}}} = \dfrac{4}{3}x\d{x}.$$
Hence $$\dint _{\Gamma _1} x\d{x} + y\d{y} = \dint _0 ^{2\sqrt{3}}
\dfrac{4}{3}x\d{x} = 8.
$$

On the arc of the circle we may put $x=4\cos \theta$, $y = 4\sin
\theta$ and integrate from $\theta = \dfrac{\pi}{6}$ to $\theta =
\dfrac{\pi}{5}$. Observe that there
$$ x\d{x} + y\d{y} = (\cos \theta )\d{\cos \theta}  +(\sin\theta)\d{\sin \theta} = -\sin\theta\cos\theta\d{\theta}+\sin\theta\cos\theta\d{\theta}=0,$$
and since the integrand is $0$, the integral will be zero.

\bigskip

Assembling these two pieces,
$$\dint _{\Gamma} x\d{x} + y\d{y} = \dint _{\Gamma _1} x\d{x} + y\d{y} +\dint _{\Gamma _2} x\d{x} + y\d{y}=8+0=8.    $$


\bigskip

Using the parametrisations from the solution of  problem
\ref{pro:path-integral2}, we find on $\Gamma _1$ that
$$ x\norm{\d{\point{x}}} = x\sqrt{(\d{x})^2+ (\d{y})^2} =x\sqrt{1+\dfrac{1}{3}}\d{x}=\dfrac{2}{\sqrt{3}}x\d{x},  $$
whence
$$ \dint _{\Gamma _1} x\norm{\d{\point{x}}} = \dint _0 ^{2\sqrt{3}}
\dfrac{2}{\sqrt{3}}x\d{x} = 4\sqrt{3}.  $$ On $\Gamma _2$ that
$$ x\norm{\d{\point{x}}} = x\sqrt{(\d{x})^2+ (\d{y})^2} =16\cos \theta\sqrt{\sin^2\theta + \cos^2\theta}\d{\theta}=16\cos\theta\d{\theta},  $$
whence
$$ \dint _{\Gamma _2} x\norm{\d{\point{x}}} = \dint _{\pi /6} ^{\pi /5}
16\cos\theta\d{\theta} = 16\sin \dfrac{\pi}{5} -16\sin
\dfrac{\pi}{6}= 4\sin \dfrac{\pi}{5}-8.
$$Assembling these we gather that
$$ \dint _{\Gamma } x\norm{\d{\point{x}}} =\dint _{\Gamma _1} x\norm{\d{\point{x}}}+\dint _{\Gamma _2}
x\norm{\d{\point{x}}}= 4\sqrt{3}-8+16\sin \dfrac{\pi}{5}.$$



\end{Answer}
\begin{Answer}{13.3}
 The
curve lies on the sphere, and to parametrise this curve, we dispose
of one of the variables, $y$ say, from where $y = 1-x$ and $x^2 +
y^2 + z^2 = 1$ give
$$\begin{array}{lll}x^2 + (1 - x)^2 + z^2 = 1  & \implies & 2x^2 -2x + z^2 = 0\\ & \implies & 2\left(x - \frac{1}{2}\right)^2   + z^2 = \frac{1}{2}\\
& \implies & 4\left(x - \frac{1}{2}\right)^2 + 2z^2 =
1.\end{array}$$ So we now put $$x = \frac{1}{2} +  \frac{\cos t}{2},
\ \ \ z = \dfrac{\sin t}{\sqrt{2}},\ \ \  y = 1 - x =
\frac{1}{2}-\frac{\cos t}{2}.
$$We must integrate on the side of the plane that can be viewed
from   the point $(1,1,0)$ (observe that the vector  $\colvec{1\\ 1 \\
0}$ is normal to the plane). On the $zx$-plane, $4\left(x -
\frac{1}{2}\right)^2 + 2z^2 = 1$ is an ellipse. To obtain a positive
parametrisation we must integrate from $t = 2\pi$ to $t = 0$ (this
is because when you look at the ellipse from the point $(1,1,0)$ the
positive $x$-axis is to your left, and not your right).   Thus
$$\begin{array}{lll}\doint _\Gamma  z\d{x} + x\d{y} + y\d{z} & = &
\dint ^0 _{2\pi} \dfrac{\sin
t}{\sqrt{2}} \d{\left( \frac{1}{2} + \frac{\cos t}{2}\right)}  \\
& & + \dint ^0 _{2\pi} \left( \frac{1}{2} + \frac{\cos
t}{2}\right)\d{\left(\frac{1}{2} -\frac{\cos t}{2}\right)} \\ & & +
\dint ^0 _{2\pi} \left(\frac{1}{2} -\frac{\cos
t}{2}\right)\d{\left(\dfrac{\sin t}{\sqrt{2}}\right)} \\
& = & \dint ^0 _{2\pi} \left(\frac{\sin t}{4}+ \frac{\cos
t}{2\sqrt{2}}+ \frac{\cos t\sin t}{4}  -
\frac{1}{2\sqrt{2}}\right)\ \d{t}  \\
& = & \dfrac{\pi}{\sqrt{2}}.
\end{array}$$

\end{Answer}
\begin{Answer}{13.4}
 We parametrise the surface by letting $x = u, y =
v, z = u + v^2.$ Observe that the domain $D$ of $\Sigma$ is the
square $[0; 1]\times [0; 2]$. Observe that
$$\d{x} \wedge \d{y}
=   \d{u} \wedge \d{v} ,$$
$$\d{y} \wedge \d{z} = -\d{u} \wedge \d{v}, $$
$$\d{z} \wedge \d{x} = -2v\d{u} \wedge \d{v},
$$and so
$$\norm{\d{ ^2\vector{x}}} = \sqrt{2 + 4v^2}\d{u} \wedge \d{v}.$$
The integral becomes
$$\begin{array}{lll}
\dint \limits_\Sigma y \norm{\d{ ^2\vector{x}}} & = & \dint_0 ^2\dint _0
^1
v\sqrt{2 + 4v^2}\d{u}\d{v} \\
& = & \left(\dint_0 ^1 \d{u}\right) \left(\dint_0 ^2 y\sqrt{2 +
4v^2}\d{v}\right) \\
& = & \dfrac{13\sqrt{2}}{3}.
\end{array}$$
\end{Answer}
\begin{Answer}{13.5}
Using $x=r\cos \theta$, $y=r\sin \theta$, $1\leq r \leq 2$, $0 \leq
\theta \leq 2\pi$, the surface area is
$$ \sqrt{2}\dint _0  ^{2\pi}\dint _1 ^2 r\d{r}\d{\theta}=3\pi\sqrt{2}.  $$
\end{Answer}
\begin{Answer}{13.6}
 We use spherical coordinates, $(x, y, z) =
(\cos\theta\sin\phi, \sin\theta\sin\phi, \cos\phi)$. Here $\theta
\in [0; 2\pi]$ is the latitude and $\phi \in [0; \pi]$ is the
longitude. Observe that
$$\d{x}\wedge\d{y}=
\sin\phi\cos\phi \d{\phi}\wedge\d{\theta} ,$$
$$\d{y}\wedge\d{z} = \cos\theta\sin^2\phi \d{\phi}\wedge\d{\theta},
$$
$$\d{z}\wedge\d{x}= -\sin\theta\sin^2\phi \d{\phi}\wedge\d{\theta},
$$and so
$$\norm{\d{ ^2 \vector{x}}} = \sin\phi \d{\phi}\wedge\d{\theta}.$$
The integral becomes
$$\begin{array}{lll}
\dint \limits_\Sigma x^2 \norm{\d{ ^2\vector{x}}} & = & \dint _0 ^{2\pi}
\dint _{0} ^{\pi} \cos^2\theta\sin^3\phi \d{\phi}  \d{\theta} \\
& = & \dfrac{4\pi}{3}.
\end{array}$$
\end{Answer}
\begin{Answer}{13.7}
 Put $x
= u, y = v, z^2 = u^2 + v^2$. Then
$$\d{x} = \d{u}, \ \d{y} = \d{v}, \ z\d{z} = u\d{u} + v\d{v},    $$
whence
$$\d{x} \wedge \d{y} =  \d{u} \wedge \d{v}, \d{y} \wedge \d{z} = -\dfrac{u}{z}  \d{u} \wedge \d{v},
\d{z} \wedge \d{x} = -\dfrac{v}{z}  \d{u} \wedge \d{v},  $$ and so
$$\begin{array}{lll}\norm{\d{ ^2\vector{x}}} & = &  \sqrt{(\d{x} \wedge \d{y})^2 + (\d{z} \wedge \d{x})^2 + (\d{y} \wedge \d{z})^2} \\
& = & \sqrt{1 + \dfrac{u^2 + v^2}{z^2}}\ \d{u}\wedge\d{v} \\ &  = &
\sqrt{2}\ \d{u}\wedge\d{v}.
\end{array}$$ Hence $$\dint \limits_{\Sigma} \ z\norm{\d{ ^2\vector{x}}} =
\dint\limits_{u^2 + v^2 \leq 1} \sqrt{u^2 + v^2}\ \sqrt{2}\
\d{u}\d{v} = \sqrt{2}\dint _0 ^{2\pi}\dint _0 ^1  \rho^2\
\d{\rho}\d{\theta} = \dfrac{2\pi\sqrt{2}}{3}.
$$
\end{Answer}
\begin{Answer}{13.8}
 If the egg
has radius $R$,  each slice will have height $2R/n$. A slice can be
parametrised by $0 \leq \theta \leq 2\pi$, $\phi_1 \leq \phi \leq
\phi_2$, with $$R\cos\phi_1 - R\cos\phi_2 = 2R/n.$$ The area of the
part of the surface of the sphere in slice is $$\dint _0
^{2\pi}\dint_{\phi_1} ^{\phi_2} R^2\sin\phi \d{\phi}\d{\theta} = 2\pi
R^2(\cos\phi_1 - \cos\phi_2 ) = 4\pi R^2/n.$$ This means that each
of the $n$ slices has identical area $4\pi R^2 /n$.
\end{Answer}
\begin{Answer}{13.9}
 We project this plane
onto the coordinate axes obtaining
$$ \dint \limits_\Sigma xy\d{y}\d{z} = \dint _0 ^6 \dint _0 ^{3-z/2} (3-y-z/2)y \d{y}  \d{z} =  \frac{27}{4},  $$
$$ -\dint \limits_\Sigma x^2\d{z}\d{x} = -\dint _0 ^3 \dint _0 ^{6-2x} x^2 \d{z}  \d{x} =  -\frac{27}{2},  $$
$$ \dint \limits_\Sigma (x + z)\d{x}\d{y} = \dint _0 ^3 \dint _0 ^{3 - y} (6 - x - 2y) \d{x}  \d{y} =  \frac{27}{2},  $$
and hence
$$\dint \limits_\Sigma xy\d{y}\d{z} -
x^2\d{z} \d{x} + (x + z)\d{x} \d{y} = \frac{27}{4}.
$$
\end{Answer}
\begin{Answer}{13.10}
Evaluating this directly would result in evaluating four path
integrals, one for each side of the square. We will use Green's
Theorem. We have
$$\begin{array}{lll}\d{\omega} &  = &  \d (x^3y) \wedge \d{x}+ \d (xy) \wedge \d{y}\\
& = & (3x^2y\d{x}+ x^3\d{y}) \wedge \d{x}+ (y\d{x}+
x\d{y} ) \wedge \d{y} \\
& = & (y - x^3)\d{x}\wedge \d{y}.
\end{array}$$
The region $M$ is the area enclosed by the square. The integral
equals
$$\begin{array}{lll}
\doint _C x^3y\d{x}+ xy\d{y} & = & \dint _0 ^2 \dint _0 ^2 (y
- x^3)\d{x} \d{y} \\
& = & -4.
\end{array}$$
\end{Answer}
\begin{Answer}{13.11}
We have
\begin{dingautolist}{202} \item $L_{AB}$ is $y = x$;  $L_{AC}$  is
$y = -x$, and  $L_{BC}$ is clearly $y = -\dfrac{1}{3}x +
\dfrac{4}{3}$.
 \item  We have \renewcommand{\arraystretch}{1.5}
$${\everymath{\dis}\begin{array}{lllll}  \dint _{AB} y^2\d{x}+ x\d{y}  & = &  \dint _0 ^1 (x^2 + x)\d{x}            & = &  \dfrac{5}{6}  \\
\dint _{BC} y^2\d{x}+ x\d{y}  & = &  \dint _1 ^{-2}
\left(\left(-\dfrac{1}{3}x + \dfrac{4}{3}\right)^2 -\dfrac{1}{3}x \right)\d{x}            & = &  -\dfrac{15}{2}  \\
\dint _{CA} y^2\d{x}+ x\d{y}  & = &  \dint _{-2} ^0 (x^2 - x)\d{x}            & = &  \dfrac{14}{3}  \\
\end{array} }$$Adding these integrals we find $$ \doint _\triangle y^2\d{x}+ x\d{y} = -2.$$

 \item   We have
$${\everymath{\dis}\begin{array}{lll}\dint\limits_{D} (1 - 2y) \d{x}\wedge\d{y} & = & \dint _{-2} ^0 \left(\dint _{-x} ^{-x/3 + 4/3} (1 - 2y)
\d{y}\right) \d{x}\\ & &\quad + \dint _{0} ^1 \left(\dint _{x} ^{-x/3
+ 4/3}
(1 - 2y)\d{y}\right) \d{x} \\
& = & -\dfrac{44}{27} - \dfrac{10}{27} \\
& = & -2.
\end{array}}$$
\end{dingautolist}
\end{Answer}
\begin{Answer}{13.15}
 Observe that
$$ \d{(x^2 + 2y^3)\wedge\d{y}} = 2x\d{x}\wedge\d{y}.$$Hence by the
generalised Stokes' Theorem the integral equals $$\dint
\limits_{\{(x - 2)^2 + y^2 \leq 4\}} 2x\d{x}\wedge\d{y} = \dint
_{-\pi/2} ^{\pi/2} \dint _0 ^{4\cos\theta} 2\rho^2 \cos\theta\d{\rho}
\wedge\d\theta = 16\pi .
$$
To do it directly, put $x - 2 = 2\cos t, y = 2\sin t, 0 \leq t \leq
2\pi$. Then the integral becomes
$$\begin{array}{lll}\dint _0 ^{2\pi} ((2 + 2\cos t)^2 + 16\sin^3t)\d{2\sin t} & = & \dint _0 ^{2\pi} (8\cos t + 16\cos^2 t\\ & & \qquad  + 8\cos^3t + 32\cos t\sin^3t) \d{t}\\ & =  & 16\pi.  \end{array}  $$

\end{Answer}
\begin{Answer}{13.16}
 At the
intersection path
$$ 0 = x^2 + y^2 + z^2 - 2(x + y) = (2 - y)^2 + y^2 + z^2 - 4 = 2y^2 - 4y + z^2 = 2(y - 1)^2 + z^2 - 2,  $$
which describes an ellipse on the $yz$-plane. Similarly we get $2(x
- 1)^2 + z^2 = 2$ on the $xz$-plane.  We have
$$ \d{\left(y\d{x} + z\d{y} + x\d{z}\right)}  = \d{y}\wedge\d{x} +
\d{z}\wedge\d{y} + \d{x}\wedge\d{z} = -\d{x}\wedge\d{y}
-\d{y}\wedge\d{z} - \d{z}\wedge\d{x}.$$ Since $\d{x}\wedge \d{y} =
0$, by Stokes' Theorem the integral sought is
$$ -\dint \limits_{2(y - 1)^2 + z^2 \leq 2} \d{y}\d{z}  -\dint \limits_{2(x- 1)^2 + z^2 \leq 2} \d{z}\d{x}
= -2\pi (\sqrt{2}). $$ (To evaluate the integrals you may resort to
the fact that  the area of the elliptical region
$\dfrac{(x-x_0)^2}{a^2} + \dfrac{(y-y_0)^2}{b^2} \leq 1$ is $\pi
ab)$.

\bigskip

If we were to evaluate this integral directly, we would set $$  y =
1 + \cos \theta, \ z =  \sqrt{2}\sin \theta , x = 2 - y = 1 -
\cos\theta .
$$The integral becomes
$$ \dint _0 ^{2\pi} (1 + \cos \theta)\d{(1 - \cos \theta)} + \sqrt{2}\sin\theta\d{(1 + \cos\theta)} + (1 - \cos\theta)\d{(\sqrt{2}\sin\theta)} $$
which in turn $$ = \dint_0 ^{2\pi} \sin\theta + \sin\theta\cos\theta
- \sqrt{2} + \sqrt{2}\cos\theta \d{\theta} = -2\pi\sqrt{2}.$$
\end{Answer}
\begin{Answer}{B.2.18}
 This is clearly
$(1\ 2\ 3 \ 4)(6\ 8\ 7) $ of order ${\textrm  lcm} (4, 3) = 12$.
\end{Answer}
\begin{Answer}{B.3.26}
 Multiplying the first column of the given matrix  by $a$,
its second column by $b$, and its third column by $c$, we obtain
$$abc\Omega =  \begin{bmatrix} abc & abc & abc \cr a^2 & b^2 & c^2 \cr a^3 & b^3 & c^3 \cr\end{bmatrix}. $$ We may factor out
$abc$ from the first row of this last matrix thereby obtaining
$$abc\Omega = abc\det
\begin{bmatrix} 1 & 1 & 1 \cr a^2 & b^2 & c^2 \cr a^3 & b^3 & c^3
\cr\end{bmatrix}. $$Upon dividing by $abc$,
$$\Omega = \det
\begin{bmatrix} 1 & 1 & 1 \cr a^2 & b^2 & c^2 \cr a^3 & b^3 & c^3
\cr\end{bmatrix}.
$$
\end{Answer}
\begin{Answer}{B.3.27}
 Performing $R_1 + R_2 + R_3 \rightarrow R_1$ we have
$$\begin{array}{l}
\Omega = \det\begin{bmatrix} a - b - c & 2a & 2a \cr 2b & b - c -
a & 2b \cr 2c & 2c & c - a - b \cr
\end{bmatrix}\vspace{2mm}\\ \qquad = \det\begin{bmatrix} a + b + c & a + b + c & a + b + c \cr 2b
& b - c - a & 2b \cr 2c & 2c & c - a - b \cr
\end{bmatrix}.\end{array}$$ Factorising $(a + b + c)$ from the first row of
this last determinant, we have
$$\Omega =   (a + b + c)\det\begin{bmatrix} 1 & 1 & 1 \cr 2b
& b - c - a & 2b \cr 2c & 2c & c - a - b \cr
\end{bmatrix}. $$Performing $C_2 - C_1 \rightarrow C_2$ and $C_3 - C_1 \rightarrow
C_3$,
$$\Omega =   (a + b + c)\det\begin{bmatrix} 1 & 0 & 0 \cr 2b
& -b - c - a & 0 \cr 2c & 0 & -c - a - b \cr
\end{bmatrix}. $$This last matrix is triangular, hence $$\Omega = (a + b + c)(-b - c - a)(-c -a - b) = (a + b + c)^3,  $$
as wanted.
\end{Answer}
\begin{Answer}{B.3.28}
 $\det A_1 = \det A = -540$ by multilinearity. $\det A_2 = -\det
 A_1 = 540$ by alternancy. $\det A_3 = 3\det A_2 = 1620$  by both
 multilinearity and homogeneity from one column. $\det A_4 = \det A_3
 = 1620$ by multilinearity, and $\det A_5 = 2\det A_4 = 3240$ by
 homogeneity from one column.
 
\end{Answer}
\begin{Answer}{B.3.30}
 From the given data, $\det B = -2.$ Hence
 $$\det ABC = (\det A)(\det B)(\det C) = -12,$$
 $$\det 5AC = 5^3\det AC = (125)(\det A)(\det C) = 750,$$
 $$(\det A^3B^{-3}C^{-1}) = \frac{(\det A)^3}{(\det B)^3(\det C)} = -\frac{27}{16}.$$
\end{Answer}
\begin{Answer}{B.3.31}
 Pick $\lambda \in \bbR \setminus \{0,
a_{11}, a_{22}, \ldots , a_{nn}\}$. Put
$$X = \begin{bmatrix} a_{11} - \lambda & 0 & 0 & \cdots &  0 \cr a_{21}  & a_{22} - \lambda & 0 & \cdots & 0 \cr
a_{31}  & a_{32}  & a_{33} - \lambda & \cdots &  0 \cr \vdots &
\vdots & \vdots & \vdots & \vdots \cr a_{n1} & a_{n2} & a_{n3} &
\cdots & a_{nn} - \lambda
\end{bmatrix}$$and
$$Y = \begin{bmatrix}
\lambda & a_{12} & a_{13} & \vdots & a_{1n} \cr   0 & \lambda &
a_{23} & \vdots & a_{2n} \cr
 0 & 0 &
\lambda & \vdots & a_{3n} \cr \vdots & \vdots & \vdots & \vdots &
\vdots \cr
 0 & 0 &
0 & \vdots & \lambda \cr


\end{bmatrix}$$ Clearly  $A = X + Y$, $\det X = (a_{11} - \lambda)(a_{22} -
\lambda)\cdots (a_{nn} - \lambda) \neq 0$, and $\det Y = \lambda^n
\neq 0$. This completes the proof.
\end{Answer}
\begin{Answer}{B.3.32}
No.
\end{Answer}
\begin{Answer}{B.4.8}
 We have
\begin{eqnarray*}\det A  & = & 2(-1)^{1 + 2}\det
\begin{bmatrix} 4 & 6 \cr 7 & 9 \end{bmatrix} + 5(-1)^{2 + 2}\det
\begin{bmatrix} 1 & 3 \cr 7 & 9 \end{bmatrix} + 8(-1)^{2 + 3}\det
\begin{bmatrix} 1 & 3 \cr 4 & 6 \end{bmatrix} \\ &  =  & -2(36 - 42) +
5(9 - 21) - 8(6 - 12) = 0. \end{eqnarray*}
\end{Answer}
\begin{Answer}{B.4.9}
 Simply expand along the first row$$a\det\begin{bmatrix} a & b
\cr c & a
\end{bmatrix} - b\det\begin{bmatrix}
c & b \cr b & a
\end{bmatrix} + c\det\begin{bmatrix}
c & a \cr b & c\end{bmatrix} = a(a^2 -bc ) - b(ca - b^2) + c(c^2 -
ab) = a^3+b^3+c^3-3abc .
$$
\end{Answer}
\begin{Answer}{B.4.10}
 Since the second column has three $0$'s, it is advantageous
to expand along it, and thus we are reduced to calculate
$$-3(-1)^{3 + 2} \det \begin{bmatrix}
                1 & -1 & 1 \cr
                2 & 0 & 1 \cr
                1 & 0 & 1 \cr
\end{bmatrix}$$
Expanding this last determinant along the second column, the
original determinant is thus
$$-3(-1)^{3 + 2}(-1)(-1)^{1 + 2}\det \begin{bmatrix}
            2 & 1 \cr
            1 & 1 \cr
\end{bmatrix} = -3(-1)(-1)(-1)(1) = 3.$$
\end{Answer}
\begin{Answer}{B.4.12}
 Expanding along the first column,
$$\begin{array}{lll} 0 & = &  \det\begin{bmatrix} 1 & 1 & 1 & 1 \cr x & a & 0 & 0 \cr
x & 0 & b & 0 \cr x & 0 & 0 & c\cr \end{bmatrix} \\
& = & \det\begin{bmatrix}a & 0 & 0 \cr 0 & b & 0 \cr 0 & 0 & c \cr
\end{bmatrix} - x\det\begin{bmatrix} 1 & 1 & 1 \cr 0 & b & 0 \cr 0
& 0 & c \cr \end{bmatrix} \\ & & \qquad  + x\det\begin{bmatrix}1 &
1 & 1 \cr a & 0 & 0 \cr 0 & 0 & c \cr  \end{bmatrix}
- x\det\begin{bmatrix} 1 & 1 & 1 \cr a & 0 & 0 \cr  0 & b & 0 \cr \end{bmatrix} \\
& = & xabc - xbc + x\det\begin{bmatrix}1 & 1 & 1 \cr a & 0 & 0 \cr
0 & 0 & c \cr  \end{bmatrix} - x\det\begin{bmatrix} 1 & 1 & 1 \cr
a & 0 & 0 \cr  0 & b & 0 \cr \end{bmatrix}.\\ \end{array}$$
Expanding these  last  two determinants along the third row,
$$\begin{array}{lll} 0 & = & abc - xbc
+ x\det\begin{bmatrix}1 & 1 & 1 \cr a & 0 & 0 \cr 0 & 0 & c \cr
\end{bmatrix}
- x\det\begin{bmatrix} 1 & 1 & 1 \cr a & a & 0 \cr  0 & b & 0 \cr \end{bmatrix}\\
& = &   abc - xbc + xc\det \begin{bmatrix} 1 & 1 \cr  a & 0 \cr\end{bmatrix}  + xb\det \begin{bmatrix} 1 & 1 \cr  a & 0 \cr\end{bmatrix} \\
& = & abc - xbc - xca -xab.
     \end{array}$$It follows that $$abc = x(bc + ab + ca), $$whence$$\frac{1}{x} = \frac{bc + ab + ca}{abc} = \frac{1}{a} +\frac{1}{b} + \frac{1}{c},  $$as wanted.

\end{Answer}
\begin{Answer}{B.4.14}
 Expanding along the first row the determinant equals
$$\begin{array}{lll}-a\det\begin{bmatrix} a & b & 0 \cr
0 & 0 & b \cr 1 & 1 & 1 \cr \end{bmatrix} + b\det\begin{bmatrix} a
& 0 & 0 \cr 0 & a & b \cr 1 & 1 & 1 \cr\end{bmatrix}  & = &  ab
\det\begin{bmatrix} a & b \cr 1 & 1  \cr
\end{bmatrix} + ab\det\begin{bmatrix} a & b \cr 1 & 1  \cr
\end{bmatrix}  \\ & =  & 2ab (a - b), \end{array}$$as wanted.
\end{Answer}
\begin{Answer}{B.4.15}
 Expanding along the first row, the determinant equals
$$ a\det\begin{bmatrix} a & 0 & b \cr 0 & d & 0 \cr c & 0 & d \cr  \end{bmatrix}
+ b \det\begin{bmatrix} 0 & a & b \cr c & 0 & 0 \cr 0 & c & d \cr
\end{bmatrix}.
$$Expanding the resulting two determinants along the second row,
we obtain $$ad\det\begin{bmatrix}a & b \cr c & d\cr \end{bmatrix}
+ b(-c)\det\begin{bmatrix}a & b \cr c & d\cr  \end{bmatrix} =
ad(ad - bc) - bc(ad - bc) = (ad - bc)^2,
$$as wanted.
\end{Answer}
\begin{Answer}{B.4.16}
 For $n = 1$ we
have $\det (1) = 1 = (-1)^{1 + 1}$. For $n = 2$ we have
$$\det \begin{bmatrix} 1 & 1 \cr 1 & 0 \cr
\end{bmatrix} = -1 = (-1)^{2 + 1}.
$$Assume that the result is true for $n -1$. Expanding the
determinant along the first column
$$ \begin{array}{lll}\det\begin{bmatrix} 1 & 1 & 1 & \cdots & 1 &
1 \cr  1& 0 & 0 & \vdots & 0  & 0 \cr  0 & 1 & 0 & \cdots & 0  & 0
\cr 0 & 0 & 1& \cdots & 0  & 0 \cr \vdots & \vdots & \cdots &
\vdots & \vdots \cr 0 & 0 & 0 & \cdots & 1 & 0 \cr
\end{bmatrix} &  = & 1\det\begin{bmatrix}  0 & 0 & \vdots & 0  & 0 \cr   1 & 0 & \cdots & 0  & 0
\cr  0 & 1& \cdots & 0  & 0 \cr  \vdots & \cdots & \vdots & \vdots
\cr  0 & 0 & \cdots & 1 & 0 \cr
\end{bmatrix} \\ & & \qquad - 1 \det\begin{bmatrix}  1 & 1 & \cdots & 1 &
1 \cr  1& 0 & \cdots & \vdots & 0   \cr  0 & 1 & \cdots & \cdots &
0 \cr 0 & 0 & \cdots & \cdots & 0   \cr \vdots & \vdots & \cdots &
\vdots \cr 0 & 0 & \cdots & 1 & 0 \cr
\end{bmatrix}\\
& = & 1(0) - (1)(-1)^{n}\\
& = &  (-1)^{n + 1},\\ \end{array}$$giving the result.
\end{Answer}
\begin{Answer}{B.4.17}
Perform $C_k - C_1 \rightarrow C_k$ for $k \in [2; n]$. Observe that
these operations do not affect the value of the determinant. Then
$$\det A = \det\begin{bmatrix} 1 & n - 1 & n - 1 & n - 1 & \cdots & n - 1 \cr n & 2 - n &
0 & 0 & \vdots & 0 \cr n & 0 & 3 - n & 0 & \cdots & 0 \cr n & 0 &
0 & 4-n & \cdots & 0 \cr \vdots & \vdots & \vdots & \cdots &
\vdots \cr n & 0 & 0 & 0 & 0 & 0 \cr
\end{bmatrix}.
$$Expand this last determinant along the $n$-th row, obtaining,
$$\begin{array}{lll}\det A & = &  (-1)^{1 + n}n\det\begin{bmatrix}  n - 1 & n - 1 & n - 1 & \cdots & n - 1 & n - 1 \cr  2 - n &
0 & 0 & \vdots & 0  & 0 \cr  0 & 3 - n & 0 & \cdots & 0  & 0 \cr 0
& 0 & 4-n & \cdots & 0  & 0 \cr  \vdots & \vdots & \cdots & \vdots
& \vdots \cr 0 & 0 & 0 & \cdots & -1 & 0  \cr
\end{bmatrix} \\
& = & (-1)^{1 + n}n(n - 1)(2 - n)(3 - n)\\ & & \qquad \cdots
(-2)(-1) \det\begin{bmatrix}  1 & 1 & 1 & \cdots & 1 & 1 \cr  1& 0
& 0 & \vdots & 0  & 0 \cr  0 & 1 & 0 & \cdots & 0  & 0 \cr 0 & 0 &
1& \cdots & 0  & 0 \cr  \vdots & \vdots & \cdots & \vdots & \vdots
\cr 0 & 0 & 0 & \cdots & 1 & 0  \cr
\end{bmatrix} \\
& = & -(n!)\det\begin{bmatrix}  1 & 1 & 1 & \cdots & 1 & 1 \cr  1&
0 & 0 & \vdots & 0  & 0 \cr  0 & 1 & 0 & \cdots & 0  & 0 \cr 0 & 0
& 1& \cdots & 0  & 0 \cr  \vdots & \vdots & \cdots & \vdots &
\vdots \cr 0 & 0 & 0 & \cdots & 1 & 0  \cr
\end{bmatrix} \\
& = & -(n!)(-1)^{n} \\
& = & (-1)^{n + 1}n!,
\end{array}$$
upon using the result of problem
\ref{exa:determinant_bunch_of_ones}.
\end{Answer}
\begin{Answer}{B.4.18}
 Recall
that $\binom{n}{k} = \binom{n}{n - k}$, $$ \sum _{k = 0} ^n
\binom{n}{k} = 2^n  $$ and$$ \sum _{k = 0} ^n (-1)^{k}\binom{n}{k}
= 0, \ \ \ \ {\textrm  if}\ \ n > 0. $$ Assume that $n$ is odd. Observe
that then there are $n + 1$  (an even number) of columns and that
on the same row, $\binom{n}{k}$ is on a column of opposite parity
to that of $\binom{n}{n - k}$. By performing  $C_1 - C_2 + C_3 -
C_4 + \cdots + C_n - C_{n + 1} \rightarrow C_1$, the first column
becomes all $0$'s, whence the determinant if $0$ if $n$ is odd.
\end{Answer}
\begin{Answer}{B.4.22}
I will prove that
$$\det \begin{bmatrix}  (b+c)^2 &  ab & ac \cr
ab & (a+c)^2 & bc\cr ac & bc & (a+b)^2 \cr
\end{bmatrix} = 2abc(a+b+c)^3. $$
Using permissible row and column operations,
$$\begin{array}{lll}
\det \begin{bmatrix}  (b+c)^2 &  ab & ac \cr ab & (a+c)^2 & bc\cr ac
& bc & (a+b)^2 \cr
\end{bmatrix} & = & \det \begin{bmatrix}  b^2+2bc+c^2 &  ab & ac \cr ab & a^2+2ca+c^2 & bc\cr ac
& bc & a^2+2ab+b^2 \cr
\end{bmatrix}\\
& = \grstep{C_1+C_2+C_3\to C_1} & \det
\begin{bmatrix} b^2+2bc+c^2+ab+ac & ab & ac \cr ab+a^2+2ca+c^2+bc & a^2+2ca+c^2 & bc\cr
ac+bc+a^2+2ab+b^2 & bc & a^2+2ab+b^2 \cr
\end{bmatrix}\\
& =  & \det
\begin{bmatrix} (b+c)(a+b+c) & ab & ac \cr (a+c)(a+b+c) & a^2+2ca+c^2 & bc\cr
(a+b)(a+b+c) & bc & a^2+2ab+b^2 \cr
\end{bmatrix}\\
\end{array}$$
Pulling out a factor, the above equals
$$
 (a+b+c)\det
\begin{bmatrix} b+c & ab & ac \cr a+c & a^2+2ca+c^2 & bc\cr
a+b & bc & a^2+2ab+b^2 \cr
\end{bmatrix}
$$ and performing $R_1+R_2+R_3\to R_1$, this is  $$ (a+b+c)\det
\begin{bmatrix} 2a+2b+2c & ab+a^2+2ca+c^2+bc & ac+bc+ a^2+2ab+b^2 \cr a+c & a^2+2ca+c^2 & bc\cr
a+b & bc & a^2+2ab+b^2 \cr
\end{bmatrix}$$
Factoring this is
$$ (a+b+c)\det
\begin{bmatrix} 2(a+b+c) & (a+c)(a+b+c) & (a+b)(a+b+c) \cr a+c & a^2+2ca+c^2 & bc\cr
a+b & bc & a^2+2ab+b^2 \cr
\end{bmatrix},$$which in turn is  $$(a+b+c)^2\det
\begin{bmatrix} 2 & a+c & a+b \cr a+c & a^2+2ca+c^2 & bc\cr
a+b & bc & a^2+2ab+b^2 \cr
\end{bmatrix}$$
Performing $C_2-(a+c)C_1\to C_2$ and $C_3-(a+b)C_1\to C_3$ we obtain
 $$(a+b+c)^2\det
\begin{bmatrix} 2 & -a-c & -a-b \cr a+c & 0 & -a^2-ab-ac\cr
a+b &-a^2- ab-ac & 0 \cr
\end{bmatrix}$$
This last matrix we will expand by the second column, obtaining that
the original determinant is thus
$$ (a+b+c)^2\left((a+c)\det\begin{bmatrix}a+c & -a^2-ab-ac \cr  a+b & 0  \end{bmatrix} +(a^2+ab+ac)\det\begin{bmatrix} 2 & -a-b \cr a+c & -a^2-ab-ac \end{bmatrix}\right)  $$
This simplifies to
$$\begin{array}{lll}
(a+b+c)^2\left((a+c)(a+b)(a^2+ab+ac\right.)\\
\qquad \left.+(a^2+ab+ac)(-a^2-ab-ac+bc)\right) & = &
a(a+b+c)^3((a+c)(a+b)-a^2-ab-ac+bc)\\ & = &
2abc(a+b+c)^3,\end{array}
$$ as claimed.




\end{Answer}
\begin{Answer}{B.4.23}
 We have

 \begin{eqnarray*}\det \begin{bmatrix} a & b & c & d \cr d & a & b
& c \cr c & d & a & b \cr b & c & d & a\cr         \end{bmatrix} &
\grstep[=]{R_1+R_2+R_3+R_4\to R_1} & \det \begin{bmatrix} a+b+c+d &
a+b+c+d & a+b+c+d & a+b+c+d \cr d & a & b & c \cr c & d & a & b \cr
b & c & d & a\cr
\end{bmatrix}\\
& = & (a+b+c+d)\det \begin{bmatrix} 1 & 1 & 1 & 1 \cr d & a & b & c
\cr c & d & a & b \cr b & c & d & a\cr         \end{bmatrix}\\
& \grstep[=]{C_4-C_3+C_2-C_1\to C_4} & (a+b+c+d)\begin{bmatrix} 1 &
1 & 1 & 0\cr d & a & b & c-b+a-d \cr c & d & a & b-a+d-c \cr b & c &
d & a-d+c-b\cr
\end{bmatrix}\\
& = &  (a+b+c+d)(a-b+c-d)\begin{bmatrix} 1 & 1 & 1 & 0\cr d & a & b
& 1\cr c & d & a & -1 \cr b & c & d & 1\cr
\end{bmatrix}\\
& \grstep[=]{R_2+R_3\to R_2,\ R_4+R_3\to R_4} &
(a+b+c+d)(a-b+c-d)\begin{bmatrix} 1 & 1 & 1 & 0\cr d+c & a+d & b+a &
0\cr c & d & a & -1 \cr b+c & c+d & a+d & 0\cr
\end{bmatrix}\\
&= & (a+b+c+d)(a-b+c-d)\begin{bmatrix} 1 & 1 & 1 \cr d+c & a+d & b+a
\cr  b+c & c+d & a+d \cr
\end{bmatrix}\\
& \grstep[=]{C_1-C_3\to C_1, \ C_2-C_3\to C_2} &
(a+b+c+d)(a-b+c-d)\begin{bmatrix} 0 & 0 & 1 \cr d+c-b-a & d-b & b+a
\cr b+c-a-d & c-a
 & a+d \cr
\end{bmatrix}\\
& = & (a+b+c+d)(a-b+c-d)\begin{bmatrix}  d+c-b-a & d-b \cr b+c-a-d &
c-a
 \cr
\end{bmatrix}\\
& = & (a+b+c+d)(a-b+c-d) (d+c-b-a)(c-a)-(d-b)(b+c-a-d)
\\
& = & (a+b+c+d)(a-b+c-d)\\
& & \qquad
((c-a)(c-a)+(c-a)(d-b)-(d-b)(c-a)-(d-b)(b-d))
\\
& = & (a+b+c+d)(a-b+c-d)((a-c)^2+(b-d)^2).
\end{eqnarray*}
\begin{rem}
Since $$ (a-c)^2+(b-d)^2=(a-c+i(b-d))(a-c-i(b-d)), $$ the
above determinant is then
$$ (a+b+c+d)(a-b+c-d)(a+ib-c-id)(a-ib-c+id). $$
Generalisations of this determinant are possible using roots of
unity.
\end{rem}
\end{Answer}
